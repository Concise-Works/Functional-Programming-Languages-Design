\section{Introduction}

Programming Languages (PL) from the perspective of a programmer can be thought of as:
\begin{itemize}
    \item  A tool for programming
    \item A text-based way of interacting
    with hardware/a computer
    \item A way of organizing and working
    with data
\end{itemize}

However \textbf{\underline{This text concerns the design of 
PLs}}, not the sole use of them. It's the difference between knowing how to fly 
an aircraft vs. designing one.
We instead \textbf{\underline{think in terms of mathematics}}, describing and defining
the specifications of our language. Our program some mathematical object,
a function with strict inputs and outputs.

\vspace{1em}
\begin{Def}[Well-formed Expression]

    An expression (sequence of symbols) that is constructed according to established rules (syntax),
    ensuring clear and unambiguous meaning.

\end{Def}

\begin{Def}[Programming Language]

    A \textbf{Programming Language (PL)} consists of three main components:
    \begin{itemize}
        \item \textbf{Syntax:} Specifies the rules for constructing well-formed expressions or programs.
        \item \textbf{Type System:} Defines the properties and constraints of possible data and expressions.
        \item \textbf{Semantics:} Provides the meaning and behavior of programs or expressions during evaluation.
    \end{itemize}
\end{Def}

\noindent
I.e., Syntax gives us meaning, Types tell us how it is used, and Semantics tell us what it does. 
Here is an example of defining the operator ($+$) for addition in a language:
\begin{Example}[Syntax for Addition]

    \label{ex:well_formed_expr}
    If $e_1$ is a well-formed expression and $e_2$ is a well-formed expression, 
    then $e_1 + e_2$ is also a well-formed expression. 
\end{Example}

\newpage

However, we can be a bit more concise using mathematic notation:

\begin{Def}[Production Rule (\texttt{::=})]
    
    \label{def:production_rule}
    A \textbf{Production Rule} defines the syntax of a language by specifying how non-terminal symbols can be expanded into
    sequences of terminal and non-terminal symbols. It is denoted by the symbol \texttt{::=}:
    \[
    \langle \text{non-terminal symbol} \rangle ::= \langle \text{definition} \rangle
    \]
    Where the left-hand side non-terminal symbol can be expanded/represented by the right-hand side definition.
    We may also define multiple rules for a single non-terminal symbol, separated by the pipe symbol (\texttt{|}):

    \[
    \langle e_1 \rangle ::= \langle e_2 \rangle | \langle e_3 \rangle | \dots | \langle e_n \rangle
    \]

    Where $\langle e_1 \rangle$ can be expanded into $\langle e_2 \rangle$, $\langle e_3 \rangle$, $\dots$, or $\langle e_n \rangle$. 
\end{Def}

\begin{Example}[Production Rule]

    Here are some possible production rules:
    \begin{itemize}
        
        \item $\langle \text{date} \rangle ::= \langle \text{month} \rangle / \langle \text{year} \rangle$
        \item $\langle \text{year} \rangle ::=  \ 2020 \ | \ 2021 \ | \ 2022 \ | \ 2023 \ | \ 2024 \ | \ 2025$
        \item $\langle \text{month} \rangle ::= 1 \ | \ 2 \ | \ 3 \ | \ 4 \ | \ 5 \ | \ 6 \ | \ 7 \ | \ 8 \ | \ 9 \ | \ 10 \ | \ 11 \ | \ 12$
        
        \item $\langle \text{OS} \rangle ::= \langle \text{Linux} \rangle \ | \ \langle \text{Windows} \rangle \ | \ \langle \text{MacOS} \rangle$
    \end{itemize}

    \noindent
    \textbf{Incorrect Derivations}: we cannot take a terminal symbol and expand it further:
    \begin{itemize}
        \item $8 \Rightarrow \langle \text{number} \rangle$ 
        \item $8 \Rightarrow 5 + \langle \text{number} \rangle$
        \item $8 \Rightarrow 5 + 3$
    \end{itemize}

    \noindent
    Here $8$ means the token $8$, it cannot be expanded any further.
\end{Example}

\noindent
Now we can clean up our previous syntax for defining addition:

\begin{Example}[Production Rule for Addition]

    \label{ex:production_rule_addition}
    Let $\langle \text{expr} \rangle$ be a non-terminal symbol representing a well-formed expression. Then,
    \vspace{-.5em}
    \[\langle \text{expr} \rangle ::= \langle \text{expr} \rangle + \langle \text{expr} \rangle\]

    \vspace{-.5em}
    \noindent
    I.e., ``$\langle \text{expr} \rangle + \langle \text{expr} \rangle$'' (right-hand side) 
    is a valid ``$\langle \text{expr} \rangle$'' (left-hand side).
\end{Example}

% We can formalize this using the following rule for expressions, 
%     where $\langle \text{expr} \rangle$ acts as a placeholder for arbitrary expressions:

%     \[
%     \langle \text{expr} \rangle \ ::= \ \langle \text{expr} \rangle + \langle \text{expr} \rangle
%     \]
\newpage

\noindent
Now that we have defined syntax, we have to define its usage using types. We first cover variables, which have 
a slightly different meaning in this context than what most programmers are used to.
\begin{Def}[Meta-variables]

    Meta-variables are placeholders that represent arbitrary expressions in a formal syntax.
    They are used to generalize the structure of expressions or programs within a language.
\end{Def}

\vspace{-1em}
\begin{Example}[Meta-variables:]

    An expression $e$ could be represented as $3$ (a literal) or $3 + 4$ (a compound expression).
    In this context, variables serve as shorthand for expressions rather than as containers for mutable data.
\end{Example}

\noindent
Now we must understand ``\textbf{context}'' and ``\textbf{judgments}'' in the context of type theory:

\begin{Def}[Judgments]

    In type theory, a \textbf{judgment} is a formal assertion about an expression. This does not 
    have to be a true assertion, but just a statement about the expression. For example: ``Pigs can fly''
    is a judgment, regardless of its validity.
\end{Def}

\vspace{-.5em}
\begin{Def}[Context and Typing Environment]

    In type theory, a \textbf{context} defines an environment which establishes data types for variables.
    Say we have an environment $\Gamma$ (Gamma) for which defines the context. This context $\Gamma$ holds 
    an ordered list of pairs. Say, $\langle x : \tau \rangle$, typically written as $x : \tau$,
    where $x$ is a variable and $\tau$ (tau) is its type. We denote our judgment as such:
    \LARGE

    \vspace{-1em}
    \[\Gamma \vdash e: \tau\]

    \normalsize 
    \noindent
    which reads ``in the context $\Gamma$, the expression $e$ has type $\tau$''. 
    We may also write judgments for functions, denoting the type of the function and its arguments.
    
    \vspace{-1em}
    \Large
    \[ f : \tau_1, \tau_2, \ldots, \tau_n \rightarrow \tau \]
    \normalsize
    \noindent 
    where $f$ is a function taking $n$ arguments ($\tau_1, \tau_2, \ldots, \tau_n$), outputting type $\tau$. We may add 
    temporary variables to the context. For example: 
    \[\Gamma, x : \text{int}\vdash e : \text{bool}\] 
    Reads,
    Given the context $\Gamma$ with variable declaration ($x:$int) added, the expression $e$ has type \textbf{bool}. \hfill \cite{WikipediaTypingEnvironment}
\end{Def}

\newpage 

\begin{Def}[Rule of Inference]

    In formal logic and type theory, an \textbf{inference rule} provides a formal structure for deriving conclusions from premises. 
    Rules of inference are usually presented in a \textbf{standard form}: \Large

    \[
    \frac{\text{Premise}_1, \quad \text{Premise}_2, \quad \ldots, \quad \text{Premise}_n}{\text{Conclusion}} \ (\text{Name})
    \]

    \normalsize
    \begin{itemize}
        \item \textbf{Premises (Numerator):} The conditions that must be met for the rule to apply.
        \item \textbf{Conclusion (Denominator):} The judgment derived when the premises are satisfied.
        \item \textbf{Name (Parentheses):} A label for referencing the rule. \hfill \cite{wiki:rule_of_inference}
    \end{itemize}
\end{Def}

\noindent
Now we may begin to create a type system for our language, starting with some basic rules.
\begin{Example}[Typing Rule for Integer Addition]

    \label{ex:integer_addition}
    Consider the typing rule for integer addition for which the inference rule is written as: \LARGE

    \[
    \frac{\Gamma \vdash e_1 : \text{int} \quad \Gamma \vdash e_2 : \text{int}}{\Gamma \vdash e_1 + e_2 : \text{int}} \ (\text{addInt})
    \]

    \vspace{.5em}
    \normalsize
    \noindent
    This reads as, ``If $e_1$ is an \textbf{int} (in the context $\Gamma$) and $e_2$ is an \textbf{int} 
    (in the context $\Gamma$), then $e_1 + e_2$ is an \textbf{int} (in the same context $\Gamma$)''.

    \vspace{1em}
    \noindent
    \textbf{Therefore}: let $\Gamma = \{x : \text{int}, y : \text{int}\}$. Then the expression $x + y$ is well-typed as an \textbf{int}, 
    since both $x$ and $y$ are integers in the context $\Gamma$.
\end{Example}

\vspace{-1em}
\begin{Example}[Typing Rule for Function Application]

    \label{ex:function_application}
    
    If $f$ is a function of type $\tau_1 \rightarrow \tau_2$ and $e$ is of type $\tau_1$, then $f(e)$ is of type $\tau_2$.
    \LARGE

    \[
    \frac{\Gamma \vdash f : \tau_1 \rightarrow \tau_2 \quad \Gamma \vdash e : \tau_1}{\Gamma \vdash f(e) : \tau_2} \ (\text{appFunc})
    \]

    \vspace{.5em}
    \normalsize
    \noindent
    This reads as, ``If $f$ is a function of type $\tau_1 \rightarrow \tau_2$ (in the context $\Gamma$) and $e$ is of type $\tau_1$ 
    (in the context $\Gamma$), then $f(e)$ is of type $\tau_2$ (in the same context $\Gamma$)''.

    \vspace{1em}
    \noindent
    \textbf{Therefore}: let $\Gamma = \{f : \text{int} \rightarrow \text{bool}, x : \text{int}\}$. Then the expression, $f(x)$, 
    is well-typed as a \textbf{bool}, since $f$ is a function that takes an integer and returns a boolean, and $x$ is an integer 
    in the context $\Gamma$.
\end{Example}

\newpage 

\noindent
Finally, we can define the semantics of our language, which describes the behavior of programs during evaluation:

\begin{Example}[Evaluation Rule for Integer Addition (Semantics)]

    \label{ex:eval_integer_addition}
    Consider the evaluation rule for integer addition. This rule specifies how the sum of two expressions is computed. 
    If $e_1$ evaluates to the integer $v_1$ and $e_2$ evaluates to the integer $v_2$, 
    then the expression $e_1 \texttr{ + } e_2$ evaluates to the integer $v_1 + v_2$. The rule is written as: \LARGE

    \[
    \frac{e_1 \Downarrow v_1 \quad e_2 \Downarrow v_2}{e_1 \texttr{ + } e_2 \Downarrow v_1 + v_2} \ (\text{evalInt})
    \]

    \vspace{.5em}
    \normalsize
    \noindent
    Read as, ``If $e_1$ evaluates to the integer $v_1$ and $e_2$ evaluates to the integer $v_2$, 
    then the expressions, $e_1 \texttr{ + } e_2$, evaluates to $v_1 + v_2$.''

    \vspace{1em}
    \textbf{Example Evaluation:}
    \begin{itemize}
        \item $2 \Downarrow 2$
        \item $3 \Downarrow 3$
        \item $2 + 3 \Downarrow 5$
        \item $4 + 5 \Downarrow 9$
        \item $(2 + 4) + (4 + 5) \Downarrow 15$
    \end{itemize}

    \noindent
    Here, the integers $2$ and $3$ evaluate to themselves, and their sum evaluates to $5$ based on the evaluation rule.
    Additionally $e_1$ could be a compound expression, such as $(2 + 4)$, which evaluates to $6$.
\end{Example}


