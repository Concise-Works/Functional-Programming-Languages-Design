\section{Development Environment with OCaml}

In this section, we introduce \textbf{OCaml} as our programming language of choice for exploring
the principles of \textbf{functional programming}. Functional programming emphasizes a declarative
style, where programs describe \textit{what to do} rather than \textit{how to do it},
in contrast to the imperative programming paradigm, which many programmers are familiar with.

To better understand these differences, we will compare functional programming in OCaml with
imperative programming in Python. This will give us a practical perspective on how the two
paradigms approach problem-solving and help us appreciate the unique features of functional programming.

\newpage

\begin{Def}[OCaml]

	\textbf{OCaml} is a general-purpose programming language from the ML family,
	known for its strong static type system, type inference, and support for functional,
	imperative, and object-oriented programming. It is widely used in areas like compilers,
	financial systems, and formal verification due to its safety, performance,
	and expressive syntax. \underline{ The \textbf{Ocaml Extension} is \snippet{.ml}}
\end{Def}

\noindent
In addition to using Ocaml we will use Dune and Opam.
\begin{Def}[Dune]

	\textbf{Dune} is a build system for \textbf{OCaml} projects, designed to simplify
	the compilation and management of code. It automates tasks such as building executables,
	libraries, and tests, while handling dependencies efficiently. Dune is widely used in the
	OCaml ecosystem due to its ease of use and minimal configuration.

\end{Def}

\begin{Def}[OPAM]

	\textbf{OPAM} (OCaml Package Manager) is the standard package manager for the OCaml programming
	language. It simplifies the installation, management, and sharing of OCaml libraries and tools,
	providing developers with a convenient way to manage dependencies and project environments.
\end{Def}

\noindent
If you are familiar with \textbf{npm} or \textbf{yarn}, \textbf{OPAM} serves a similar purpose but is specifically designed for the \textbf{OCaml} ecosystem. Like npm and yarn, OPAM is a package manager that simplifies the installation and management of libraries and dependencies. Additionally, it offers features tailored to OCaml development, such as managing multiple compiler versions and isolating project environments.



\begin{Note}
	\underline{\textbf{Window Users:}} It may be easier to use WSL or a Linux VM to run OCaml and Dune rather than a native install.
	This text will use \textbf{Ubuntu} distro. If using WSL, make sure the terminal is running the distro, it will give you a fresh
	file system to work with. If you are a Mac user, you may use \textbf{Homebrew} to install OCaml and Dune.

	\noindent
	\textbf{WSL Installation:} \href{https://learn.microsoft.com/en-us/windows/wsl/setup/environment}{https://learn.microsoft.com/en-us/windows/wsl/setup/environment}
\end{Note}

\begin{Tip}
	If you plan to use github as your repository manager, you may have to create a personal access token to connect your account to your local machine.\\
	\textbf{Creating a Personal Access Token:} \href{https://docs.github.com/en/authentication/keeping-your-account-and-data-secure/managing-your-personal-access-tokens#creating-a-personal-access-token-classic}{https://docs.github.com/en/authentication/kee...}
\end{Tip}

\newpage

\noindent
We use the terminal in this text, but an IDE could be used with additional setup.

\begin{Def}[Basic Terminal Commands]
	
	\begin{itemize}
		\item \textbf{Navigation:}
		      \begin{itemize}
			      \item \snippet{cd <directory>}: Change to a specified directory.
			      \item \snippet{cd ~}: Navigate to the home directory.
			      \item \snippet{cd ../}: Move up one level in the directory hierarchy.
			      \item \snippet{pwd}: Print the current working directory.
		      \end{itemize}

		\item \textbf{Viewing and Listing:}
		      \begin{itemize}
			      \item \snippet{ls}: List the contents of the current directory.
			      \item \snippet{ls -l}: Display detailed information about files and directories.
			      \item \snippet{cat <file>}: Display the contents of a file.
			      \item \colorbox{OliveGreen!20}{\snippet{tree} <directory>}:
				  Prettier \snippet{ls -l}, install: \colorbox{OliveGreen!20}{\snippet{sudo apt install tree}}.
		      \end{itemize}

		\item \textbf{Creating:}
		      \begin{itemize}
			      \item \snippet{mkdir <directory>}: Create a new directory.
			      \item \snippet{touch <file>}: Create an empty file.
		      \end{itemize}

		\item \textbf{Deleting:}
		      \begin{itemize}
			      \item \snippet{rm <file>}: Delete a file.
			      \item \snippet{rm -r <directory>}: Delete a directory and its contents recursively.
		      \end{itemize}

		\item \textbf{Renaming and Moving:}
		      \begin{itemize}
				  \item \snippet{mv file.txt /path/to/new/directory/}
			      \item \snippet{mv <oldname> <newname>}: Rename or move a file.
		      \end{itemize}
		\item \textbf{File Properties:}
			  \begin{itemize}
			  \item \snippet{chmod <permissions> <file>}: Change the permissions of a file.
			  \item \snippet{chmod u+rwx file.txt}: Gives \snippet{u} (owner) \textbf{r}ead, \textbf{w}rite, and e\textbf{x}ecute permissions. 
			  \item \snippet{chmod g-w file.txt}: Removes \snippet{g} (group) \textbf{w}rite permission.
			  \item \snippet{file <file>}: Determine the type of a file.
			  \end{itemize}
		
	\end{itemize}
	\vspace{1em}
\end{Def}

\newpage 
\noindent
\textbf{Vim} will be our text-editor of choice. We will write code, and edit files using Vim.

\begin{Def}[Vim Common Commands]

\begin{itemize}
    \item \textbf{Starting and Exiting:}
        \begin{itemize}
            \item \snippet{vim <file>}: Open or a create a file in Vim.
            \item \snippet{:w}: Save (write) changes to the file.
            \item \snippet{:q}: Quit Vim.
            \item \snippet{:wq}: Save changes and quit Vim.
            \item \snippet{:q!}: Quit without saving changes.
        \end{itemize}
    
    \item \textbf{Modes:}
        \begin{itemize}
            \item \snippet{i}: Switch to \textbf{\textit{Insert Mode}} to start editing text.
            \item \snippet{Esc}: Return to \textbf{\textit{Normal Mode}}, read-only mode for \textbf{navigation} and \textbf{commands}.
        \end{itemize}
    
    \item \textbf{Navigation:}
        \begin{itemize}
            \item \snippet{h} (left), \snippet{j} (down), \snippet{k} (up), \snippet{l} (right): Moves the cursor.
            \item \snippet{:<line number>}: Jump to a specific line in the file.
            \item \snippet{G}: Jump to the end of the file.
            \item \snippet{gg}: Jump to the beginning of the file.
        \end{itemize}
    
    \item \textbf{Editing:}
        \begin{itemize}
            \item \snippet{x}: Delete the character under the cursor.
            \item \snippet{dd}: Delete the current line.
            \item \snippet{yy}: Copy (yank) the current line.
            \item \snippet{p}: Paste copied or deleted text.
            \item \snippet{u}: Undo the last change.
            \item \snippet{Ctrl+r}: Redo the undone change.
        \end{itemize}
    
    \item \textbf{Searching:}
        \begin{itemize}
            \item \snippet{/text}: Search for \texttt{text} in the file.
            \item \snippet{n}: Jump to the next occurrence of the search term.
            \item \snippet{N}: Jump to the previous occurrence of the search term.
        \end{itemize}
\end{itemize}

\vspace{1em}
\snippet{:help}: for more Vim commands and options.
\end{Def}






\newpage


\vspace{-1em}
\subsection{Preparing the Environment}
\noindent
Next we enable our machine to compile and run OCaml code. Choose a line below that
corresponds to your operating system, and run it in the terminal.
\begin{lstlisting}[language=Bash, caption={Installing OPAM on Various Systems}, numbers=left]
    # Homebrew (macOS)
    brew install opam
    
    # MacPort (macOS)
    port install opam
    
    # Ubuntu
    apt install opam
    
    # Debian
    apt-get install opam
    
    # Arch Linux
    pacman -S opam
\end{lstlisting}

\noindent
Before we can use OPAM to manage OCaml libraries and tools, we need to prepare the system by running the \snippet{opam init} command. This sets up OPAM by:

\begin{lstlisting}[language=Bash, caption={Initializing OPAM}]
    # Initialize OPAM
    opam init
    
    # Configure your shell environment
    eval $(opam env)
    
    # Verify OPAM is ready to use
    opam --version
\end{lstlisting}

\noindent
After these steps, OPAM will be ready to manage OCaml dependencies, compilers, and project environments.
\begin{Note}
	\underline{\textbf{Important:}} With every new terminal, \snippet{eval \$(opam env)} must be ran for OCaml use.
	Without it, the terminal might not recognize OPAM-installed tools or compilers.
\end{Note}

\subsection{Creating and Using an OPAM Switch}

To manage different versions of OCaml and keep project dependencies isolated, OPAM provides a feature called a \textbf{switch}.
A switch is an environment tied to a specific OCaml compiler version and a unique set of installed packages. This is especially
useful for working on multiple projects with different requirements.

\newpage
\noindent
For this setup, we will create a new switch to ensure a clean environment with the required version of OCaml.
Follow these steps:

\begin{lstlisting}[language=Bash, caption={Creating and Activating an OPAM Switch}]
    # Step 1: Create a new switch named "my_switch" with OCaml version 5.2.1
    opam switch create my_switch 5.2.1
    
    # Step 2: Activate the newly created switch
    opam switch my_switch
    
    # Step 3: Update your terminal environment to reflect the switch
    eval $(opam env)
    
    # Step 4: Verify the switch is active
    opam switch
    
    # (Or / Optionally) Check the OCaml version
    ocaml -version
\end{lstlisting}

\noindent
Once these commands are executed, your terminal will be configured to use the OCaml version and environment
defined by the switch \snippet{my\_switch}.

\subsection{Updating OPAM and Installing Essential Packages}
After initializing OPAM and creating a switch, the next step is to update OPAM's package repository and install
the tools we’ll need for development. These packages provide essential utilities for OCaml programming and project management.
Run the following commands:

\begin{lstlisting}[language=Bash, caption={Updating OPAM and Installing Packages}]
# Step 1: Update OPAM to fetch the latest package information
opam update

# Step 2: Install essential development tools
opam install dune utop ounit2 menhir ocaml-lsp-server

# Step 3: Install the custom library for this course
opam install stdlib320/.
\end{lstlisting}

\noindent
Here's what each package does:
\begin{itemize}
	\item \snippet{dune}: A modern build system for OCaml projects. It automates the compilation and management of OCaml code.
	\item \snippet{utop}: A user-friendly OCaml REPL (Read-Eval-Print Loop) for testing and experimenting with OCaml code interactively.
	\item \snippet{ounit2}: A testing framework for OCaml, similar to JUnit for Java, used for writing and running unit tests.
	\item \snippet{menhir}: A parser generator for OCaml, often used for developing compilers and interpreters.
	\item \snippet{ocaml-lsp-server}: A Language Server Protocol (LSP) implementation for OCaml, enabling features like autocompletion, type inference, and error checking in editors.
	\item \snippet{stdlib320/}: A custom library created for the CS320 course at Boston University by Nathan Mull. It provides
	      It's a very small subset of the OCaml Standard Library with a bit more documentation. Documentation: \href{https://nmmull.github.io/CS320/landing/Spring-2025/Specifications/Stdlib320/index.html}{https://nmmull.github.io/CS320/...}.
\end{itemize}

\noindent
These will be the main tools used throughout this text.

\subsection{Creating a Dune Project: Hello, World!}

To understand how \snippet{dune} structures projects and facilitates OCaml development, we'll create a simple project called \snippet{hello\_dune}. This hands-on example will demonstrate the purpose of each folder and guide you through building, running, and testing an OCaml project.

\subsubsection{Step 1: Prepare Your Environment}

Before starting, ensure OPAM and your environment are set up. Run the following command to prepare the shell:

\begin{lstlisting}[language=Bash, caption={Preparing Your OPAM Environment}]
    eval $(opam env)
\end{lstlisting}

\noindent
This ensures that your terminal is configured correctly to work with OCaml and \snippet{dune}.

\subsubsection{Step 2: Create the Project}

\noindent
Run the following commands to create a new \snippet{dune} project called \snippet{hello\_dune}:

\begin{lstlisting}[language=Bash, caption={Creating the Project}]
    mkdir demo # Create a new folder named hello_dune for our project
    cd demo    # Move into the project directory
    dune init project hello_dune # Initialize a new dune project
\end{lstlisting}

\noindent
This will generate the following project structure inside the \snippet{demo} folder:
\begin{lstlisting}[language=Bash, caption={Generated Project Structure}]
    hello_dune/
    |-- bin/         # Contains the main executable code
    |-- lib/         # Contains reusable library code
    |-- test/        # Contains test code
    |-- dune-project         # Defines the project
    |-- hello_dune.opam      # OPAM package definition
\end{lstlisting}

\noindent
For now, we will focus on the \snippet{bin/} and \snippet{lib/} folders.

\subsubsection{Step 3: Build and Verify the Project}

To ensure everything is set up correctly, use the following command to build the project:

\begin{lstlisting}[language=Bash, caption={Building the Project}]
    dune build
\end{lstlisting}

\noindent
\textbf{This command:}
\begin{itemize}
    \item Compiles the OCaml source files in your project.
    \item Resolves dependencies and ensures libraries and executables are built in the correct order.
    \item Creates a build cache to speed up future builds.
    \item Verifies that your project is configured correctly.
\end{itemize}

\noindent
\textbf{Important Notes:}
\begin{itemize}
    \item You must run \snippet{dune build} every time you make changes to your code to ensure the build reflects your edits.
    \item Running \snippet{dune build} from any subdirectory within redirect to the project root and build.
    \item If there are any issues (e.g., syntax errors, missing files, or incorrect configurations), \snippet{dune} will report them.
\end{itemize}

\subsubsection{Step 4: Modify and Run the Program}

\noindent
To modify the program, first open the file \snippet{bin/main.ml} using Vim:

\begin{lstlisting}[language=Bash, caption={Opening the File in Vim}]
    vim bin/main.ml
\end{lstlisting}

\noindent
This opens the \underline{\textbf{main executable file}} in the Vim editor. Once the file is open, press \snippet{i} to switch to \textit{Insert Mode} and replace its contents with the following code:

\begin{lstlisting}[language=OCaml, caption={Hello, Dune Program}]
    let () = print_endline "Hello, Dune!"
\end{lstlisting}

\noindent
After editing, press \snippet{Esc} to return to \textit{Normal Mode}, then type \snippet{:wq} to save the changes and exit Vim.
Now, run the program using the following command:

\begin{lstlisting}[language=Bash, caption={Running the Program}]
    dune exec ./bin/main.exe
\end{lstlisting}

\noindent
You should see the output:
\begin{lstlisting}[language=Bash]
    Hello, Dune!
\end{lstlisting}


\subsubsection{Step 5: Add a Library and Explore Its Use}

The \snippet{lib/} folder is reserved for reusable code that can be shared across different parts of a project. 
In object-oriented programming languages like Java, this is analogous to creating static utility classes 
(e.g., a \texttt{Math} class for reusable mathematical functions).\\

\noindent
\textbf{Steps to Add and Use the Library:}\\

\noindent
1. Create a new file in the \snippet{lib/} folder. \underline{\textbf{Important:}} The name of the file must match the project name. 
   If your project is named \snippet{hello\_dune}, the file should be named:
   \begin{lstlisting}[language=Bash]
   vim lib/hello_dune.ml
   \end{lstlisting}

\vspace{.5em}
\noindent
2. Add a reusable function to \snippet{lib/hello\_dune.ml} (\snippet{ \^} concats strings, \snippet{+} is strictly for integers):
   \begin{lstlisting}[language=OCaml]
   let greet name = "Hello, " ^ name ^ "!"
   \end{lstlisting}

\vspace{.5em}
\noindent
3. Verify or update the \snippet{lib/dune} file to expose the library. \underline{The \snippet{name} in the \snippet{dune} file}\\
   \underline{should also match the project name:}
   \begin{lstlisting}[language=PlainText]
   (library
    (name hello_dune))
   \end{lstlisting}

   \noindent If this file is already configured with the above content, no changes are needed.\\

\vspace{.5em}
\noindent
4. Interactively use the library in \snippet{utop}:
   \begin{lstlisting}[language=Bash]
   dune utop
   \end{lstlisting}

   \noindent Once inside \snippet{utop}, you can interact with the library:
   \begin{lstlisting}[language=OCaml, caption={Using the Library in Utop}]
   Hello_dune.greet "Testing123";;
   \end{lstlisting}

   \noindent
   You should see the output: 

    \begin{lstlisting}[language=OCaml]
    - : string = "Hello, Testing123!".
    \end{lstlisting}
   \vspace{.5em}
   \noindent
   \underline{\textbf{Important}}: Despite \snippet{lib/hello\_dune.ml} being lowercase, it's referenced as \snippet{Hello\_dune} in utop (capitalized).
   More on \snippet{utop} will be discussed later. But you may think of it as a calculator where we can access our functions and libraries.\\

\noindent
5. To quit \snippet{utop}, type \snippet{\#quit;;} or press \snippet{Ctrl+d}.
   \begin{lstlisting}[language=OCaml, caption={Quitting utop}]
   #quit;;
   \end{lstlisting}

\newpage
\noindent
6. We may also modify \snippet{bin/main.ml} to use the library:
   \begin{lstlisting}[language=OCaml, caption={Using the Library in Main}]
   let () = print_endline (Hello_dune.greet "Library")
   \end{lstlisting}

\noindent
7. Build and run the program:
   \begin{lstlisting}[language=Bash]
   dune build
   dune exec ./bin/main.exe
   \end{lstlisting}

   \noindent The output should now be:
   \begin{lstlisting}[language=Bash]
   Hello, Library!
   \end{lstlisting}

\vspace{1em}
\noindent
\textbf{What Are Dune Files?}\\
As you explore the project, you’ll notice \snippet{dune} files in various folders such as \snippet{bin/} and \snippet{lib/}. These files 
are configuration files used by the \textit{Dune build system} to manage how your project is compiled and linked.\\

\noindent
1. \textbf{Dune File in \snippet{lib/}:}
   \begin{lstlisting}[language=PlainText, caption={Library Dune File}]
   (library
    (name hello_dune))
   \end{lstlisting}

   \noindent This file defines the \snippet{hello\_dune} library. Dune compiles the code in \snippet{lib/hello\_dune.ml} into a reusable 
   module named \snippet{Hello\_dune}, which can be used in other parts of the project.\\

\noindent
2. \textbf{Dune File in \snippet{bin/}:}
   \begin{lstlisting}[language=PlainText, caption={Executable Dune File}]
   (executable
    (public_name hello_dune)
    (name main)
    (libraries hello_dune))
   \end{lstlisting}

   \noindent This file specifies the executable program:
   \begin{itemize}
       \item \snippet{public\_name hello\_dune}: Defines the name of the program, which you can run with \snippet{dune exec hello\_dune}.
       \item \snippet{name main}: Points to \snippet{bin/main.ml}, which serves as the entry point.
       \item \snippet{libraries hello\_dune}: Links the \snippet{hello\_dune} library to the executable.
   \end{itemize}

\noindent

\newpage

\subsubsection{Step 6: Add Tests}

To test the library, we use the \snippet{test/} folder:

1. Create a new test file:
\begin{lstlisting}[language=Bash]
   touch test/test_hello.ml
   \end{lstlisting}

2. Add the following code to \snippet{test/test\_hello.ml}:
\begin{lstlisting}[language=OCaml, caption={Test Code}]
   let () =
     let open Alcotest in
     check string "same message" "Hello from the library!" Hello.message
   \end{lstlisting}

3. Update the \snippet{test/dune} file to include the test:
\begin{lstlisting}[language=PlainText, caption={Test Dune File}]
   (test
    (name test_hello)
    (libraries hello alcotest))
\end{lstlisting}

4. Run the tests:
\begin{lstlisting}[language=Bash]
   dune runtest
\end{lstlisting}

If everything is set up correctly, the test will pass.

\subsubsection{Summary of Roles}
- \textbf{\snippet{bin/}:} Contains the main executable code (e.g., \snippet{main.ml}).
- \textbf{\snippet{lib/}:} Contains reusable library code.
- \textbf{\snippet{test/}:} Contains tests to verify the project works as expected.

This step-by-step guide demonstrates how \snippet{dune} organizes projects and how each folder contributes to OCaml development. By creating and running your own project, you'll gain a deeper understanding of the tools and structure used in OCaml programming.


