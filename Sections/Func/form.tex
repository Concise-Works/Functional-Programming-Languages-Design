\section{Formalizing Ocaml Expressions}
\subsection{Basic Ocaml Expressions}
Now we can begin to formalize expressions
in OCaml. We again re-iterate what steps are needed to build expressions
in our language, given that we have some \textit{context} now.

\begin{Def}[Building Expressions]

    When creating new expressions we must follow these steps:
    \begin{enumerate}
        \item \textbf{Context:} Define variable-to-type mappings.
        \item \textbf{Syntax:} Establish how the expression/operation should be written.
        \item \textbf{Typing Rules:} Define the type of the whole expression and its sub-expressions.
        \item \textbf{Semantics:} Clarify the resulting value/evaluation of the defined expression.
    \end{enumerate}
    I.e., what are our types, how are they used, what type of data do they represent, and how does it evaluate?
\end{Def}

\noindent
Now we begin to formalize, though we will abstract the context to $\Gamma$, assuming 
all the types we've defined before (\ref{tab:ocaml-types}).
\begin{Def}[Formalizing Let-Expressions]

    Let $\Gamma$ be the OCaml context, and $=$ be mathematical equality, and $\texttr{=}$ be an OCaml token:
    \begin{itemize}
        \item \textbf{Syntax:} \LARGE $\langle$\text{expr}$\rangle::=$ \texttr{let} $\langle$\text{var}$\rangle$ \texttr{=} $\langle$\text{expr}$\rangle$ \texttr{in} $\langle$\text{expr}$\rangle$ \normalsize\\
        
        \vspace{-.5em}
        \noindent
        If $x$ is a valid variable name, and $e_1$ is a well-formed expression and $e_2$ is a well-formed expression. Then $\texttr{let }x\texttr{ = }e_1\texttr{ in }e_2$ is also a well-formed expression.
        \item \textbf{Typing-Rule:} \LARGE $\dfrac{\Gamma\vdash e_1:\tau_1 \quad \Gamma,x:\tau_1\vdash e_2:\tau}{\Gamma\vdash\texttr{let }x\texttr{ = }e_1\texttr{ in }e_2:\tau}$ \normalsize\\
        
        \noindent
        Given context $\Gamma$, if there's some well-formed expression $e_1$ of type $\tau_1$ and some well-formed expression $e_2$ of type $\tau$, given a variable declaration of $x$ of type $\tau_1$, 
        then within this context, the expression $\texttr{let }x\texttr{  = }e_1\texttr{ in }e_2$ is of type $\tau$.
        
        \item \textbf{Semantics:} \LARGE $\dfrac{e_1\Downarrow v_1 \quad [v_1/x]e_2\Downarrow v}{\texttr{let }x\texttr{ = }e_1\texttr{ in }e_2\Downarrow v}$ \normalsize\\
        
        \noindent
        Following our context $\Gamma$, if a well-formed expression $e_1$ evaluates to $v_1$ and the substitution of $v_1$ for variable $x$ in another 
        well-formed expression $e_2$ evaluates to $v$, then the expression $\texttr{let }x\texttr{ = }e_1\texttr{ in }e_2$ evaluates to $v$.
    \end{itemize}
\end{Def}

\newpage 

\noindent
Before we continue, we introduce the concept of $\top$ and $\bot$.

\begin{Def}[Top and Bottom ($\top$, $\bot$)]
    
    In logic and computer science:
    \begin{itemize}
        \item $\top$ is used to represent \textit{true}, \textit{valid}.
        \item $\bot$ is used to represent \textit{false} or \textit{invalid}.
    \end{itemize}

    \noindent
    Specifically, they are the greatest and least element of a lattice/boolean algebra (hence top and bottom), which when it comes to logic means truthhood and falsehood.
\end{Def}

\noindent
We continue with the formalization of the \texttt{if} expression in OCaml.
\begin{Def}[Formalizing If-Expressions]
    
    Let $\Gamma$ be the OCaml context, then:
    \begin{itemize}
        \item \textbf{Syntax:} \LARGE $\langle$\text{expr}$\rangle::=$ \texttr{if} $\langle$\text{expr}$\rangle$ \texttr{then} $\langle$\text{expr}$\rangle$ \texttr{else} $\langle$\text{expr}$\rangle$ \normalsize\\
     
        \vspace{-.5em}
        \noindent
        If $e_1$ is a well-formed expression, $e_2$ is a well-formed expression, and $e_3$ is a well-formed expression, then $\texttr{if }e_1\texttr{ then }e_2\texttr{ else }e_3$ is also a well-formed expression.

        \item \textbf{Typing-Rule:} \LARGE $\dfrac{\Gamma\vdash e_1:\texttt{bool} \quad \Gamma\vdash e_2:\tau \quad \Gamma\vdash e_3:\tau}{\Gamma\vdash\texttr{if }e_1\texttr{ then }e_2\texttr{ else }e_3:\tau}$ \normalsize\\

        \noindent
        Given context $\Gamma$, let there be well-formed expressions, $e_1$ of type \texttt{bool}, $e_2$ of type $\tau$, and $e_3$ of type $\tau$. Then the expression $\texttr{if }e_1\texttr{ then }e_2\texttr{ else }e_3$ is of type $\tau$.

        \item \textbf{Semantics:} \LARGE $\dfrac{e_1\Downarrow \top \quad e_2\Downarrow v}{\texttr{if }e_1\texttr{ then }e_2\texttr{ else }e_3\Downarrow v}$ (trueCond.) \normalsize\\
        
        \noindent
        Following our context $\Gamma$, if a well-formed expression $e_1$ evaluates $\top$ and another well-formed expression $e_2$ evaluates to $v$, then 
        the expression $\texttr{if }e_1\texttr{ then }e_2\texttr{ else }e_3$ evaluates to $v$ ($e_3$ a well-formed expression).
        \item \textbf{Semantics:} \LARGE $\dfrac{e_1\Downarrow \bot \quad e_3\Downarrow v}{\texttr{if }e_1\texttr{ then }e_2\texttr{ else }e_3\Downarrow v}$ (falseCond.) \normalsize\\
        
        \noindent
        Following our context $\Gamma$, if a well-formed expression $e_1$ evaluates $\bot$ and another well-formed expression $e_3$ evaluates to $v$, then
        the expression $\texttr{if }e_1\texttr{ then }e_2\texttr{ else }e_3$ evaluates to $v$ ($e_2$ a well-formed expression).
    \end{itemize}

    \noindent
    Take note that we must write two semantics rules for the \texttt{if} expression, one for when the condition evaluates to $\top$ and one for when it evaluates to $\bot$.
\end{Def}

\newpage 

\noindent
We continue with the formalization of the \texttt{function} expression in OCaml.
\begin{Def}[Formalizing Functions]

    \label{def:func}
    
    Let $\Gamma$ be the OCaml context, then:
    \begin{itemize}
        \item \textbf{Syntax:} \LARGE $\langle$expr$\rangle::=$\texttr{ fun }$\langle$var$\rangle$\texttr{ -> }$\langle$expr$\rangle$\normalsize\\

        \vspace{-.5em}
        \noindent
        If $x$ is a valid variable name and $e$ is a well-formed expression, then $\texttr{fun }x\texttr{ -> }e$ is also a well-formed expression.
        
        \item \textbf{Typing-Rule:} \LARGE $\dfrac{\Gamma,x:\tau_1\vdash e:\tau_2}{\Gamma\vdash\texttr{fun }x\texttr{ -> }e:\tau_1\rightarrow\tau_2}$ \normalsize
        
        \noindent
        Given context $\Gamma$ with a variable declaration of $(x:\tau_1)$ added, if there's a well-formed expression $e$ of type $\tau_2$ and, then the expression $\texttr{fun }x\texttr{ -> }e$ is of type $\tau_1\rightarrow\tau_2$.

        \item \textbf{Semantics:} \LARGE $\dfrac{}{\texttr{fun }x\texttr{ -> }e\Downarrow\lambda x.e}$ \normalsize\\
        
        \noindent
        Under no premises, the expression $\texttr{fun }x\texttr{ -> }e$ evaluates to the lambda function $\lambda x.e$.
    \end{itemize}
\end{Def}

\begin{Def}[Formalizing Application] 

    Let $\Gamma$ be the OCaml context, then:
    \begin{itemize}
        \item \textbf{Syntax:} \LARGE $\langle$expr$\rangle::=$\texttr{ }$\langle$expr$\rangle$\texttr{ }$\langle$expr$\rangle$\normalsize\\
        
        \vspace{-.5em}
        \noindent
        If $e_1$ is a well-formed expression and $e_2$ is a well-formed expression, then $e_1\ e_2$ is also a well-formed expression.

        \item \textbf{Typing-Rule:} \LARGE $\dfrac{\Gamma\vdash e_1:\tau_1\texttr{ -> }\tau_2 \quad \Gamma\vdash e_2:\tau_1}{\Gamma\vdash e_1\ e_2:\tau_2}$ \normalsize\\
        
        \noindent
        Given context $\Gamma$, if there's a well-formed expression $e_1$ of type $\tau_1$\texttr{ -> }$\tau_2$ (Functions.\ref{def:func}) and a well-formed expression $e_2$ of type $\tau_1$, then the expression $e_1\ e_2$ is of type $\tau_2$.

        \item \textbf{Semantics:} \LARGE $\dfrac{e_1\Downarrow\lambda x.e \quad e_2\Downarrow v \quad [v/x]e\Downarrow v'}{e_1\ e_2\Downarrow v'}$ \normalsize\\
        
        \noindent
        Following our context $\Gamma$, if a well-formed expression $e_1$ evaluates to a lambda function $\lambda x.e$, another well-formed expression $e_2$ evaluates to $v$, and the substitution of $v$ for $x$ in $e$ evaluates to $v'$, then the expression $e_1\ e_2$ evaluates to $v'$.
    \end{itemize}
\end{Def}

\newpage 

\noindent
Onto tuples and matching:

\begin{Def}[Formalizing Tuples]

    Let $\Gamma$ be the OCaml context, then:
    \begin{itemize}
        \item \textbf{Syntax:} \LARGE $\langle$expr$\rangle::=$\texttr{ (}$\langle$expr$\rangle$\texttr{,}$\dots$\texttr{,}$\langle$expr$\rangle$\texttr{)\normalsize}\normalsize\\
        
        \vspace{-.5em}
        \noindent
        If $e_1$ is a well-formed expression and $e_2$ is a well-formed expression, then $(e_1,e_2)$ is also a well-formed expression.

        \item \textbf{Typing-Rule:} \LARGE $\dfrac{\Gamma\vdash e_1:\tau_1 \quad \Gamma\vdash e_2:\tau_2 \quad \dots \quad \Gamma\vdash e_n:\tau_n}{\Gamma\vdash(e_1,e_2,\dots,e_n):\tau_1*\tau_2*\dots*\tau_n}$ \normalsize\\
        
        \noindent
        Given context $\Gamma$, if there are well-formed expressions $e_1$ of type $\tau_1$, $e_2$ of type $\tau_2$, and $e_n$ of type $\tau_n$, then the expression $(e_1,e_2,\dots,e_n)$ is of type $\tau_1*\tau_2*\dots*\tau_n$.

        \item \textbf{Semantics:} \LARGE $\dfrac{e_1\Downarrow v_1 \quad e_2\Downarrow v_2 \quad \dots \quad e_n\Downarrow v_n}{(e_1,e_2,\dots,e_n)\Downarrow(v_1,v_2,\dots,v_n)}$ \normalsize\\
        
        \noindent
        Following our context $\Gamma$, if well-formed expressions $e_1$ evaluates to $v_1$, $e_2$ evaluates to $v_2$, and $e_n$ evaluates to $v_n$, then the expression $(e_1,e_2,\dots,e_n)$ evaluates to $(v_1,v_2,\dots,v_n)$.
    \end{itemize}
\end{Def}

\begin{Def}[Formalizing Lists]

    Let $\Gamma$ be the OCaml context, then:
    \begin{itemize}
        \item \textbf{Syntax:} \LARGE 
        $\langle$expr$\rangle::=$ \texttr{[]} $\mid$ $\langle$expr$\rangle\ \texttr{::}\ \langle$expr$\rangle$ \normalsize\\
        
        \vspace{-.5em}
        \noindent
        The empty list \texttr{[]} is a well-formed expression. If $e_1$ is a well-formed expression and $e_2$ is a well-formed list, then $e_1\ \texttr{::}\ e_2$ is also a well-formed expression.

        \item \textbf{Typing-Rule:} \large \quad 
            $\dfrac{}{\Gamma\vdash []:\tau\ \texttr{list}}$ (nil)
            \quad
             $\dfrac{\Gamma\vdash e_1:\tau \quad \Gamma\vdash e_2:\tau\ \texttr{list}}{\Gamma\vdash e_1\ \texttr{::}\ e_2:\tau\ \texttr{list}}$ (cons) 
      

        \normalsize
        \noindent
        Given context $\Gamma$, the empty list \texttr{[]} has type $\tau\ \texttr{list}$ for any type $\tau$ (nil).
        If $e_1$ is of type $\tau$ and $e_2$ is of type $\tau\ \texttr{list}$, then the expression $e_1\ \texttr{::}\ e_2$ has type $\tau\ \texttr{list}$ (cons).

        \item \textbf{Semantics:} \large \quad
        $\dfrac{}{\texttr{[]} \Downarrow \varnothing}$ (nilEval)
        \quad $\dfrac{e_2 \Downarrow [v_2, ..., v_k] \quad e_1 \Downarrow v_1}{e_1\ \texttr{::}\ e_2 \Downarrow [v_1, v_2, ..., v_k]}$ (consEval) \normalsize\\
        
        \noindent
        The empty list \texttr{[]} evaluates to the empty list as a value (nilEval).
        If $e_2$ evaluates to the list $[v_2, ..., v_k]$ and $e_1$ evaluates to $v_1$, then $e_1\ \texttr{::}\ e_2$ evaluates to the list $[v_1, v_2, ..., v_k]$ (consEval).
    \end{itemize}
\end{Def}

\newpage

\noindent
Matching in a general sense is complex (deep matching), we can simplify with weak matching:

\begin{Def}[Weak Matching]

    Weak matching is a form of pattern matching that is for specific cases.
\end{Def}

\noindent
Additionally, we introduce the concept of \textit{side conditions} before we jump into weak matching on lists.

\begin{Def}[Side Conditions in Formal Semantics]

    A \textbf{side condition} is an additional constraint that must be satisfied before applying a rule. Side conditions are used to prevent undefined behavior and ensure correctness in evaluation.

    \textbf{Example 1: Integer Division Rule}

    \begin{itemize}
        \item Consider the evaluation rule for integer division:

        \LARGE
        \[
        \dfrac{e_1 \Downarrow v_1 \quad e_2 \Downarrow v_2 \quad v_2 \neq 0}
        {e_1 \div e_2 \Downarrow v_1 \div v_2}
        \]
        \normalsize

        \item The side condition \LARGE $v_2 \neq 0$ \normalsize ensures that:
        \begin{itemize}
            \item The denominator $v_2$ is not zero before performing division.
            \item If $v_2 = 0$, the rule cannot be applied to avoid division by zero.
        \end{itemize}

        \item Without this side condition, the expression could cause an error or undefined behavior.
    \end{itemize}

    \textbf{Example 2: Exponentiation with Non-Negative Exponents}

    \begin{itemize}
        \item Consider the evaluation rule for exponentiation:

        \LARGE
        \[
        \dfrac{e_1 \Downarrow v_1 \quad e_2 \Downarrow v_2 \quad v_2 \geq 0}
        {e_1^{e_2} \Downarrow v_1^{v_2}}
        \]
        \normalsize

        \item The side condition \LARGE $v_2 \geq 0$ \normalsize ensures that:
        \begin{itemize}
            \item The exponent $v_2$ is non-negative before applying the exponentiation operation.
            \item If $v_2 < 0$, the rule cannot be applied to avoid undefined results in integer arithmetic.
        \end{itemize}

        \item Without this side condition, expressions like $2^{-3}$ would be invalid in integer arithmetic.
    \end{itemize}

    Side conditions help enforce correctness by restricting operations to only valid inputs.

\end{Def}



\newpage

\begin{Def}[Weak Matching on Lists]

    Let $\Gamma$ be the OCaml context, then:
    \begin{itemize}
        \item \textbf{Syntax:} \LARGE 
        $\langle$expr$\rangle::=$ \texttr{match }$\langle$expr$\rangle$\texttr{ with }\\
        \text{}\hspace{7.5em}\texttr{| [] -> }$\langle$expr$\rangle$\\
        \text{}\hspace{7.5em}\texttr{| }$\langle$var$\rangle$ \texttr{:: }$\langle$var$\rangle$ \texttr{-> }$\langle$expr$\rangle$ \normalsize\\

        \vspace{-1em}
        \noindent
        If $e, e_1, e_2$ are well-formed expressions and $x, y$ are valid variable names, then\\
        $\texttr{match } e \texttr{ with | [] -> } e_1 \texttr{| } x \texttr{ :: } y \texttr{ -> } e_2$
        is a well-formed expression.

        \item \textbf{Typing Rule:}\\ \LARGE 
        
        $\dfrac{\Gamma\vdash e:\tau' \texttr{ list} \quad \Gamma\vdash e_1:\tau \quad \Gamma, x:\tau', y:\tau' \texttr{ list} \vdash e_2:\tau}
        {\Gamma\vdash \texttr{match } e \texttr{ with | [] -> } e_1 \texttr{| } x \texttr{ :: } y \texttr{ -> } e_2 : \tau}$ \normalsize\\
        
        \noindent
        If $e$ is of type $\tau' \texttr{ list}$ in the context $\Gamma$, and $e_1$ is of type $\tau$ in the context $\Gamma$, and $e_2$ is of type $\tau$ in the context $\Gamma$ with $(x:\tau')$ and $(y:\tau' \texttr{ list})$ added, then the entire match expression is of type $\tau$.

        \item \textbf{Semantics:}\\ 
        \LARGE 
            $\dfrac{e \Downarrow \varnothing \quad e_1 \Downarrow v}{\texttr{match } e \texttr{ with} \texttr{ [] -> } e_1 \texttr{| } x \texttr{:: } y \texttr{ -> } e_2 \Downarrow v}$ (nil) \normalsize\\
            
            \noindent
            If $e$ evaluates to the empty list $\varnothing$ and $e_1$ evaluates to $v$, then the entire match expression evaluates to $v$.

        \LARGE 
            $\dfrac{e \Downarrow h \texttr{ :: } t \quad e'_2 = [t/y][h/x]e_2 \quad e'_2 \Downarrow v}
            {\texttr{match } e \texttr{ with | [] -> } e_1 \texttr{| } x \texttr{ :: } y \texttr{ -> } e_2 \Downarrow v}$ (cons) \normalsize\\
            
            \noindent
            If $e$ evaluates to a nonempty list $h::t$ with first element $h$ and remainder $t$, and the expression $e_2$ with $h$ substituted for $x$ and $t$ substituted for $y$ evaluates to $v$, then the entire match expression evaluates to $v$.
       
    \end{itemize}
\end{Def}

\noindent
\subsection{All Ocaml Formalizations}
\noindent Below is the full list from which we will reference throughout the text.
\begin{gbox}
    \textbf{Full Specifications:} For a full list of all the formalized expressions we'll be using. This list 
    also includes the next topic we'll discuss \textbf{Derivations}:\\
    \href{https://nmmull.github.io/PL-at-BU/320Caml/notes.html}{https://nmmull.github.io/PL-at-BU/320Caml/notes.html}
\end{gbox}

\newpage

\subsection{Derivations}

\noindent
Derivations allow us to unpack the formal expressions to prove their validity.
\begin{Def}[Tree Derivations]

    \textbf{Tree derivations} are a structured way of representing step-by-step reasoning in formal systems. They are often used in type systems, operational semantics, and logic proofs to show how conclusions follow from premises.\\

    \noindent
    Each derivation is represented as a \textbf{tree}, where:
    \begin{itemize}
        \item \textbf{Leaves} represent axioms or base cases.
        \item \textbf{Internal nodes} apply inference rules to derive new conclusions.
        \item \textbf{The root} represents the final conclusion of the derivation, i.e., the starting point.
    \end{itemize}
\end{Def}

\vspace{-1em}
\begin{Example}[Typing Derivation]

    Say we wanted to prove the typing derivation: $\texttt{let y = 2 in y + y} : \texttt{int}$

    \begin{center}
    \begin{prooftree}
        % Leftmost branch for the integer literal
        \Infer0[(intLit)]{\{\} \vdash \textcolor{purple}{2} : \textcolor{purple}{\texttt{int}}}
    
        % Middle branch for variable y
        \Infer0[(var)]{\{ \textcolor{purple}{y : \texttt{int}} \} \vdash \textcolor{purple}{y} : \textcolor{purple}{\texttt{int}}}
    
        % Rightmost branch for variable y
        \Infer0[(var)]{\{ \textcolor{purple}{y : \texttt{int}} \} \vdash \textcolor{purple}{y} : \textcolor{purple}{\texttt{int}}}
    
        % Adding the two y's together
        \Infer2[(intAdd)]{\{ \textcolor{purple}{y : \texttt{int}} \} \vdash \textcolor{purple}{y + y} : \textcolor{purple}{\texttt{int}}}
    
        % Let-binding y = 2
        \Infer2[(let)]{\{\} \vdash \textcolor{purple}{\texttt{let y = 2 in y + y} : \texttt{int}}}
    \end{prooftree}\\
\end{center}

    \vspace{1em}
    \noindent
    Here the bottom of the tree is the final conclusion. We now unpack the highest level wrapper expression. The first expression 
    we encounter is the \snippet{let} expression. The syntax of which are $(e_3::=\texttt{let }x\texttt{ = }e_1\texttt{ in }e_2)$. Hence we 
    split into two branches to examine $e_1$ and $e_2$. The left branch 
    examines the integer literal. The right branch looks at \snippet{(y + y)}. Since \snippet{(y + y)} is an addition operation, we must unravel once 
    more into two branches examining both sides of the expression. The left and right branch examines the variable as an integer. Tunnelling from 
    the leaf axioms to the root conclusion justifies the typing derivation as valid.

\end{Example}

\vspace{-1em}
\begin{Example}[Semantic Derivations]

    Continuing the same example, now with the semantics derivation: $\texttt{let y = 2 in y + y} \Downarrow 4$
    \begin{center}
    \begin{prooftree}
        % Evaluating the integer 2
        \Infer0[(intLit)]{\textcolor{purple}{2} \Downarrow \textcolor{purple}{2}}
    
        % Evaluating y (y is substituted with 2)
        \Infer0[(var)]{\textcolor{purple}{y} \Downarrow \textcolor{purple}{2}}
    
        \Infer0[(var)]{\textcolor{purple}{y} \Downarrow \textcolor{purple}{2}}
    
        % Addition step
        \Infer2[(intAdd)]{\textcolor{purple}{y + y} \Downarrow \textcolor{purple}{4}}
    
        % Let-binding evaluation (substitutes y = 2 in y + y)
        \Infer2[(let)]{\textcolor{purple}{\texttt{let y = 2 in y + y}} \Downarrow \textcolor{purple}{4}}
    \end{prooftree}
    \end{center}
\end{Example}
