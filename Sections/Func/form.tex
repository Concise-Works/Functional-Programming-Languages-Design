\section{Formalizing Ocaml Expressions}

Now we can begin to formalize expressions
in OCaml. We again re-iterate what steps are needed to build expressions
in our language, given that we have some \textit{context} now.

\begin{Def}[Building Expressions]

    When creating new expressions we must follow these steps:
    \begin{enumerate}
        \item \textbf{Context:} Define variable-to-type mappings.
        \item \textbf{Syntax:} Establish how the expression/operation should be written.
        \item \textbf{Typing Rules:} Define the type of the whole expression and its sub-expressions.
        \item \textbf{Semantics:} Clarify the resulting value/evaluation of the defined expression.
    \end{enumerate}
    I.e., what are our types, how are they used, what type of data do they represent, and how does it evaluate?
\end{Def}

\noindent
Now we begin to formalize, though we will abstract the context to $\Gamma$, assuming 
all the types we've defined before (\ref{tab:ocaml-types}).
\begin{Def}[Formalizing Let-Expressions]

    Let $\Gamma$ be the OCaml context, and $=$ be mathematical equality, and $\texttr{=}$ be an OCaml token:
    \begin{itemize}
        \item \textbf{Syntax:} \LARGE $\langle$\text{expr}$\rangle::=$ \texttr{let} $\langle$\text{var}$\rangle$ \texttr{=} $\langle$\text{expr}$\rangle$ \texttr{in} $\langle$\text{expr}$\rangle$ \normalsize\\
        
        \vspace{-.5em}
        \noindent
        If $x$ is a valid variable name, and $e_1$ is a well-formed expression and $e_2$ is a well-formed expression. Then $\texttr{let }x\texttr{ = }e_1\texttr{ in }e_2$ is also a well-formed expression.
        \item \textbf{Typing-Rule:} \LARGE $\dfrac{\Gamma\vdash e_1:\tau_1 \quad \Gamma,x:\tau_1\vdash e_2:\tau}{\Gamma\vdash\texttr{let }x\texttr{ = }e_1\texttr{ in }e_2:\tau}$ \normalsize\\
        
        \noindent
        Given context $\Gamma$, if there's some well-formed expression $e_1$ of type $\tau_1$ and some well-formed expression $e_2$ of type $\tau$, given a variable declaration of $x$ of type $\tau_1$, 
        then within this context, the expression $\texttr{let }x\texttr{  = }e_1\texttr{ in }e_2$ is of type $\tau$.
        
        \item \textbf{Semantics:} \LARGE $\dfrac{e_1\Downarrow v_1 \quad [v_1/x]e_2\Downarrow v}{\texttr{let }x\texttr{ = }e_1\texttr{ in }e_2\Downarrow v}$ \normalsize\\
        
        \noindent
        Following our context $\Gamma$, if a well-formed expression $e_1$ evaluates to $v_1$ and the substitution of $v_1$ for variable $x$ in another 
        well-formed expression $e_2$ evaluates to $v$, then the expression $\texttr{let }x\texttr{ = }e_1\texttr{ in }e_2$ evaluates to $v$.
    \end{itemize}

    \noindent
    Thus, we have formalized the \texttt{let} expression in OCaml.
\end{Def}

\newpage 

\noindent
Before we continue, we introduce the concept of $\top$ and $\bot$.

\begin{Def}[Top and Bottom ($\top$, $\bot$)]
    
    In logic and computer science:
    \begin{itemize}
        \item $\top$ is used to represent \textit{true}, \textit{valid}.
        \item $\bot$ is used to represent \textit{false} or \textit{invalid}.
    \end{itemize}

    \noindent
    Specifically, they are the greatest and least element of a lattice/boolean algebra (hence top and bottom), which when it comes to logic means truthhood and falsehood.
\end{Def}

\noindent
We continue with the formalization of the \texttt{if} expression in OCaml.
\begin{Def}[Formalizing If-Expressions]
    
    Let $\Gamma$ be the OCaml context, then:
    \begin{itemize}
        \item \textbf{Syntax:} \LARGE $\langle$\text{expr}$\rangle::=$ \texttr{if} $\langle$\text{expr}$\rangle$ \texttr{then} $\langle$\text{expr}$\rangle$ \texttr{else} $\langle$\text{expr}$\rangle$ \normalsize\\
     
        \vspace{-.5em}
        \noindent
        If $e_1$ is a well-formed expression, $e_2$ is a well-formed expression, and $e_3$ is a well-formed expression, then $\texttr{if }e_1\texttr{ then }e_2\texttr{ else }e_3$ is also a well-formed expression.

        \item \textbf{Typing-Rule:} \LARGE $\dfrac{\Gamma\vdash e_1:\texttt{bool} \quad \Gamma\vdash e_2:\tau \quad \Gamma\vdash e_3:\tau}{\Gamma\vdash\texttr{if }e_1\texttr{ then }e_2\texttr{ else }e_3:\tau}$ \normalsize\\

        \noindent
        Given context $\Gamma$, let there be well-formed expressions, $e_1$ of type \texttt{bool}, $e_2$ of type $\tau$, and $e_3$ of type $\tau$. Then the expression $\texttr{if }e_1\texttr{ then }e_2\texttr{ else }e_3$ is of type $\tau$.

        \item \textbf{Semantics:} \LARGE $\dfrac{e_1\Downarrow \top \quad e_2\Downarrow v}{\texttr{if }e_1\texttr{ then }e_2\texttr{ else }e_3\Downarrow v}$ (trueCond.) \normalsize\\
        
        \noindent
        Following our context $\Gamma$, if a well-formed expression $e_1$ evaluates $\top$ and another well-formed expression $e_2$ evaluates to $v$, then 
        the expression $\texttr{if }e_1\texttr{ then }e_2\texttr{ else }e_3$ evaluates to $v$ ($e_3$ a well-formed expression).
        \item \textbf{Semantics:} \LARGE $\dfrac{e_1\Downarrow \bot \quad e_3\Downarrow v}{\texttr{if }e_1\texttr{ then }e_2\texttr{ else }e_3\Downarrow v}$ (falseCond.) \normalsize\\
        
        \noindent
        Following our context $\Gamma$, if a well-formed expression $e_1$ evaluates $\bot$ and another well-formed expression $e_3$ evaluates to $v$, then
        the expression $\texttr{if }e_1\texttr{ then }e_2\texttr{ else }e_3$ evaluates to $v$ ($e_2$ a well-formed expression).
    \end{itemize}

    \noindent
    Take note that we must write two semantics rules for the \texttt{if} expression, one for when the condition evaluates to $\top$ and one for when it evaluates to $\bot$.
\end{Def}

