\section{Formalizing Ocaml Expressions}

Now we can begin to formalize expressions
in OCaml. We again re-iterate what steps are needed to build expressions
in our language, given that we have some \textit{context} now.

\begin{Def}[Building Expressions]

    When creating new expressions we must follow these steps:
    \begin{enumerate}
        \item \textbf{Context:} Define variable-to-type mappings.
        \item \textbf{Syntax:} Establish how the expression/operation should be written.
        \item \textbf{Typing Rules:} Define the type of the whole expression and its sub-expressions.
        \item \textbf{Semantics:} Clarify the resulting value/evaluation of the defined expression.
    \end{enumerate}
    I.e., what are our types, how are they used, what type of data do they represent, and how does it evaluate?
\end{Def}

Now we begin to formalize, though we will abstract the context to $\Gamma$, assuming 
all the types we've defined before (\ref{tab:ocaml-types}).
\begin{Def}[Formalizing Let-Expressions]

    Let $\Gamma$ be the OCaml context, then:
    \begin{itemize}
        \item \textbf{Syntax:} \LARGE $\langle$\text{expr}$\rangle::=$ \texttt{let} $\langle$\text{var}$\rangle$ \texttt{=} $\langle$\text{expr}$\rangle$ \texttt{in} $\langle$\text{expr}$\rangle$ \normalsize\\
        
        \noindent
        If $x$ is a valid variable name, and $e_1$ is a well-formed expression and $e_2$ is a well-formed expression. Then $\texttt{let }x=e_1\texttt{ in }e_2$ is also a well-formed expression.
        \item \textbf{Typing-Rule:} \LARGE $\dfrac{\Gamma\vdash e_1:\tau_1 \quad \Gamma,x:\tau_1\vdash e_2:\tau}{\Gamma\vdash\texttt{let }x=e_1\texttt{ in }e_2:\tau}$ \normalsize\\
        
        \noindent
        If 
    \end{itemize}

\end{Def}
