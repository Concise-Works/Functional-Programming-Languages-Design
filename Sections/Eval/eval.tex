\newpage
\section{Semantic Evaluation}
\subsection{Small-step Semantics}

\noindent
In our previous derivations, we've been doing \textbf{Big-step semantics}:

\begin{Def}[Big-Step Semantics]

    Big-step semantics describes how well-formed expressions evaluate to a final value, without detailing each intermediate step.
    
    \medskip
    \noindent\textbf{Notation:} We write $e \Downarrow v$ to mean that the expression $e$ evaluates to the value $v$.\\
    \noindent\textbf{Example:}
    \[
    (\text{sub}\ 10\ (\text{add}\ (\text{add}\ 1\ 2)\ (\text{add}\ 2\ 3))) \Downarrow 2
    \]
    \end{Def}

\noindent
Here, we now introduce \textbf{Small-step semantics}:

\begin{Def}[Small-Step Semantics]

    Small-step semantics describes how an well-formed expressions reduce one step at a time until a final value is reached.
    \noindent\textbf{Notation:} $e \rightarrow e'$, means that $e$ reduces to $e'$ in a single step. These semantics are 
    housed in a \textbf{configuration} with some state-structure and program. We write:
    
    \begin{center}
    \LARGE
    $\underbracket{\langle S,p\rangle}_{\text{configuration}} \longrightarrow \underbracket{\langle S',p'\rangle}_{\text{transformation.}}$
    \normalsize
    \end{center}
    \noindent
    Where $S$ is the state of the program and $p$ is the program. The rightarrow shows the \textbf{transformation} or \textbf{reduction} of the program.
    For purposes in FP there is no state, hence:
    \begin{center}
    \LARGE
    $\langle \varnothing, p\rangle \longrightarrow \langle \varnothing,p'\rangle$
    \normalsize
    \end{center}
    \noindent
    Moving forward we \underline{\textbf{shorthand} this to $p \rightarrow p'$} for brevity. Inference 
    rules outline the semantics of the language:
    \Large
    \[
    \begin{prooftree}
    \hypo{e_1 \rightarrow e_1'}
    \Infer1[(\text{reduce-left-first})]{e_1 + e_2 \longrightarrow e_1' + e_2}
    \end{prooftree}
    \]
    \normalsize
    \noindent
    This rule states that if the $e_1$ position is reducible ($e_1\to e_1'$), reduce it first. \textbf{Additionally},
    $\epsilon$ (epsilon), e.g., $\langle \varnothing, \epsilon\rangle$ is the empty program with an empty state.

\end{Def}

\newpage

\noindent
Let's try defining some small-step semantics for a toy-language:
\begin{Example}[Defining Grammars in Small-Step Semantics]
    
    \label{ex:small-step-semantics}
    Say we have some grammar that can only add from 0 to 9:
    \begin{lstlisting}[numbers=none]
    <expr> ::= "(" <op> <expr> <expr> ")"
            | <int>
    <op>   ::= add
    <int>  ::= 0-9
    \end{lstlisting}

    \noindent
    Let's assume our language reads from \textbf{left-to-right} and define the semantics of \texttt{add}:
    \begin{itemize}
        \item \textbf{Both arguments are expressions:}
        \[
        \begin{prooftree}
        \hypo{\ e_1 \rightarrow e_1'}
        \Infer1[\text{(add-left)}]{ (\text{add}\ e_1\ e_2) \rightarrow (\text{add}\ e_1'\ e_2) }
        \end{prooftree}
        \]
        \item \textbf{Left argument is an integer:}
        \[
        \begin{prooftree}
        \hypo{n \text{ is an int}\qquad e_2 \rightarrow e_2'}
        \Infer1[\text{(add-right)}]{ (\text{add}\ n\ e_2) \rightarrow (\text{add}\ n\ e_2') }
        \end{prooftree}
        \]
        \item \textbf{Both arguments are integers:}
        \[
        \begin{prooftree}
        \hypo{n_1 \text{ and } n_2 \text{ are int} \qquad n_1 + n_2 = v \text{ (integer addition)}}
        \Infer1[\text{(add-ok)}]{ (\text{add}\ n_1\ n_2) \rightarrow v }
        \end{prooftree}
        \]
    \end{itemize}

    \noindent
    When writing semantics (in this case \textbf{add}) think, ``What are all the possible argument states?''
    If we have \texttt{(add 5 e2)}, we must figure out what e2 represents in order to add.\\

    \noindent
    We can almost think of these terminal-symbols as \textbf{base cases}.
    Additionally, since we read left-to-right, we do not need a rule for \texttt{(add e1 n)}, where $e1$ is an expression and $n$ is an integer.
    This scenario is already covered by (add-left).

\end{Example}
        
\begin{Tip}
    States can represent data structures like stacks, making them ideal for modeling stack-oriented languages. For example ($\epsilon$ is the empty program):
    \begin{align*}
    &(\varnothing,\ \texttt{push 2;\ push 3;\ add}) \\
    \rightarrow\quad &(2\ \texttt{::}\ \varnothing,\ \texttt{push 3;\ add}) \\
    \rightarrow\quad &(3\ \texttt{::}\ 2\ \texttt{::}\ \varnothing,\ \texttt{add}) \\
    \rightarrow\quad &(5\ \texttt{::}\ \varnothing,\ \epsilon)
    \end{align*}

\end{Tip}
    
\newpage
\begin{Def}[Multi-Step Semantics]

    Multi-step semantics captures the idea of reducing a configuration through \textbf{zero or more \underline{single-step reductions}}.
    We write $C \rightarrow^{\star} D$ to mean that configuration $C$ reduces to configuration $D$ in zero or more steps.
This relation is defined inductively with two rules:
    
\Large
    \begin{center}
    \begin{minipage}{0.45\textwidth}
        \centering

        \vspace{{.5em}}
        \begin{prooftree}
        \infer0[(reflexivity)]{C \rightarrow^{\star} C}
        \end{prooftree}
    \end{minipage}
    \hfill
    \begin{minipage}{0.45\textwidth}
        \begin{prooftree}
        \hypo{C \rightarrow C'}
        \hypo{C' \rightarrow^{\star} D}
        \infer2[(transitivity)]{C \rightarrow^{\star} D}
        \end{prooftree}
    \end{minipage}
    \end{center}
    
    \normalsize
    \noindent
    These rules formalize:
    \begin{itemize}
        \item Every configuration reduces to itself \hfill \textit{(reflexivity)}
        \item Multi-step reductions can be extended by single-step reductions \hfill \textit{(transitivity)}
        \item If there are multiple ways to reduce $C\rightarrow^\star D$, then the semantics are \textbf{ambiguous}.
    \end{itemize}
    \end{Def}
    
\begin{Example}[Proving Multi-step Reductions]

    \noindent
    We show (add (add 3 4) 5)$\rightarrow^{\star} 14$ based off semantics defined in Example (\ref{ex:small-step-semantics}). We 
    will do multiple rounds of single-step reductions to yield a result. At each round, read bottom up:
    \begin{enumerate}
    \item \hspace{6em}
    \begin{prooftree}
        
        \Infer0[\textcolor{blue}{(add-ok)}]{\text{\textcolor{blue}{add 3 4}} \rightarrow 7}
        \Infer1[\textcolor{blue}{(add-right)}]{\text{add 5 \textcolor{blue}{(add 3 4)}} \rightarrow \text{add 5 7}}
        \Infer1[\textcolor{blue}{(add-left)}]{\text{(add \textcolor{blue}{(add 5 (add 3 4))} 2)} \rightarrow \text{(add (add 5 7) 2)}}
    \end{prooftree}

    \vspace{2em}    
    
    \item \hspace{8em}
    \begin{prooftree}
        \Infer0[\textcolor{blue}{(add-ok)}]{\text{\textcolor{blue}{(add 5 7)}} \rightarrow 12}
        \Infer1[\textcolor{blue}{(add-left)}]{\text{(add \textcolor{blue}{(add 5 7)} 2)} \rightarrow \text{(add 12 2)}}
    \end{prooftree}

    \vspace{2em}   

    \item \hspace{12em}
    \begin{prooftree}
        \Infer0[\textcolor{blue}{(add-ok)}]{\text{\textcolor{blue}{(add 12 2)}} \rightarrow 14}
    \end{prooftree}\\

    \end{enumerate}
    Thus, $\text{(add (add 3 4) 5)} \rightarrow^{\star} 14$. Following rules, we match on the next outermost pattern. 
    Our very first reduction matches with 
    (add-left). In particular, $e_1:=$ (add 5 (add 3 4)), and we see that that's our starting value the next layer up.\\

    \noindent
    Moreover, the trailing 2 in (add (add 5 (add 3 4)) 2), is not evaluated until the very last step (3), as we read from left-to-right.
    Even though \textit{we can see it}, our patterns do not.

\end{Example}

\newpage 

\noindent
Though there is a clear distinction between big-step and multi-step semantics:

\begin{theo}[Multi-Step vs. Big-Step Semantics]

    Both Multi-step and Big-step semantics have the same outcome, but differ in meaning.
    \Large
    \[
      e \rightarrow^\star v \quad \approx \quad e \Downarrow v
    \]
    \normalsize
    Unlike big-step semantics, small-step semantics concerns all possible patterns, allowing us to choose how we reduce terms, and \textbf{in which order}. This eliminates possible ambiguity in
    in big-step pattern matching.
\end{theo}
  
\subsection{Lambda Calculus}

\noindent
We briefly touched on \textbf{Lambda Calculus} in a previous section when discussing the Anonymous Function Definition (\ref{def:anon-func}). Here
we go more in-depth:

\begin{Def}[Lambda Calculus Syntax]

    Lambda calculus is a formal system for representing computations using only single-argument functions, avoiding the need for multiple parameters or state. Lambda calculus has three basic constructs:
    \begin{itemize}
      \item \textbf{Variables:} \(x\), \(y\), \(z\), etc.
      \item \textbf{Abstraction:} \(\lambda x. e\) (a function that takes an argument \(x\) and returns expression \(e\)).
      \item \textbf{Application:} \(e_1\, e_2\) (applying function \(e_1\) to argument \(e_2\)).
    \end{itemize}
  
    \noindent
    The OCaml translation:
    \LARGE
    \[
        \lambda x. e \quad \equiv \quad \texttt{fun } x \texttt{ -> } e
    \]\\
    \normalsize
    \noindent
    Replacing `\texttt{fun}' with `$\lambda$', and `\texttt{->}' with `\texttt{.}'. 
    In pure lambda calculus (no additional syntax), \underline{\textbf{the only value} is a lambda \textbf{abstraction}.}
  \end{Def}
  
  \begin{Def}[The Identity Function]

    \label{def:identity-func}
    The identity function is a function that returns its argument unchanged. Represented as:
    $
    \lambda x. x
    $.
    \noindent
    If we introduce numbers,
    $
    (\lambda x. x)\ 5 \rightarrow 5
    $ (application).
    \noindent
    In OCaml, this is represented as: \texttt{(fun x -> x) 5}. This is also know as the ``\texttt{I}'' combinator.
  \end{Def}

  \newpage 

\noindent
Before moving on we address a few notational conventions:
\begin{Def}[Symbols $\triangleq$ vs. :=]

    The symbol \texttt{:=} is used to define a variable or expression.  
    The symbol $\triangleq$ is used to state that two expressions are equal \textbf{by definition}.
    For example, we may write a paper which reuses some large specific configuration $\langle\{\dots\}, \dots  \rangle$; Instead of 
    writing it again and again, we assign one variable to represent such idea:
    
    \Large
    \[
    \Delta^{\star}_{\Pi}   \triangleq \langle\{\dots\}, \dots  \rangle
    \]
    \normalsize
    Now throughout our paper, $\Delta^{\star}_{\Pi}$ signals to the reader that we are using this configuration.
    As opposed to \texttt{:=}, where we might temporarily assign the variable $a$ to some value multiple times over 
    the course of a document.
    \end{Def}
    
\noindent
Next we look at what happens when we apply the identity function to itself:

\begin{Def}[The Diverging Term \(\Omega\)]

    Lambda Calculus is capable of recursion; The backbone of which is the \textbf{diverging term},
    for which we define as $\Omega$ (Omega):

    \Large
    \[
      \Omega\triangleq (\lambda x. x\,x)(\lambda x. x\,x)
    \]
    \normalsize
    \noindent    
    The inner function \(\lambda x. x\,x\) is sometimes called the \textbf{mockingbird combinator}, as it applies its argument to itself:
    \Large
    \[
      M \triangleq \lambda x. x\,x
    \]
    \normalsize
    Thus, \(\Omega = M\,M\) creates an infinite loop of self-application.
  \end{Def}

  \begin{Example}[Showing \(\Omega\) Divergence]

    \noindent
    We can show that \(\Omega\) diverges by repeated self-application:
    \begin{align*}
      \Omega &\triangleq (\lambda x. x\,x)(\lambda x. x\,x) \\
      &\rightarrow (\lambda x. (\lambda x. x\,x)\ (\lambda x. x\,x)) \\
      &\rightarrow (\lambda x. (\lambda x. (\lambda x. x\,x)\,(\lambda x. x\,x))) \\
      &\rightarrow (\lambda x. (\lambda x. (\lambda x. (\lambda x. x\,x)\,(\lambda x. x\,x)))) \\
      &\rightarrow \ldots
    \end{align*}
    \noindent
    Hence, \(\Omega\) diverges as it continues to apply itself indefinitely.
\end{Example}

\newpage 

\noindent
Application has a formal definition in lambda calculus:
\begin{Def}[Application \& $\beta$-Reduction]

    \label{def:beta-reduction}
    A (beta) \textbf{$\beta$-reduction} is the process of applying a function as an argument in lambda calculus:

    \begin{enumerate}
        \item \[
        \begin{prooftree}
        \hypo{e_1 \rightarrow e_1'}
        \Infer1[(\text{beta-left})]{e_1\ e_2 \rightarrow e_1'\ e_2}
        \end{prooftree}
        \]
        \item \[
        \begin{prooftree}
        \hypo{e_2 \rightarrow e_2'}
        \Infer1[(\text{beta-right})]{(\lambda x . e_1)\ e_2 \rightarrow (\lambda x . e_1)\ e_2'}
        \end{prooftree}
        \]
        \item  \[
        \begin{prooftree}
        \Infer0[(\text{beta-ok})]{(\lambda x. e)\ (\lambda y. e') \rightarrow [(\lambda y. e')/x]e}
        \end{prooftree}
        \]
    \end{enumerate}

    \noindent
    Where (1) we reduce the left-hand side of the application, (2) we reduce the right-hand side of the application, and (3) we apply a function to another function by substitution.\\

    \noindent
    \textbf{Note:} (3) is our base case where \textbf{only values} are allowed to be substituted.
    We can take a different approach and not \textbf{eagerly} evaluate expressions.

    \begin{enumerate}
        \item \[
        \begin{prooftree}
        \hypo{e_1 \rightarrow e_1'}
        \Infer1[(\text{beta-left})]{e_1\ e_2 \rightarrow e_1'\ e_2}
        \end{prooftree}
        \]
        \item \[
        \begin{prooftree}
           \Infer0[(\text{beta-ok})]{(\lambda x. e)\ e' \rightarrow [e'/x]e}
        \end{prooftree}
        \]
    \end{enumerate}

    \noindent
    \textbf{Note:} (2) though we don't evaluate $e'$ immediately, we \textbf{still} expect it to be a value when evaluating the full expression.
    As when we evaluate, we are expecting a value, \textbf{not some unknown variable}.

\end{Def}

\begin{Example}[Simple $\beta$-Reduction]

    \noindent
    Consider the following example of $\beta$-reduction:
    \Large
    \[
    (\lambda x. x + 1)\ 2 \quad \rightarrow \quad [2/x](x + 1) \quad\rightarrow\quad 2 + 1 \rightarrow 3
    \]
    \normalsize
    \noindent
    Here, we supply the function with \(2\), substitute \(2\) for \(x\) in the body of the function, and then evaluate the expression, yielding \(3\).
\end{Example}

\newpage
\noindent
Though we must be wary of what we are substituting for:
\begin{Def}[$\alpha$-Equivalence]

    Two expressions are (alpha) \textbf{$\alpha$-equivalent} if they differ only variable names. 
    This formalizes the \textbf{principle of name irrelevance}: renaming bound variables does not change the meaning of an expression.
    
    \LARGE
    \[
    \lambda x. \lambda y. x \ =_\alpha\ \lambda v. \lambda w. v
    \]
    
    \normalsize
    \noindent
    In OCaml-like syntax:
    
    \vspace{-1em}
    \LARGE
    \[
    \texttt{let x = 2 in x} \ =_\alpha\ \texttt{let z = 2 in z}
    \]
    
    \normalsize
    \noindent
    \textbf{Substitution should preserve $\alpha$-equivalence}. If \( e_1 =_\alpha e_2 \), then for any term \( v \), we have:
    
    \vspace{-1em}
    \LARGE
    \[
    [v/x]e_1 =_\alpha [v/x]e_2
    \]
    \normalsize
    \end{Def}



\noindent
To continue we make the following distinction:
\begin{Def}[Free \& Bound Variables]

    In lambda calculus, a variable in an expression can be either \textbf{free} or \textbf{bound}:
    
    \begin{itemize}
      \item A variable is \textbf{bound} if it is defined by a $\lambda$ abstraction in the expression. 
      \begin{itemize}
        \item \textbf{E.g.,} $(\lambda x. x + 1)$, the variable $x$ is bound.
      \end{itemize}
      \item A variable is \textbf{free} if it is not bound by any enclosing $\lambda$ abstraction.
        \begin{itemize}
            \item \textbf{E.g.,} $(\lambda x. y + 1)$, the variable $y$ is free, \textbf{but} $x$ is still bound.
        \end{itemize} 
    \end{itemize}
    
    \noindent
    Formally, the set of free variables in an expression $e$, written $\mathit{FV}(e)$, is defined inductively as:
    \[
    \begin{aligned}
      \mathit{FV}(x) & = \{x\} && \textit{(Alone variable)} \\
      \mathit{FV}(\lambda x. e) & = \mathit{FV}(e) \setminus \{x\} && \textit{(Function body)}\\
      \mathit{FV}(e_1\ e_2) & = \mathit{FV}(e_1) \cup \mathit{FV}(e_2) && \textit{(Application)}\\
    \end{aligned}
    \]
    

    \end{Def}



    \noindent
    In just a moment, we will define substitution in a way that preserves $\alpha$-equivalence. 
    The high-level idea is that we should \textbf{avoid} substituting variables that are \textbf{bound} within an expression.
    
\newpage

\noindent
Now we define the semantics of substitution:
\begin{Def}[Substitution Semantics]

    \label{def:substitution}
    
    Substitution replaces \textbf{free} occurrences of a variable with another expression. The rules are defined inductively as follows, where in $[v/x]e$, $v$ is assumed to be a \textbf{value} (abstraction):
    
    \begin{center}
        \begin{tabular}{@{}l@{\quad}l@{}}
        (1) & \hspace{2em}
        $[v/y]x = 
        \begin{cases}
        v & \text{if } x = y \\
        x & \text{otherwise}
        \end{cases}$ \\[1em]
        
        (2) & \hspace{0em}
        $[v/y](\lambda x. e) = 
        \begin{cases}
        \lambda x. e & \text{if } x = y \\
        \lambda z. [v/y]([z/x]e) & \text{if } x \in \text{FV}(v),\ \text{choose a fresh} z \\
        \lambda x. [v/y]e & \text{otherwise}
        \end{cases}$ \\[1em]
        
        (3) & \hspace{-.2em}
        $[v/y](e_1\ e_2) = ([v/y]e_1)\ ([v/y]e_2)$
        \end{tabular}
    \end{center}
    
    \noindent
    Moreover on (2)'s middle condition.
    $FV(v)$ means free variables in $v$,
    so if $v:=(\lambda w.y)$ then $FV(v)=\{y\}$.
    Then $[v/x](\lambda\ y.x)$ would be a major problem:
    \LARGE
    $$\Large \lambda y.\lambda w.y \,\equiv \, \lambda y.y \,\neq_\alpha\, \lambda y.x$$
    \normalsize
    The middle condition avoids
    this by swapping in a \textbf{fresh variable} $z$ that does not have any
    conflicts in the body. \textbf{E.g.,} $[v/x](\lambda y.x(\lambda z.y)) $, $y\in FV(v)$, ignoring freshness of $z$:
    
    \vspace{-1em}
    \LARGE
    $$ (\lambda z.v(\lambda z.z)) \neq_\alpha (\lambda y.x(\lambda z.y))$$
    \normalsize
    The abstraction $(\lambda z.y)$ \textbf{captures} the subbed $z$ in $(\lambda z.z)$. Now we pick some arbitrary \textbf{fresh} variable $z$, such 
    that no captures occur:

    \vspace{-1em}
    \LARGE
    $$ (\lambda u.y(\lambda z.u)) =_\alpha (\lambda y.x(\lambda z.y))$$
    \normalsize

    \noindent
    Here we chose the variable $u$ as it does not conflict with the rest of the expression.
\end{Def}
    .
\begin{Example}[Multi-step $\beta$-Reductions]

    \noindent
    Consider the following derivations of $(\lambda f. \lambda x. fx)(\lambda y. y)$ using our substitution semantics:

    \begin{enumerate}
        \item \[
        \begin{prooftree}
        \Infer0[(beta-ok)]{(\lambda f. \lambda x. fx)(\lambda y. y) \rightarrow [(\lambda y. y)/f](\lambda x. fx)=(\lambda x. (\lambda y. y)x)}
        \end{prooftree}
        \]
    \end{enumerate}
    \noindent
    Yielding $(\lambda x. (\lambda y. y)x)$. We \textbf{cannot} evaluate inner-terms. I.e., let $w:=(\lambda y.y)x$, then we have, $(\lambda x.w)$. Hence,
    there is no rule in Definition (\ref{def:beta-reduction}) that allows further reduction.
\end{Example}

\newpage

\noindent
There are times where we can't evaluate a term
\begin{Def}[Stuck Terms]

    A \textbf{stuck term} is a well-formed expression that cannot be reduced, yet is not a value. Applying a non-function value to an argument often causes such issue:
    \Large
    \[
    ((\lambda x. yx)(\lambda x. x)) \rightarrow y(\lambda x. x)
    \]
    \normalsize
    Here, the variable \(y\) is free, and thus, cannot proceed with application. Hence, we are ``stuck'' in the evaluation.
    We can avoid such scenarios via \textbf{typing systems} (discussed in Sec. (\ref{sec:types})).
    
\end{Def}

\noindent
There are two main evaluation strategies in program evaluation:
\begin{Def}[Call-by-Value vs. Call-by-Name]

    There are two main evaluation strategies:
    
    \begin{itemize}
        \item \textbf{Call-by-value (CBV)} evaluates the argument \emph{before} substitution.
        \item \textbf{Call-by-name (CBN)} substitutes the argument expression \emph{directly} without evaluating it first.
    \end{itemize}
    
    \noindent 
    We may illustrate this with the following rules:
    
    \[
    \text{(CBV)} \quad
    \frac{
    e_1 \Downarrow \lambda x. e_1'
    \quad
    e_2 \Downarrow v_2
    \quad
    [v_2/x]e_1' \Downarrow v
    }{
    e_1\ e_2 \Downarrow v
    }
    \quad\quad
    \text{(CBN)} \quad
    \frac{
    e_1 \Downarrow \lambda x. e_1'
    \quad
    [e_2/x]e_1' \Downarrow v
    }{
    e_1\ e_2 \Downarrow v
    }
    \]
    
    \noindent
    The benefit of CBV is that it \textbf{only evaluates an argument once} and is reused. With 
    CBN, the argument is evaluated \textbf{every time} it needs to be computed. This is good if 
    an expensive computation is passed around, but barely touched in execution (e.g., expensive edge-case).
    \end{Def}

    \noindent 
    \underline{We saw this before in Definition (\ref{def:beta-reduction}).} The first semantics were CBV, while the latter was CBN.
    
    \begin{Tip} There are many evaluation strategies optimizing different aspects of computation. In addition to \textbf{CBV} and \textbf{CBN} there are:
    \textbf{Call-by-need} (lazy eval)---like call-by-name, but avoids recomputation by memoizing results. Used in Haskell.
    \textbf{Call-by-reference}---used in languages with pointers (functions receive variable references).
    \textbf{Call-by-sharing}---also pointer focused langues (functions receive object references).

\end{Tip}
        
\newpage 

    
\begin{Def}[Well-Scopedness and Closedness]

    \label{def:well-scopedness}

    We expand on the idea of \textbf{scope} in respect to free and bound variables: 

    \begin{itemize}
        
        \item 
        An expression \( e \) is \textbf{closed} if it contains \textbf{no free variables}, also called a \textbf{combinator}.
        \item 
        An expression \( e \) is \textbf{well-scoped} if the expression is closed or is bound under the configuration's state.
    \end{itemize}
    \noindent
    \underline{\textbf{Every closed expression is well-scoped}}, \textbf{but} not every well-scoped expression is closed.
\end{Def}

\begin{Note}
\textbf{Note:} The expression $x\mapsto y$ means that $x$ is mapped to $y$. Hence, $x$ is bound to $y$.
\end{Note}
\begin{Example}[Closed vs. Open Terms]
    
    \label{ex:closed-terms}
    Recall that abstractions bind to their argument variable:
    \begin{itemize}
        \item \textbf{Open Term:} \((\lambda x. y)\) is \emph{not closed}, since \(y\) is free.
        \item \textbf{Closed Term \& Well-scoped:} \((\lambda x. \lambda y. y)\) is \emph{closed}, since both \(x\) and \(y\) are bound.
        \item \textbf{Well-scoped:} $\langle \{x\mapsto (\lambda y. y)\}, (\lambda w.x)\rangle$ is \emph{well-scoped}, since \(x\) is bound in state, but not closed.
    \end{itemize}

\end{Example}

\begin{Def}[Lexical vs Dynamic Scope]

    \label{def:scope}

    \noindent
    A variable's \textbf{scope} determines where in the program the variable can be referenced.

    \begin{itemize}
        \item \textbf{Lexical (or static) scope:} Textual delimiters define the scope of a binding.
        
        \item \textbf{Dynamic scope:} Bindings are determined at runtime based on the call stack. I.e., the most recent binding in the call stack is used regardless 
        was declared.
    \end{itemize}
\end{Def}

\newpage 

\noindent
To understand the difference between lexical and dynamic scoping:

\begin{Example}[Dynamic vs Lexical Scoping]

    \noindent
    Consider the following Bash code:
    
    \begin{lstlisting}[language=bash,numbers=none]
    f() { x=23; g; } 
    g() {  y=$x; }
    f
    echo $y   # prints 23
    \end{lstlisting}
    
    \noindent
    In Bash, the variable \texttt{x} is not defined in \texttt{g}, but since \texttt{f} called \texttt{g} and \texttt{x} was defined in \texttt{f}, \texttt{g} sees it. This is \textbf{dynamic scoping}.
    In contrast, consider the following Python code:

    \begin{lstlisting}[language=python,numbers=none]
    x = 0
    def f():
        x = 1
        return x
    
    assert f() == 1
    assert x == 0
    \end{lstlisting}
    
    \noindent
    Now consider the following OCaml code:
    \begin{lstlisting}[language=ML,numbers=none]
    let x = 0
    let f () = 
        let x = 1 in
        x
    
    let _ = assert (f () = 1)
    let _ = assert (x = 0)
    \end{lstlisting}
    
    \noindent
    Both Python and OCaml use \textbf{lexical scoping}, meaning each use of \texttt{x} refers to the closest enclosing definition in the source code, not the caller's environment.
\end{Example}

\noindent
Lastly, we can model if statements in lambda calculus:
\begin{theo}[Lambda Calculus -- True-False Conditions]

    True,false, and if conditions are defined as:
    \Large
    \[
    \texttt{true} \triangleq \lambda x. \lambda y. x
    \quad\quad
    \texttt{false} \triangleq \lambda x. \lambda y. y
    \]
    \LARGE
    \[
    \texttt{IF} \triangleq \lambda cond. \lambda then. \lambda else. cond\ then\ else
    \]

    \normalsize
    \noindent
    \textbf{E.g.,}\\ 
    (\texttt{IF true 1 2}) $\to [2/y][1/x](\lambda x.\lambda y.x) \to 1$.\\
     (\texttt{IF false 1 2}) $\to [2/y][1/x](\lambda x.\lambda y.y) \to 2$.
\end{theo}