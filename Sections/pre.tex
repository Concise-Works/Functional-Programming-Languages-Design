This text assumes that the reader has a basic understanding of programming languages and grade-school mathematics along
with a fundamentals grasp of discrete mathematics. The following definitions are provided to ensure that the reader is familiar with
the terminology used in this document.
\begin{Def}[Token]

    A \textbf{token} is a basic, indivisible unit of a programming language or formal grammar, representing 
    a meaningful sequence of characters. Tokens are the smallest building blocks of syntax and are typically 
    generated during the lexical analysis phase of a compiler or interpreter.

    \vspace{1em}
    \noindent
    Examples of tokens include:
    \begin{itemize}
        \item \texttt{keywords}, such as \texttt{if}, \texttt{else}, and \texttt{while}.
        \item \texttt{identifiers}, such as \texttt{x}, \texttt{y}, and \texttt{myFunction}.
        \item \texttt{literals}, such as \texttt{42} or \texttt{"hello"}.
        \item \texttt{operators}, such as \texttt{+}, \texttt{-}, and \texttt{=}.
        \item \texttt{punctuation}, such as \texttt{(}, \texttt{)}, \texttt{\{}, and \texttt{\}}.
    \end{itemize}

    \vspace{1em}
    Tokens are distinct from characters, as they group characters into meaningful units based on the language's syntax.
\end{Def}

\begin{Def}[Non-terminal and Terminal Symbols]
    
    \label{def:non-terminal-terminal}
    \textbf{Non-terminal symbols} are placeholders used to represent abstract categories or structures in a language. 
    They are expanded or replaced by other symbols (either terminal or non-terminal) as part of generating valid sentences 
    in the language.
    \begin{itemize}
    \item \textbf{E.g.}, ``Today is $\langle$name$\rangle$'s birthday!!!'', where $\langle$name$\rangle$ is a non-terminal symbol, 
    expected to be replaced by a terminal symbol (e.g., ``Alice''). 
    \end{itemize}

    \noindent
    \textbf{Terminal symbols} are the basic, indivisible symbols in a formal grammar. They represent the actual characters 
    or tokens that appear in the language and cannot be expanded further. For example:
    \begin{itemize}
        \item \texttt{+}, \texttt{1}, and \texttt{x} are terminal symbols in an arithmetic grammar.
    \end{itemize}
\end{Def}

\newpage







\begin{Def}[Symbol ``\texttt{:=}'']

    The symbol \texttt{:=} is used in programming and mathematics to denote ``assignment'' or ``is assigned the value of''. 
    It represents the operation of giving a value to a variable or symbol.

    \vspace{1em}
    \noindent
    For example:
    \[
    x := 5
    \]
    This means the variable $x$ is assigned the value $5$.

    \vspace{1em}
    \noindent
    In some contexts, \texttt{:=} is also used to indicate that a symbol is being defined, such as:
    \[
    f(x) := x^2 + 1
    \]
    This means the function $f(x)$ is defined as $x^2 + 1$.
\end{Def}

\begin{Def}[Substitution:\ \text{$[v/x]e$}]
    
    \label{def:substitution}
    Formally, $[v/x]e$ denotes the substitution of $v$ for $x$ in the expression $e$.
    For example:
    \[
    [3/x](x + x) = 3 + 3
    \]
    This means that every occurrence of $x$ in $e$ is replaced with $v$. We may string multiple substitutions together,
    such as:
    \[
    [3/x][4/y](x + y) = 3 + 4
    \]
    Where $x$ is replaced with $3$ and $y$ is replaced with $4$.
\end{Def}

