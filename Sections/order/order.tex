\section{Function Order}
In Higher-Order Programming, classes of values are established.

\begin{Def}[First-Class Values]

    \noindent
    A \textbf{first-class value} in a programming language is an entity that can be:
    \begin{itemize}
        \item \textbf{Assigned to variables}
        \item \textbf{Passed as an argument} to a function
        \item \textbf{Returned from a function}
        \item \textbf{Stored in data structures}
    \end{itemize}

    \noindent
    In \texttt{OCaml}, \textbf{functions are first-class values}, meaning they can be used like any other value. This allows for:
    \begin{itemize}
        \item Defining functions as values using \texttt{let}
        \item Passing functions as arguments to other functions
        \item Returning functions from other functions (closures)
    \end{itemize}

    \noindent
    However, \textbf{types are not first-class in OCaml}, meaning they cannot be manipulated as runtime values (e.g., dynamically created or passed as arguments).
\end{Def}

\noindent 
Passing functions to other functions is where the idea of \textbf{higher-order functions} comes into play.
\begin{Def}[Higher-Order Functions]

    \noindent
    A \textbf{higher-order function} is a function that:
    \begin{itemize}
        \item \textbf{Takes one or more functions as arguments}
        \item \textbf{Returns a function as a result}
    \end{itemize}

    \noindent
    We've discussed such functions before in Subsection (\ref{subsec:func-ocaml}). E.g.,
    (\texttt{fun f x -> f x}) is\\ (\texttt{fun f -> fun x -> f x}) where the first function returns and takes a function as an argument.
    Recall (\texttt{f x}) is a function application, where (\texttt{f}) is a function value.

\end{Def}

\newpage 


\begin{Def}[Order of Functions]
    
    The concept of ``higher-order'' extends beyond first-order functions, which take and return only values. A function's \textbf{order} is determined by how many levels of function application it involves:
    \begin{itemize}
        \item \textbf{1st order}: \texttt{int}
        \item \textbf{2nd order}: \texttt{int -> int}
        \item \textbf{3rd order}: \texttt{(int -> int) -> int}
        \item \textbf{4th order}: \texttt{((int -> int) -> int) -> int}
    \end{itemize}
    
    In theory, this hierarchy can extend infinitely, but in practice, functions rarely exceed \textbf{third or fourth order}.
\end{Def}

\begin{Tip}[What Does ''Higher-Order" Mean?]
    \textit{``Like things and functions are different, so are functions whose arguments are functions 
    \textbf{radically different} from functions whose arguments \textbf{must be things}. 
    I call the latter functions of first order, the former functions of second order.''}\\

    \noindent
    -- Gottlob Frege
\end{Tip}

\subsection{The Abstraction Principle: Maps, Filters, Folds}
The \textbf{Abstraction Principle} is a fundamental concept in computer science that states:
\begin{Def}[Abstraction Principle]

    The \textbf{Abstraction Principle} states that programs should be structured by separating \textbf{core functionality} from specific details.\\

    \noindent
    This principle is applied by:
    \begin{itemize}
        \item \textbf{Abstracting core functionality} to improve reusability and modularity.
        \item Using \textbf{higher-order functions} to \textbf{parametrize} behavior based on specific problem requirements.
        \item Understanding the \textbf{algebra of programming}, which helps reason about program structure and transformations.
    \end{itemize}

    \noindent
    Following this principle results in more \textbf{flexible, maintainable, and reusable} programs.
\end{Def}

\noindent
We'll discuss three common patterns we see a lot in programmings \textbf{Maps}, \textbf{Filters}, and \textbf{Folds}.

\newpage 

\noindent

\begin{Def}[Map Function]

    Given a function \( f \) and a list \([x_1, x_2, \dots, x_n]\), the \textbf{map} function produces:
    \[
    \texttt{map } f [x_1, x_2, \dots, x_n] = [f(x_1), f(x_2), \dots, f(x_n)]
    \]
    \noindent
    I.e., it applies \( f \) to each element of the list, returning a new list with the results.\\
    \begin{lstlisting}[language=OCaml, caption={Ocaml Implementation of Map}, numbers=none]
    let rec map f lst =
      match lst with
      | [] -> []
      | x :: xs -> f x :: map f xs

    (* Example usage *)
    let doubled = map (fun x -> x * 2) [1; 2; 3]  (* Returns [2; 4; 6] *)
    \end{lstlisting}

    \noindent
    The \textbf{map} function abstracts over how we apply a function to each element of a list, allowing us to reuse the same logic for different functions.
\end{Def}

\noindent
\begin{Def}[Filter Function]

    Given a predicate function \( p \) and a list \([x_1, x_2, \dots, x_n]\), the \textbf{filter} function produces:
    \[
    \texttt{filter } p\ [x_1, x_2, \dots, x_n] = [x_i \mid p(x_i) \texttt{ is true}]
    \]
    \noindent
    I.e., it returns a new list containing only elements for which \( p \) evaluates to \texttt{true}.\\

    \begin{lstlisting}[language=OCaml, caption={Ocaml Implementation of Filter}, numbers=none]
    let rec filter p lst =
      match lst with
      | [] -> []
      | x :: xs ->  (if p x then [x] else []) @ filter p xs

    (* Example usage *)
    let evens = filter (fun x -> x mod 2 = 0) [1; 2; 3; 4; 5]
    (* Returns [2; 4] *)
    \end{lstlisting}

    \noindent
    The \textbf{filter} function abstracts over how we select elements from a list, allowing us to express selection logic concisely.
\end{Def}

\newpage 

\noindent
\begin{Def}[Fold Functions]

    Given a binary function \( f \), an initial accumulator value \( a_0 \), and a list \([x_1, x_2, \dots, x_n]\), the \textbf{fold} functions \texttt{fold\_left} and \texttt{fold\_right} reduce the list to a single value by combining elements recursively.\\

    \noindent
    The \textbf{fold left} function applies \( f \) from left to right (recursive statement on the right):
    \[
    \texttt{fold\_left } (-) \ a_0\ [x_1, x_2, \dots, x_n] = (((a_0 - x_1) - x_2) \dots - x_n)
    \]
    \begin{lstlisting}[language=OCaml, caption={Ocaml Implementation of Fold\_Left}, numbers=none]
    let rec fold_left f acc lst =
      match lst with
      | [] -> acc
      | x :: xs -> fold_left f (f acc x) xs

    (* Example usage *)
    let subtract = fold_left (-) 10 [1; 2; 3]  
    (* Returns 4, since ((10 - 1) - 2) - 3 = 4 *)
    \end{lstlisting}

    \vspace{1em}
    \noindent
    The \textbf{fold right} function applies \( f \) from right to left (recursive statement on the left):
    \[
    \texttt{fold\_right } (-) \ a_0\ [x_1, x_2, \dots, x_n] = (x_1 - (x_2 - (\dots - (x_n - a_0))))
    \]

    \noindent
    \begin{lstlisting}[language=OCaml, caption={Ocaml Implementation of Fold\_Right}, numbers=none]
    let rec fold_right f lst acc =
      match lst with
      | [] -> acc
      | x :: xs -> f x (fold_right f xs acc)

    (* Example usage *)
    let subtract = fold_right (-) [1; 2; 3] 10  
    (* Returns -8, since 1 - (2 - (3 - 10)) = -8 *)
    \end{lstlisting}
\end{Def}

\begin{Tip}
    Think in opposition: fold left (recursive statement on the right), fold right (recursive statement on the left).
\end{Tip}

\newpage 

\noindent
To illustrate the difference between \texttt{fold\_left} and \texttt{fold\_right}, consider the following illustration:

\begin{figure}[h]
    
    \hspace{-.5em}
    \includegraphics[width=1\textwidth]{Sections/order/fold.png}
    \caption{Fold Left vs. Fold Right}
\end{figure}

\noindent
In both, we still build from the base case/last element to the front of the list. So above, we are always building from right to left.
The difference: 

\begin{itemize}
    \item \textbf{Fold Left}: We are essentially wrapping the accumulator form the start to the end of the list. Meaning,
    The list folds from right to left starting with the last element.
    \item \textbf{Fold Right}: Here we make recursive calls starting with the first element and appending the accumulator to the end of the list. Meaning,
    we fold from left to right starting with the first element.
\end{itemize}

\begin{Example}[Additional Fold Use-cases]

    To get more comfortable with folds, consider the following examples:

    \begin{lstlisting}[language=OCaml, numbers=none]
    List.fold_left  (fun x acc -> max x acc) 3 [0;7;2;1];;
    List.fold_right (fun x acc -> max x acc) [0;7;2;1] 3;;
    (* Both return 7 *)

    type pairs = Int of int | TPairs of pairs * pairs;;

    List.fold_left (fun x acc -> TPairs (x, acc)) (Int 0) [Int 1; Int 2];;
    - : pairs = TPairs (TPairs (Int 0, Int 1), Int 2)


    List.fold_right (fun x acc -> TPairs (x,acc)) [Int 0; Int 1] (Int 2);;
    - : pairs = TPairs (Int 0, TPairs (Int 1, Int 2))
    \end{lstlisting}
\end{Example}




