\newpage 

\section{Type Theory}
\subsection{Simply Typed Lambda Calculus}

\label{sec:types}

\noindent
An additional way to protect and reduce ambiguity in programming languages is to use \textbf{types}:

\begin{Def}[A Type]

    A \textbf{type} is a syntactic object that describes 
    the kind of values that an expression pattern is allowed to take.
    This happens before evaluation to safeguard unintended behavior.
\end{Def}

\noindent
Recall our work in Section (\ref{sec:formalizing-ocaml-expressions}). We add the following:

\begin{Def}[Contexts \& Typing Judgments]

    \textbf{Contexts:} $\Gamma$ is a finite mapping of variables to types. \textbf{Typing Judgments:} $\Gamma \vdash e : \tau$, reads ``$e$ has type $\tau$ in context $\Gamma$''. It is 
    said that $e$ is \textbf{well-typed} if $\cdot \vdash e : \tau$ for some $\tau$, where ($\cdot$) is the \textbf{empty context}.
    Such types we may inductively define:\\
    \begin{minipage}{0.45\textwidth}
        \begin{align*}
            \Gamma &::= \cdot \mid \Gamma, x : \tau \\
            x &::= vars\\
            \tau &::= types
        \end{align*}
    \end{minipage}
    \hfill
    \begin{minipage}{0.45\textwidth}
        \[
        \frac{
          \Gamma \vdash e_1 : \tau_1 \quad \cdots \quad \Gamma \vdash e_k : \tau_k
        }{
          \Gamma \vdash e : \tau
        }
        \]
    \end{minipage}

    \vspace{.5em}
    \noindent
    In practice, a context is a set (or ordered list) of variable declarations (variable-type pairs). Our 
    inference rules operate with these contexts to determine the type of an expression:
\end{Def}

\noindent 
This leads us to an extension of lambda calculus:
\begin{Def}[Simply Typed Lambda Calculus (STLC)]

    The syntax of the Simply Typed Lambda Calculus (STLC) extends the lambda calculus by including types and a unit expression.
    
    \begin{lstlisting}[numbers=none, mathescape=true]
    <e>  ::=  () | <v> | <e> <e> 
          |   fun "("<v> : <ty>")" -> <e>
    <ty> ::=  unit | <ty> -> <ty>
    <v>  ::=  [a-zA-Z]
    \end{lstlisting}
    
    \noindent
    We include the unit type (arbitrary value/void) and that functions are now typed. We transition into 
    a more mathematical notation:

\[
\begin{array}{ll}
e ::= & \bullet \mid x \mid \lambda x^{\tau}.\, e \mid e\,e \\
\tau ::= & \top \mid \tau \rightarrow \tau \\
x ::= & \textit{variables}
\end{array}
\]

\end{Def}
    


\newpage 
\noindent
This brings us to the typing rules for STLC:


\begin{Def}[Typing Rules for STLC]

    \label{def:stlc-typing-rules}
    \textbf{Typing Rules:} The typing rules for STLC are as follows:
    \[
\begin{array}{cc}
\begin{prooftree}
  \hypo{}
  \infer1[\textsf{(unit)}]{\Gamma \vdash \bullet : \top}
\end{prooftree}
&
\begin{prooftree}
  \hypo{\Gamma, x{:}\tau \vdash e : \tau'}
  \infer1[\textsf{(abstraction)}]{\Gamma \vdash \lambda x^{\tau} .\, e : \tau \rightarrow \tau'}
\end{prooftree}
\\[2em]
\begin{prooftree}
  \hypo{(x{:}\tau) \in \Gamma}
  \infer1[\textsf{(variable)}]{\Gamma \vdash x : \tau}
\end{prooftree}
&
\begin{prooftree}
  \hypo{\Gamma \vdash e_1 : \tau \rightarrow \tau'}
  \hypo{\Gamma \vdash e_2 : \tau}
  \infer2[\textsf{(application)}]{\Gamma \vdash e_1 e_2 : \tau'}
\end{prooftree}
\end{array}
\]
\noindent
Such rules enforce that application is only valid when the $e_1$ position is a function type and the $e_2$ position is a valid argument type.
\end{Def}

When encountering notation, types are often omitted in some contexts:
\begin{Def}[Church vs. Curry Typing]

There are two main styles of typing:\\

\noindent
 \textbf{Curry-style typing:} Typing is \textbf{implied} (extrinsic) via typing judgement: 
    \begin{lstlisting}[numbers=none, mathescape=true]
    fun x -> x
    \end{lstlisting}
    
\vspace{1em}
\noindent
\textbf{Church-style typing:} Types are \textbf{explicitly} (intrinsic) annotated in the expression:
    \begin{lstlisting}[numbers=none, mathescape=true]
    fun (x : unit) -> x
    \end{lstlisting}


\noindent
\textbf{Important:} Curry-style does not imply polymorphism, expressions are judgement-backed.

\end{Def}

\noindent 
This leads us the an important lemma:
\begin{Def}[Lemma -- Uniqueness of Types ]

    Let $\Gamma$ be a typing context and $e$ a well-formed expression in STLC:
    \Large
    \begin{center}
        If $\Gamma \vdash e : \tau_1$ and $\Gamma \vdash e : \tau_2$, then $\tau_1 = \tau_2$.
    
    \end{center}
    \normalsize

    \noindent
    I.e., typing in STLC is \textbf{deterministic} -- a well-typed expression has a \textbf{unique type} under any fixed context.
    
    \end{Def}

    \newpage 

\noindent 
To prove the above lemma we must recall structural induction:
\begin{Def}[Structural Induction]

    \textbf{Structural induction} is a proof technique used to prove properties of recursively defined structures. It consists of two parts:
    \begin{itemize}
        \item \textbf{Base case:} Prove the property for the simplest constructor (e.g., a variable or unit).
        \item \textbf{Inductive step:} Assume the property holds for immediate substructures, and prove it holds for the structure built from them.
    \end{itemize}
    
    \noindent
    This differs from standard mathematical induction over natural numbers, where the base case is typically $n = 0$ (or $1$), and the inductive step proves $(n+1)$ assuming $(n)$.
    
    \noindent
    In the context of \textbf{lambda calculus}, expressions are recursively defined and built from smaller expressions. Structural induction proceeds as:
    \begin{itemize}
        \item \textbf{Base case:} Prove the property for the simplest expressions (e.g., variables and units).
        \item \textbf{Inductive step:} Assume the property holds for sub-expressions $e_1, e_2, \ldots, e_k$, and prove it holds for a compound expression $e$ (e.g., abstractions and application).
    \end{itemize}
    
    \end{Def}
    
\noindent
% Generation Lemma
A proof of the previous lemma is as follows:
\begin{Proof}[Lemma -- Uniqueness of Types]

    We prove that if $\Gamma \vdash e : \tau_1$ and $\Gamma \vdash e : \tau_2$, then $\tau_1 = \tau_2$ ($\Gamma$ is a fixed typing context, and $e$ a well-formed expression)
    via structural-induction:\\

    \noindent
    \textbf{Case 1:} $e = \bullet$ (unit value).  
    We define a generation lemma (\textbf{*}): if $\Gamma \vdash \bullet : \tau \to \tau = \top$. Therefore, if 
    $\Gamma \vdash \bullet : \tau_1$ and $\Gamma \vdash \bullet : \tau_2$, then $\tau_1=\tau_2=\top$, by lemma (\textbf{*}).  
    \\

    \noindent
    \textbf{Case 2:} $e = x$ (a variable). 
    We define a generation lemma (\textbf{**}): $\Gamma,x\vdash \tau\to x:\tau \in \Gamma$ 
    Since $(x : \tau_1), (x : \tau_2) \in \Gamma$ and $\Gamma$ is fixed, $x$ maps to a single type. Thus, $\tau_1 = \tau_2$, by lemma (\textbf{**}).
    \\

    \noindent
    \textbf{Case 3:} $e = \lambda x^\tau. e'$.  
    Abstraction typings require the form $\tau \to \tau'$. Both derivations,
    $\Gamma \vdash \lambda x^\tau. e' : \tau_1$ and $\Gamma \vdash \lambda x^\tau. e' : \tau_2$,
    must have such form. Hence, by inductive hypothesis on $e'$ (under $\Gamma,x:\tau$), we conclude $\tau_1 = \tau_2$.
    \\ 

    \noindent
    \textbf{Case 4:} $e = e_1\, e_2$ (application).  
    Suppose $\Gamma \vdash e_1\, e_2 : \tau_1$ and $\Gamma \vdash e_1\, e_2 : \tau_2$.  
    Then $e_1$ must have type $\tau' \to \tau_1$ and $\tau' \to \tau_2$ respectively, and $e_2$ type $\tau'$.  
    With likewise reasoning from Case 3, and through the inductive hypothesis on $e_1$ and $e_2$, we conclude $\tau_1 = \tau_2$.
    \\

    \noindent
    Hence, by induction on the typing derivation, the type assigned to any expression is unique.

    \end{Proof}

    \newpage 

    \noindent
    We continue with the following theorems:
        
    \begin{theo}[Well-Typed Implies Well-Scoped]

        If $e$ is well-typed in $\Gamma$, then $e$ is well-scoped.
    \end{theo}

    \begin{Proof}[Well-Typed Implies Well-Scoped]

        We prove this by induction on the structure of a well-formed expression $e$:
        \begin{itemize}
            \item Base cases: $e = \bullet$ or $e = x$ (variable), as based on Definition (\ref{def:stlc-typing-rules}), maps to a single type. Hence, they are well-typed.
            \item Inductive cases:
            \begin{itemize}
                \item $e = \lambda x^{\tau}. e'$: The abstraction argument $x$ is bound and explicitly typed. By the inductive hypothesis, $e'$ is well-typed in $\Gamma, x : \tau$.
                \item $e = e_1\, e_2$: Expression $e_1$ must be a function type $\tau' \to \tau_1$ and $e_2$ must be of type $\tau'$. By the inductive hypothesis, both $e_1$ and $e_2$ are well-typed in $\Gamma$.
            \end{itemize}
        \end{itemize}

        \noindent
        Therefore by induction, if $e$ is well-typed, then all sub-expressions must also be bound, and hence well-scoped.
    \end{Proof}
        
    \noindent
    We've been assuming the following properties of our evaluation relation:
    \begin{theo}[Big-Step Soundness]

        If $\cdot \vdash e : \tau$ then there is a value $v$ such that $\langle \varnothing, e \rangle \Downarrow v$ and $\cdot \vdash v : \tau$.
        \end{theo}
        
        \noindent
        Or to be more specific with small-step evaluation:
    \begin{theo}[Progress \& Preservation]

    If $\cdot \vdash e : \tau$, then
    \begin{itemize}
        \item \textbf{(Progress)} Either $e$ is a value or there is an $e'$ such that $e \rightarrow e'$.
        \item \textbf{(Preservation)} If $\cdot \vdash e : \tau$ and $e \rightarrow e'$, then $\cdot \vdash e' : \tau$.
    \end{itemize}
    \end{theo}
        
    \newpage 

\begin{Proof}[Progress \& Preservation]

    We prove the progress and preservation theorem by induction on the structure of a well-formed expression $e$:

    \begin{itemize}
        \item \textbf{Base cases:} $e = \bullet$, $e = x$ (well-scoped variable), or $e = \lambda x^{\tau}. e'$ are values by Definition(\ref{def:stlc-typing-rules}), so they satisfy progress and need not reduce.

        \item \textbf{Inductive case:} $e = e_1\, e_2$ (application):

        \begin{itemize}
            \item \emph{Progress.} By the inductive hypothesis, either $e_1$ and $e_2$ are values, or they can take a step. 
            If both are values, then $e_1$ must be a lambda abstraction (i.e., $\lambda x^{\tau}. e'$), and $e_1\, e_2$ can step to $[e_2 / x]e'$, given that $e'$ is the body of $e_1$. Thus, progress holds.

            \item \emph{Preservation.} Suppose $\Gamma \vdash e_1\, e_2 : \tau$. Then by inversion, $\Gamma \vdash e_1 : \tau_1 \to \tau$ and $\Gamma \vdash e_2 : \tau_1$ for some $\tau_1$. Assume $e_1 = \lambda x^{\tau_1}. e'$. Then $e_1\, e_2 \rightarrow [e_2/x]e'$. By the typing rule for abstractions, we have $\Gamma, x : \tau_1 \vdash e' : \tau$. Then by the \emph{Substitution Lemma}:

            \begin{quote}
                If $\Gamma \vdash e_2 : \tau_1$ and $\Gamma, x:\tau_1 \vdash e' : \tau$, then $\Gamma \vdash [e_2/x]e' : \tau$.
            \end{quote}

            Therefore, $\Gamma \vdash [e_2/x]e' : \tau$, which proves preservation for this case.
        \end{itemize}
    \end{itemize}

    \noindent
    Hence by induction, if $e$ is well-typed, then either $e$ is a value or there exists $e'$ such that $e \rightarrow e'$, and if $e \rightarrow e'$, then $e'$ is of the same type as $e$.
\end{Proof}

\noindent
Now for some practice: 
\begin{Example}[Determining Type of an Expression (Part-1)]

    We determine the smallest typing context $\Gamma$ for the expression $(\lambda x ^{(\top \to \top) \to \top} .\ x(\lambda z^{\top}.\ x(wz)))\ y$:
    \begin{align*}
         \varnothing \vdash (\lambda x ^{(\top \to \top) \to \top} .\ x(\lambda z^{\top}.\ x(wz)))\ y\ :\ ?&& \text{(Given)}\\
         \{y: (\top \to \top)\to\top\} \vdash  (\lambda x ^{(\top \to \top) \to \top} .\ x(\lambda z^{\top}.\ x(wz)))\ y\ :\ ?&& \text{(Application Arg.)}\\
            \{y: (\top \to \top)\to\top, x: (\top \to \top) \to \top\} \vdash x(\lambda z^{\top}.\ x(wz))\ :\ ?&& \text{(Abstraction Type Sub.)}
    \end{align*}

    \noindent
    We note that $(\lambda z^{\top}.\ x(wz))$ must be of type $(\top \to \top)$ to satisfy $x$:
    \begin{align*}
        \{y: (\top \to \top)\to\top, x: (\top \to \top) \to \top, z: \top\} \vdash x(wz)\ :\ ?&& \text{(Application Arg.)}
    \end{align*}

    \noindent
    We see $(wz)$ must be type $(\top \to \top)$ as well. We know $z$ is of type $\top$, therefore $w$ must accept such type. In addition, 
    $w$ must return type $(\top \to \top)$ for the application of $x$ to be valid. Hence, we conclude:
    
    $$\Gamma := \{y: (\top \to \top)\to\top, x: (\top \to \top) \to \top, z: \top, w: \top \to (\top \to \top)\}$$

    \noindent
    Since $x$ is the outermost abstraction, we can conclude that the output type is $\top$.
\end{Example}

\newpage 

\begin{Example}[Typing an Ocaml Expression]
    
    We find the typing context $\Gamma$ for the following expressions:
    \begin{lstlisting}[language=OCaml, numbers=none, mathescape=true]
    (*1*) fun f -> fun x -> f (x + 1)      : ?
    (*2*) let rec f x = f (f (x + 1)) in f : ?
    \end{lstlisting}
    
    \noindent
    1. We note the application of $f\ (x + 1)$. Therefore, $f$ must be an (\texttt{int}$\to$\texttt{?}), as addition returns an \texttt{int}. Subsequently,
    the $x$ used in such addition and the function argument, must also be an \texttt{int}. The rest of the expression (\texttt{f (x + 1)}) is some arbitrary type \texttt{'a}. 
    Hence:
    $$\Gamma := \{f: \texttt{int} \to \texttt{'a}, x: \texttt{int}\}$$
    \noindent
    With a final type of \texttt{(int -> 'a) -> int -> 'a} for the entire expression.\\

    \noindent
    2. Again, we note the application of $f\ (x + 1)$. Therefore, $f$ must be an (\texttt{int}$\to$\texttt{?}). This function is enclosed
    within another $f$ yielding $f(f(x+1))$, therefore, $f$ must return an \texttt{int} to satisfy the outer $f$. Hence, we conclude $\Gamma$ as: 

    $$\Gamma := \{f: \texttt{int} \to \texttt{int}, x: \texttt{int}\}$$

\end{Example}

\subsection{Polymorphism}

\noindent
There are moments when we might redundantly define functions such as:

\begin{lstlisting}[language=OCaml, numbers=none]
    let rec rev_int (l : int list) : int list =
        match l with
        | [] -> []
        | x :: l -> rev l @ [x]
    let rec rev_string (l : string list) : string list =
        match l with
        | [] -> []
        | x :: l -> rev l @ [x]
\end{lstlisting}

\noindent
Here we have two functions that are identical in structure, but differ in type.

\begin{Def}[Polymorphism]

    \textbf{Polymorphism} is the ability of a function to operate on values of different types while using a single uniform interface (signature).
    There are two types:
    \begin{itemize}
        \item \textbf{Ad hoc:} The ability to \textbf{overload} (redefine) a function name to accept different types.
        \item \textbf{Parametric:} The ability to define a function that can accept any type as an argument, and return a value of the same type.
    \end{itemize}
\end{Def}

\newpage
\noindent
We will focus on \textbf{parametric polymorphism}, as simply overloading in OCaml redefines the function name.
An example of a parametric polymorphic function in OCaml is the identity function:
\begin{lstlisting}[language=OCaml, numbers=none]
    let id = fun x -> x (* 'a -> 'a *)
    let a = id 0        (* int *)
    let b = id (0 = 0)  (* bool *)
    let c = id id       (* (int -> int) *)
\end{lstlisting}

\begin{Def}[Polymorphism vs Type Inference]

    Polymorphism and type inference are distinct concepts: \textbf{polymorphism} allows a function to work uniformly over many types, while \textbf{type inference} is the compiler's ability to deduce types automatically. Polymorphism does not require inference, and \textbf{inference does not imply polymorphism}.\\

    \noindent
    Additionally, Parametric Polymorphism \textbf{cannot be used for dispatch} (inspecting types at runtime). 
    \end{Def}
    
    \noindent
    To implement such, there are two main systems:
    \begin{Def}[Implementing Polymorphism]

        Parametric polymorphism can be implemented in two main ways:
        \begin{itemize}
            \item \textbf{Hindley-Milner (OCaml):} Automatically infer the most general polymorphic type for an expression, without requiring explicit type annotations.
            \item \textbf{System F (Second-Order $\lambda$-Calculus):} Extend the language to take types as explicit arguments in functions.
        \end{itemize}
        Both approaches introduce the concept of a \textbf{type variable}, representing an unknown or arbitrary type.
        For example:
        \begin{lstlisting}[language=OCaml, numbers=none]
    let id : 'a -> 'a = fun x -> x
        \end{lstlisting}

        \noindent
        Here \texttt{'a} is a type variable, and the function \texttt{id} can accept any type as an argument and return a value of the same type.
    \end{Def}

    \noindent
    Though we will focus on OCaml, we discuss System F briefly.

    \newpage 
    \noindent
    
    \begin{Def}[Quantification]

        A polymorphic type like \texttt{'a -> 'a} is read as: 
        \large
        \begin{center}
            ``for any type \texttt{'a}, this function has type \texttt{'a -> 'a}.'' 
        
        \end{center}
        \normalsize

        \noindent
        Also notated as: \texttt{'a . 'a -> 'a},\ or,\ $\forall \alpha.\alpha \to \alpha$
    \end{Def}

    \noindent
    System F expands on this idea, providing extended syntax:
        
    \begin{Def}[System F Syntax]

        The following is System F syntax:
        \begin{lstlisting}[numbers=none, mathescape=true, escapeinside={(*}{*)}]
            e ::=  (*$\bullet$*)  |  x  |  (*$\lambda x^\tau. e$*)  |  e e  |  (*$\Lambda \alpha. e$*)  |  e (*$\tau$*)
            (*$\tau$*) ::=  (*$\top$*)  |  (*$\tau \to \tau$*)  |  (*$\alpha$*)  |  (*$\forall \alpha. \tau$*)
            x ::= variables
            (*$\alpha$*) ::= type variables
        \end{lstlisting}

        \noindent
    Notably: $\Lambda$ (capital lambda) refers to type variables in the same way $\lambda$ refers to expression variables. Moreover,
    $e$ and $\tau$ are expressions and types respectively.
    \end{Def}

    \begin{Def}[Polymorphic Identity Function]

        The identity function $\lambda x.x$ can be expressed in System F as a polymorphic function:
        \large 
        \[
        id \triangleq \Lambda \alpha. \lambda x^{\alpha}. x
        \]
        \normalsize

        \noindent
        This motivates application: ($id\ \tau) \rightarrow^\star (\lambda x^\tau.x):\tau\to\tau$. Note $\Lambda\alpha$ is dropped after substitution.
    \end{Def}

    \begin{Def}[System F Typing Rules]

        The typing rules for System F are as follows:
        \begin{center}
            \begin{prooftree}
                \hypo{}
                \infer1
                {\Gamma \vdash \bullet : \top}
            \end{prooftree}
            \quad
            \begin{prooftree}
                \hypo{(x : \tau) \in \Gamma}
                \infer1
                {\Gamma \vdash x : \tau}
            \end{prooftree}
            \quad
            \begin{prooftree}
                \hypo{\Gamma, x : \tau \vdash e : \tau'}
                \infer1
                {\Gamma \vdash \lambda x^\tau . e : \tau \to \tau'}
            \end{prooftree}
            \quad
            \begin{prooftree}
                \hypo{\Gamma \vdash e_1 : \tau \to \tau'}
                \hypo{\Gamma \vdash e_2 : \tau}
                \infer2
                {\Gamma \vdash e_1 e_2 : \tau'}
            \end{prooftree}
            \end{center}
            
            \vspace{1em}
            
            \begin{center}
            \begin{prooftree}
                \hypo{\Gamma \vdash e : \tau}
                \hypo{\alpha \text{ not free in } \Gamma}
                \infer2[(var abstr.)]
                {\Gamma \vdash \Lambda \alpha . e : \forall \alpha . \tau}
            \end{prooftree}
            \quad
            \begin{prooftree}
                \hypo{\Gamma \vdash e : \forall \alpha . \tau}
                \hypo{\tau' \text{ is a type}}
                \infer2[(type app.)]
                {\Gamma \vdash e \tau' : [\tau' / \alpha] \tau}
            \end{prooftree}
            \end{center}
            \noindent
            Unit, variable, abstraction, application, type abstraction, and type application respectively.
        \end{Def}            

\newpage

\noindent
Now to define how we handle our substitution:

\begin{Def}[System F Substitution]

    The rules for substitution in System F are as follows:
   
    \[  \hspace{-4.8em}
[\tau/\alpha]\, \top = \top
\]

\[
[\tau/\alpha]\, \alpha' =
\begin{cases}
    \tau     & \alpha' = \alpha \\
    \alpha'  & \texttt{else}
\end{cases}
\]

\[
\hspace{-.8em}
[\tau/\alpha](\tau_1 \to \tau_2) = [\tau/\alpha]\tau_1 \to [\tau/\alpha]\tau_2
\]

\[
\hspace{8.8em}
[\tau/\alpha](\forall \alpha' . \tau') =
\begin{cases}
    \forall \alpha' . \tau' & \alpha' = \alpha \\
    \forall \beta . [\tau/\alpha][\beta/\alpha']\tau' & \texttt{else } (\beta \text{ is fresh})
\end{cases}
\]
\end{Def}
    
\begin{Example}[Typing a System F Expression]

    \label{ex:system-f-typing}
    We derive the type of
    $(\Lambda \alpha . \lambda x^\alpha . x)\ (\top \to \top)\ \lambda x^\top . x
    $ in System F (read from bottom to top):
    
    \vspace{2em}
    \resizebox{.9\textwidth}{!}{%
    \begin{prooftree}
        \infer0{\{ x : \alpha \} \vdash x : \alpha}
        % Abs rule
        \infer1{\cdot\vdash \lambda x^\alpha . x : \alpha \to \alpha}
        % TyAbs rule
        \infer1{\cdot\vdash \Lambda \alpha . \lambda x^\alpha . x : \forall \alpha. \alpha \to \alpha}
        \Infer1{\textcolor{blue}{\cdot \vdash (\Lambda \alpha . \lambda x^\alpha . x)\ (\top \to \top) : (\top \to \top) \to (\top \to \top)}}
        \infer0{\{ x : \top \} \vdash x : \top}
        \Infer1{\textcolor{red}{\cdot \vdash \lambda x^\top . x : \top \to \top}}
        \infer2{\cdot \vdash (\Lambda \alpha . \lambda x^\alpha . x)\ (\top \to \top)\ \lambda x^\top . x : \top \to \top}
    \end{prooftree}

}
    
    \end{Example}
    \newpage 

\noindent
% \DTsetlength{offset}{width}{sep}{rule-width}{dot-size}
\DTsetlength{.2em}{1.5em}{.2em}{0.4pt}{0pt}

\noindent
We switch from doing bottom up proof trees, to a top down file tree structure to save on space:
\begin{Def}[File Tree Derivations]

    Given the above Example (\ref{ex:system-f-typing}), we represent it as a file tree:

    \noindent
    \rule{\textwidth}{0.4pt}
    \vspace{1em}
    \LARGE
    \dirtree{%
    .1 \ensuremath{\cdot \vdash (\Lambda \alpha . \lambda x^\alpha . x)\ (\top \to \top)\ \lambda x^\top . x : \top \to \top}.
    .2 \ensuremath{\textcolor{red}{\cdot \vdash \lambda x^\top . x : \top \to \top}}.
    .3 \ensuremath{\{ x : \top \} \vdash x : \top}.
    .2 \ensuremath{\textcolor{blue}{\cdot \vdash (\Lambda \alpha . \lambda x^\alpha . x)\ (\top \to \top) : (\top \to \top) \to (\top \to \top)}}.
    .3 \ensuremath{\cdot \vdash \Lambda \alpha . \lambda x^\alpha . x : \forall \alpha. \alpha \to \alpha}.
    .4 \ensuremath{\cdot \vdash \lambda x^\alpha . x : \alpha \to \alpha}.
    .5 \ensuremath{\{ x : \alpha \} \vdash x : \alpha}.
    }
    \normalsize

    \vspace{1em}
    \noindent
    \rule{\textwidth}{0.4pt}\\


    \noindent
    Where the conclusion is the root node, each directory level defines the premises 
    for the parent node, and the leaf nodes are the base cases.
\end{Def}
    
\noindent

\begin{Def}[Hindley-Milner Type Systems Corollary]

    \textbf{A Hindley-Milner (HM)} enables automatic type inference of polymorphic types of non-explicitly typed expressions.
    It supports a limited form of polymorphism where type variables are always quantified at the outermost level (e.g., $\forall \alpha.\forall \beta.\alpha \to \beta$, not $\forall \alpha.\alpha.\to \forall \beta.\beta\to\alpha$).
    
    These systems power languages like OCaml and Haskell, and make type inference both \textbf{decidable} and fairly \textbf{efficient}.
    \end{Def}

\noindent
HM does this by employing a constraint-based approach to type inference:
\begin{Def}[Type Inference with Constraints]

    In Hindley-Milner type inference, we aim to assign the most general type $\tau$ to an expression $e$, while collecting a set of constraints $\mathcal{C}$ that must hold for $\tau$ to be valid.
    If the type of a subexpression is unknown, we generate a fresh type variable to stand in for it.
    
    \vspace{-1em}
    \LARGE
    \[
    \Gamma \vdash e : \tau \dashv \mathcal{C}
    \]
    \normalsize
    
    \noindent
    Meaning, under context $\Gamma$, expression $e$ has type $\tau$ if constraints $\mathcal{C}$ are satisfied.
\end{Def}

\newpage 


\noindent
What are constraints?

\begin{Def}[Type Constraint and Unification]

    A \textbf{type constraint} is a requirement that two types must be equal. We write this as:
    
    \vspace{-1.5em}
    \LARGE
    \[
    \tau_1 \doteq \tau_2
    \]
    \normalsize
    
    \noindent
    This means ``$\tau_1$ should be the same as $\tau_2$.''
    Solving such a constraint—i.e., making $\tau_1$ and $\tau_2$ equal—is called \textbf{unification}.
    In particular, we are unifying $\tau_1$ and $\tau_2$.
    \end{Def}
    
\noindent

\noindent
Next, to introduce Hindley-Milner syntax: 
\begin{Def}[Expressions and Types in Hindley-Milner]

    We define the syntax of expressions and types under the Hindley-Milner type system:
    
    \begin{lstlisting}[numbers=none, mathescape=true, escapeinside={(*}{*)}]
        e ::=  (*$\lambda x. e$*)  |  (*$e\ e$*) 
            |  let (*$x = e$*) in (*$e$*) 
            |  if (*$e$*) then (*$e$*) else (*$e$*) 
            |  (*$e + e$*)  |  (*$e = e$*) 
            |  (*$n$*) 
            |  (*$x$*)

        (*$\sigma$*) ::= int  |  bool  |  (*$\alpha$*)  |  (*$\sigma \to \sigma$*)  (monotypes)
        
        (*$\tau$*) ::= (*$\sigma$*)  |  (*$\forall \alpha. \tau$*)                    (type schemes)
    \end{lstlisting}
    
    \noindent
    \textbf{Monotypes} ($\sigma$), are types without any quantification. A type is called \textbf{monomorphic} if it is a monotype
    with no type variables.\\
    
    \noindent
    \textbf{Type schemes} ($\tau$), allow quantification over type variables via the $\forall$ operator. These represent \textbf{polymorphic types}, and a type is polymorphic if it is a closed type scheme; Meaning,
    it contains no free type variables.
    
    \end{Def}

\begin{Example}[OCaml Quantification]

    OCaml is a Hindley-Milner type system, and it allows for polymorphic types. For example, the identity function can be expressed as:
    \begin{lstlisting}[language=OCaml, numbers=none]
        let id : 'a . 'a -> 'a = fun x -> x
    \end{lstlisting}
    
    \noindent
    When placed in utop it returns:
    \begin{lstlisting}[language=OCaml, numbers=none]
        val id : 'a -> 'a = <fun>
    \end{lstlisting}
\end{Example}
    
\newpage
\noindent 
For now we introduce a reduced form of the Hindley-Milner typing rules:
\begin{Def}[Hindly-Milner Light]

    \textbf{Hindley-Milner Light (HM$^-$)} contains the following typing rules:

        \begin{center}
        \begin{prooftree}
            \hypo{n \text{ is an integer}}
            \infer1[(int)]{\Gamma \vdash n : \texttt{int} \dashv \varnothing}
        \end{prooftree}
        
        \vspace{1.5em}
        \begin{prooftree}
            \hypo{\Gamma \vdash e_1 : \tau_1 \dashv \mathcal{C}_1}
            \hypo{\Gamma \vdash e_2 : \tau_2 \dashv \mathcal{C}_2}
            \infer2[(add)]{\Gamma \vdash e_1 + e_2 : \texttt{int} \dashv \tau_1 \doteq \texttt{int},\ \tau_2 \doteq \texttt{int},\ \mathcal{C}_1,\ \mathcal{C}_2}
        \end{prooftree}
        
        \vspace{1.5em}
        \begin{prooftree}
            \hypo{\Gamma \vdash e_1 : \tau_1 \dashv \mathcal{C}_1}
            \hypo{\Gamma \vdash e_2 : \tau_2 \dashv \mathcal{C}_2}
            \infer2[(eq)]{\Gamma \vdash e_1 = e_2 : \texttt{bool} \dashv \tau_1 \doteq \tau_2,\ \mathcal{C}_1,\ \mathcal{C}_2}
        \end{prooftree}
        
        \vspace{1.5em}
        \begin{prooftree}
            \hypo{\Gamma \vdash e_1 : \tau_1 \dashv \mathcal{C}_1}
            \hypo{\Gamma \vdash e_2 : \tau_2 \dashv \mathcal{C}_2}
            \hypo{\Gamma \vdash e_3 : \tau_3 \dashv \mathcal{C}_3}
            \infer3[(if)]{
                \Gamma \vdash \texttt{if } e_1 \texttt{ then } e_2 \texttt{ else } e_3 : \tau_3 \dashv \tau_1 \doteq \texttt{bool},\ \tau_2 \doteq \tau_3,\ \mathcal{C}_1,\ \mathcal{C}_2,\ \mathcal{C}_3
            }
        \end{prooftree}
        
        \vspace{1.5em}
        \begin{prooftree}
            \hypo{\alpha \text{ is fresh} \quad \Gamma, x : \alpha \vdash e : \tau \dashv \mathcal{C}}
            \infer1[(fun)]{\Gamma \vdash \lambda x . e : \alpha \to \tau \dashv \mathcal{C}}
        \end{prooftree}
        
        \vspace{1.5em}
        \begin{prooftree}
            \hypo{\Gamma \vdash e_1 : \tau_1 \dashv \mathcal{C}_1}
            \hypo{\Gamma \vdash e_2 : \tau_2 \dashv \mathcal{C}_2 \quad \alpha \text{ is fresh}}
            \infer2[(app)]{\Gamma \vdash e_1\ e_2 : \alpha \dashv \tau_1 \doteq \tau_2 \to \alpha,\ \mathcal{C}_1,\ \mathcal{C}_2}
        \end{prooftree}
        
        \vspace{1.5em}
        \begin{prooftree}
            \hypo{(x : \forall \alpha_1 \ldots \forall \alpha_k . \tau) \in \Gamma \quad \beta_1, \ldots, \beta_k \text{ are fresh}}
            \infer1[(var)]{
                \Gamma \vdash x : [\beta_1/\alpha_1] \ldots [\beta_k/\alpha_k]\tau \dashv \varnothing
            }
        \end{prooftree}
        
        \vspace{1.5em}  
        \begin{prooftree}
            \hypo{\Gamma \vdash e_1 : \tau_1 \dashv \mathcal{C}_1}
            \hypo{\Gamma, x : \tau_1 \vdash e_2 : \tau_2 \dashv \mathcal{C}_2}
            \infer2[(let)]{\Gamma \vdash \texttt{let } x = e_1\ \texttt{in } e_2 : \tau_2 \dashv \mathcal{C}_1, \mathcal{C}_2}
          \end{prooftree}
          
        \end{center}
        
        \vspace{1em}
        \noindent 
        This differs from the actual Hindley-Milner typing rules as our (let) is not polymorphic.\\
        
        \noindent
        The (add) rule reads, given a fixed-context $\Gamma$, if we have integer addition of two expressions $e_1$ and $e_2$, we inspect their types left to right;
        $e_1$ is of type $\tau_1$ under constraints $\mathcal{C}_1$. Such constraints may arise if $e_1$ is complex (e.g., it's another addition which returns its own constraints; Otherwise, if it is an monotype (base-case), the constraint is empty).
        Then we do the same for $e_2$.Finally, me make the declaration that $\tau_1$ and $\tau_2$ must be of type \texttt{int} (i.e., $\tau_1 \doteq \texttt{int}$ and $\tau_2 \doteq \texttt{int}$), and then union any accumulated constraints from both expressions.\\

        \noindent
        The (var) case reads that for
        any arbitrary number of quantifiers (typing-scheme, $\forall \alpha_1 \ldots \forall \alpha_k . \tau$), we substitute a fresh type variable $\beta_i$ for each $\alpha_i$.
        Therefore for expressions such as, ``\texttt{if} $e_1$ \texttt{then} $e_2$ \texttt{else} $e_3$,'' our place-holder $\tau_1, \tau_2, \tau_3$ types do not conflict with each other.

\end{Def}
        
    
\newpage 