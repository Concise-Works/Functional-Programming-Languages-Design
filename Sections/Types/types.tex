\newpage 

\section{Type Theory}
\subsection{Simply Typed Lambda Calculus}

\label{sec:types}

\noindent
An additional way to protect and reduce ambiguity in programming languages is to use \textbf{types}:

\begin{Def}[A Type]

    A \textbf{type} is a syntactic object that describes 
    the kind of values that an expression pattern is allowed to take.
    This happens before evaluation to safeguard unintended behavior.
\end{Def}

\noindent
Recall our work in Section (\ref{sec:formalizing-ocaml-expressions}). We add the following:

\begin{Def}[Contexts \& Typing Judgments]

    \textbf{Contexts:} $\Gamma$ is a finite mapping of variables to types. \textbf{Typing Judgments:} $\Gamma \vdash e : \tau$, reads ``$e$ has type $\tau$ in context $\Gamma$''. It is 
    said that $e$ is \textbf{well-typed} if $\cdot \vdash e : \tau$ for some $\tau$, where ($\cdot$) is the \textbf{empty context}.
    Such types we may inductively define:\\
    \begin{minipage}{0.45\textwidth}
        \begin{align*}
            \Gamma &::= \cdot \mid \Gamma, x : \tau \\
            x &::= vars\\
            \tau &::= types
        \end{align*}
    \end{minipage}
    \hfill
    \begin{minipage}{0.45\textwidth}
        \[
        \frac{
          \Gamma \vdash e_1 : \tau_1 \quad \cdots \quad \Gamma \vdash e_k : \tau_k
        }{
          \Gamma \vdash e : \tau
        }
        \]
    \end{minipage}

    \vspace{.5em}
    \noindent
    In practice, a context is a set (or ordered list) of variable declarations (variable-type pairs). Our 
    inference rules operate with these contexts to determine the type of an expression:
\end{Def}

\noindent 
This leads us to an extension of lambda calculus:
\begin{Def}[Simply Typed Lambda Calculus (STLC)]

    The syntax of the Simply Typed Lambda Calculus (STLC) extends the lambda calculus by including types and a unit expression.
    
    \begin{lstlisting}[numbers=none, mathescape=true]
    <e>  ::=  () | <v> | <e> <e> 
          |   fun "("<v> : <ty>")" -> <e>
    <ty> ::=  unit | <ty> -> <ty>
    <v>  ::=  [a-zA-Z]
    \end{lstlisting}
    
    \noindent
    We include the unit type (arbitrary value/void) and that functions are now typed. We transition into 
    a more mathematical notation:

\[
\begin{array}{ll}
e ::= & \bullet \mid x \mid \lambda x^{\tau}.\, e \mid e\,e \\
\tau ::= & \top \mid \tau \rightarrow \tau \\
x ::= & \textit{variables}
\end{array}
\]

\end{Def}
    


\newpage 
\noindent
This brings us to the typing rules for STLC:

\begin{Def}[Typing Rules for STLC]

    \textbf{Typing Rules:} The typing rules for STLC are as follows:
    \[
\begin{array}{cc}
\begin{prooftree}
  \hypo{}
  \infer1[\textsf{(unit)}]{\Gamma \vdash \bullet : \top}
\end{prooftree}
&
\begin{prooftree}
  \hypo{\Gamma, x{:}\tau \vdash e : \tau'}
  \infer1[\textsf{(abstraction)}]{\Gamma \vdash \lambda x^{\tau} .\, e : \tau \rightarrow \tau'}
\end{prooftree}
\\[2em]
\begin{prooftree}
  \hypo{(x{:}\tau) \in \Gamma}
  \infer1[\textsf{(variable)}]{\Gamma \vdash x : \tau}
\end{prooftree}
&
\begin{prooftree}
  \hypo{\Gamma \vdash e_1 : \tau \rightarrow \tau'}
  \hypo{\Gamma \vdash e_2 : \tau}
  \infer2[\textsf{(application)}]{\Gamma \vdash e_1 e_2 : \tau'}
\end{prooftree}
\end{array}
\]
\noindent
Such rules enforce that application is only valid when the $e_1$ position is a function type and the $e_2$ position is a valid argument type.
\end{Def}

When encountering notation, types are often omitted in some contexts:
\begin{Def}[Church vs. Curry Typing]

There are two main styles of typing:\\

\noindent
 \textbf{Curry-style typing:} Typing is \textbf{implied} (extrinsic) via typing judgement: 
    \begin{lstlisting}[numbers=none, mathescape=true]
    fun x -> x
    \end{lstlisting}
    
\vspace{1em}
\noindent
\textbf{Church-style typing:} Types are \textbf{explicitly} (intrinsic) annotated in the expression:
    \begin{lstlisting}[numbers=none, mathescape=true]
    fun (x : unit) -> x
    \end{lstlisting}


\noindent
\textbf{Important:} Curry-style does not imply polymorphism, expressions are judgement-backed.

\end{Def}

\noindent 
This leads us the an important lemma:
\begin{Def}[Lemma -- Uniqueness of Types ]

    Let $\Gamma$ be a typing context and $e$ a well-formed expression in STLC:
    \Large
    \begin{center}
        If $\Gamma \vdash e : \tau_1$ and $\Gamma \vdash e : \tau_2$, then $\tau_1 = \tau_2$.
    
    \end{center}
    \normalsize

    \noindent
    I.e., typing in STLC is \textbf{deterministic} -- a well-typed expression has a \textbf{unique type} under any fixed context.
    
    \end{Def}

    \newpage 

\noindent 
To prove the above lemma we must recall structural induction:
\begin{Def}[Structural Induction]

    \textbf{Structural induction} is a proof technique used to prove properties of recursively defined structures. It consists of two parts:
    \begin{itemize}
        \item \textbf{Base case:} Prove the property for the simplest constructor (e.g., a variable or unit).
        \item \textbf{Inductive step:} Assume the property holds for immediate substructures, and prove it holds for the structure built from them.
    \end{itemize}
    
    \noindent
    This differs from standard mathematical induction over natural numbers, where the base case is typically $n = 0$ (or $1$), and the inductive step proves $(n+1)$ assuming $(n)$.
    
    \noindent
    In the context of \textbf{lambda calculus}, expressions are recursively defined and built from smaller expressions. Structural induction proceeds as:
    \begin{itemize}
        \item \textbf{Base case:} Prove the property for the simplest expressions (e.g., variables and units).
        \item \textbf{Inductive step:} Assume the property holds for sub-expressions $e_1, e_2, \ldots, e_k$, and prove it holds for a compound expression $e$ (e.g., abstractions and application).
    \end{itemize}
    
    \end{Def}
    
\noindent
A proof of the previous lemma is as follows:
\begin{Proof}[Lemma -- Uniqueness of Types]

    We prove that if $\Gamma \vdash e : \tau_1$ and $\Gamma \vdash e : \tau_2$, then $\tau_1 = \tau_2$ ($\Gamma$ is a fixed typing context, and $e$ a well-formed expression)
    via structural-induction:\\

    \noindent
    \textbf{Case 1:} $e = \bullet$ (unit value).  
    Since $\bullet\Downarrow\top$ it is an atom. Therefore, $\tau_1 = \tau_2 = \top$.
    \\

    \noindent
    \textbf{Case 2:} $e = x$ (a variable).  
    Since $(x : \tau_1), (x : \tau_2) \in \Gamma$ and $\Gamma$ is fixed, $x$ maps to a single type. Thus, $\tau_1 = \tau_2$.
    \\

    \noindent
    \textbf{Case 3:} $e = \lambda x^\tau. e'$.  
    Abstraction typings require the form $\tau \to \tau'$. Both derivations,
    $\Gamma \vdash \lambda x^\tau. e' : \tau_1$ and $\Gamma \vdash \lambda x^\tau. e' : \tau_2$,
    must have such form. Hence, by inductive hypothesis on $e'$ (under $\Gamma,x:\tau$), we conclude $\tau_1 = \tau_2$.
    \\ 

    \noindent
    \textbf{Case 4:} $e = e_1\, e_2$ (application).  
    Suppose $\Gamma \vdash e_1\, e_2 : \tau_1$ and $\Gamma \vdash e_1\, e_2 : \tau_2$.  
    Then $e_1$ must have type $\tau' \to \tau_1$ and $\tau' \to \tau_2$ respectively, and $e_2$ type $\tau'$.  
    With likewise reasoning from Case 3, and through the inductive hypothesis on $e_1$ and $e_2$, we conclude $\tau_1 = \tau_2$.
    \\

    \noindent
    Hence, by induction on the typing derivation, the type assigned to any expression is unique.

    \end{Proof}
        
    