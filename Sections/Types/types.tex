\newpage 

\section{Type Theory}
\subsection{Simply Typed Lambda Calculus}

\label{sec:types}

\noindent
An additional way to protect and reduce ambiguity in programming languages is to use \textbf{types}:

\begin{Def}[A Type]

    A \textbf{type} is a syntactic object that describes 
    the kind of values that an expression pattern is allowed to take.
    This happens before evaluation to safeguard unintended behavior.
\end{Def}

\noindent
Recall our work in Section (\ref{sec:formalizing-ocaml-expressions}). We add the following:

\begin{Def}[Contexts \& Typing Judgments]

    \textbf{Contexts:} $\Gamma$ is a finite mapping of variables to types. \textbf{Typing Judgments:} $\Gamma \vdash e : \tau$, reads ``$e$ has type $\tau$ in context $\Gamma$''. It is 
    said that $e$ is \textbf{well-typed} if $\cdot \vdash e : \tau$ for some $\tau$, where ($\cdot$) is the \textbf{empty context}.
    Such types we may inductively define:\\
    \begin{minipage}{0.45\textwidth}
        \begin{align*}
            \Gamma &::= \cdot \mid \Gamma, x : \tau \\
            x &::= vars\\
            \tau &::= types
        \end{align*}
    \end{minipage}
    \hfill
    \begin{minipage}{0.45\textwidth}
        \[
        \frac{
          \Gamma \vdash e_1 : \tau_1 \quad \cdots \quad \Gamma \vdash e_k : \tau_k
        }{
          \Gamma \vdash e : \tau
        }
        \]
    \end{minipage}

    \vspace{.5em}
    \noindent
    In practice, a context is a set (or ordered list) of variable declarations (variable-type pairs). Our 
    inference rules operate with these contexts to determine the type of an expression:
\end{Def}

\noindent 
This leads us to an extension of lambda calculus:
\begin{Def}[Simply Typed Lambda Calculus (STLC)]

    The syntax of the Simply Typed Lambda Calculus (STLC) extends the lambda calculus by including types and a unit expression.
    
    \begin{lstlisting}[numbers=none, mathescape=true]
    <e>  ::=  () | <v> | <e> <e> 
          |   fun "("<v> : <ty>")" -> <e>
    <ty> ::=  unit | <ty> -> <ty>
    <v>  ::=  [a-zA-Z]
    \end{lstlisting}
    
    \noindent
    We include the unit type (arbitrary value/void) and that functions are now typed. We transition into 
    a more mathematical notation:

\[
\begin{array}{ll}
e ::= & \bullet \mid x \mid \lambda x^{\tau}.\, e \mid e\,e \\
\tau ::= & \top \mid \tau \rightarrow \tau \\
x ::= & \textit{variables}
\end{array}
\]

\end{Def}
    


\newpage 
\noindent
This brings us to the typing rules for STLC:


\begin{Def}[Typing Rules for STLC]

    \label{def:stlc-typing-rules}
    \textbf{Typing Rules:} The typing rules for STLC are as follows:
    \[
\begin{array}{cc}
\begin{prooftree}
  \hypo{}
  \infer1[\textsf{(unit)}]{\Gamma \vdash \bullet : \top}
\end{prooftree}
&
\begin{prooftree}
  \hypo{\Gamma, x{:}\tau \vdash e : \tau'}
  \infer1[\textsf{(abstraction)}]{\Gamma \vdash \lambda x^{\tau} .\, e : \tau \rightarrow \tau'}
\end{prooftree}
\\[2em]
\begin{prooftree}
  \hypo{(x{:}\tau) \in \Gamma}
  \infer1[\textsf{(variable)}]{\Gamma \vdash x : \tau}
\end{prooftree}
&
\begin{prooftree}
  \hypo{\Gamma \vdash e_1 : \tau \rightarrow \tau'}
  \hypo{\Gamma \vdash e_2 : \tau}
  \infer2[\textsf{(application)}]{\Gamma \vdash e_1 e_2 : \tau'}
\end{prooftree}
\end{array}
\]
\noindent
Such rules enforce that application is only valid when the $e_1$ position is a function type and the $e_2$ position is a valid argument type.
\end{Def}

When encountering notation, types are often omitted in some contexts:
\begin{Def}[Church vs. Curry Typing]

There are two main styles of typing:\\

\noindent
 \textbf{Curry-style typing:} Typing is \textbf{implied} (extrinsic) via typing judgement: 
    \begin{lstlisting}[numbers=none, mathescape=true]
    fun x -> x
    \end{lstlisting}
    
\vspace{1em}
\noindent
\textbf{Church-style typing:} Types are \textbf{explicitly} (intrinsic) annotated in the expression:
    \begin{lstlisting}[numbers=none, mathescape=true]
    fun (x : unit) -> x
    \end{lstlisting}


\noindent
\textbf{Important:} Curry-style does not imply polymorphism, expressions are judgement-backed.

\end{Def}

\noindent 
This leads us the an important lemma:
\begin{Def}[Lemma -- Uniqueness of Types ]

    Let $\Gamma$ be a typing context and $e$ a well-formed expression in STLC:
    \Large
    \begin{center}
        If $\Gamma \vdash e : \tau_1$ and $\Gamma \vdash e : \tau_2$, then $\tau_1 = \tau_2$.
    
    \end{center}
    \normalsize

    \noindent
    I.e., typing in STLC is \textbf{deterministic} -- a well-typed expression has a \textbf{unique type} under any fixed context.
    
    \end{Def}

    \newpage 

\noindent 
To prove the above lemma we must recall structural induction:
\begin{Def}[Structural Induction]

    \textbf{Structural induction} is a proof technique used to prove properties of recursively defined structures. It consists of two parts:
    \begin{itemize}
        \item \textbf{Base case:} Prove the property for the simplest constructor (e.g., a variable or unit).
        \item \textbf{Inductive step:} Assume the property holds for immediate substructures, and prove it holds for the structure built from them.
    \end{itemize}
    
    \noindent
    This differs from standard mathematical induction over natural numbers, where the base case is typically $n = 0$ (or $1$), and the inductive step proves $(n+1)$ assuming $(n)$.
    
    \noindent
    In the context of \textbf{lambda calculus}, expressions are recursively defined and built from smaller expressions. Structural induction proceeds as:
    \begin{itemize}
        \item \textbf{Base case:} Prove the property for the simplest expressions (e.g., variables and units).
        \item \textbf{Inductive step:} Assume the property holds for sub-expressions $e_1, e_2, \ldots, e_k$, and prove it holds for a compound expression $e$ (e.g., abstractions and application).
    \end{itemize}
    
    \end{Def}
    
\noindent
% Generation Lemma
A proof of the previous lemma is as follows:
\begin{Proof}[Lemma -- Uniqueness of Types]

    We prove that if $\Gamma \vdash e : \tau_1$ and $\Gamma \vdash e : \tau_2$, then $\tau_1 = \tau_2$ ($\Gamma$ is a fixed typing context, and $e$ a well-formed expression)
    via structural-induction:\\

    \noindent
    \textbf{Case 1:} $e = \bullet$ (unit value).  
    We define a generation lemma (\textbf{*}): if $\Gamma \vdash \bullet : \tau \to \tau = \top$. Therefore, if 
    $\Gamma \vdash \bullet : \tau_1$ and $\Gamma \vdash \bullet : \tau_2$, then $\tau_1=\tau_2=\top$, by lemma (\textbf{*}).  
    \\

    \noindent
    \textbf{Case 2:} $e = x$ (a variable). 
    We define a generation lemma (\textbf{**}): $\Gamma,x\vdash \tau\to x:\tau \in \Gamma$ 
    Since $(x : \tau_1), (x : \tau_2) \in \Gamma$ and $\Gamma$ is fixed, $x$ maps to a single type. Thus, $\tau_1 = \tau_2$, by lemma (\textbf{**}).
    \\

    \noindent
    \textbf{Case 3:} $e = \lambda x^\tau. e'$.  
    Abstraction typings require the form $\tau \to \tau'$. Both derivations,
    $\Gamma \vdash \lambda x^\tau. e' : \tau_1$ and $\Gamma \vdash \lambda x^\tau. e' : \tau_2$,
    must have such form. Hence, by inductive hypothesis on $e'$ (under $\Gamma,x:\tau$), we conclude $\tau_1 = \tau_2$.
    \\ 

    \noindent
    \textbf{Case 4:} $e = e_1\, e_2$ (application).  
    Suppose $\Gamma \vdash e_1\, e_2 : \tau_1$ and $\Gamma \vdash e_1\, e_2 : \tau_2$.  
    Then $e_1$ must have type $\tau' \to \tau_1$ and $\tau' \to \tau_2$ respectively, and $e_2$ type $\tau'$.  
    With likewise reasoning from Case 3, and through the inductive hypothesis on $e_1$ and $e_2$, we conclude $\tau_1 = \tau_2$.
    \\

    \noindent
    Hence, by induction on the typing derivation, the type assigned to any expression is unique.

    \end{Proof}

    \newpage 

    \noindent
    We continue with the following theorems:
        
    \begin{theo}[Well-Typed Implies Well-Scoped]

        If $e$ is well-typed in $\Gamma$, then $e$ is well-scoped.
    \end{theo}

    \begin{Proof}[Well-Typed Implies Well-Scoped]

        We prove this by induction on the structure of a well-formed expression $e$:
        \begin{itemize}
            \item Base cases: $e = \bullet$ or $e = x$ (variable), as based on Definition (\ref{def:stlc-typing-rules}), maps to a single type. Hence, they are well-typed.
            \item Inductive cases:
            \begin{itemize}
                \item $e = \lambda x^{\tau}. e'$: The abstraction argument $x$ is bound and explicitly typed. By the inductive hypothesis, $e'$ is well-typed in $\Gamma, x : \tau$.
                \item $e = e_1\, e_2$: Expression $e_1$ must be a function type $\tau' \to \tau_1$ and $e_2$ must be of type $\tau'$. By the inductive hypothesis, both $e_1$ and $e_2$ are well-typed in $\Gamma$.
            \end{itemize}
        \end{itemize}

        \noindent
        Therefore by induction, if $e$ is well-typed, then all sub-expressions must also be bound, and hence well-scoped.
    \end{Proof}
        
    \noindent
    We've been assuming the following properties of our evaluation relation:
    \begin{theo}[Big-Step Soundness]

        If $\cdot \vdash e : \tau$ then there is a value $v$ such that $\langle \varnothing, e \rangle \Downarrow v$ and $\cdot \vdash v : \tau$.
        \end{theo}
        
        \noindent
        Or to be more specific with small-step evaluation:
    \begin{theo}[Progress \& Preservation]

    If $\cdot \vdash e : \tau$, then
    \begin{itemize}
        \item \textbf{(Progress)} Either $e$ is a value or there is an $e'$ such that $e \rightarrow e'$.
        \item \textbf{(Preservation)} If $\cdot \vdash e : \tau$ and $e \rightarrow e'$, then $\cdot \vdash e' : \tau$.
    \end{itemize}
    \end{theo}
        
    \newpage 

\begin{Note}
\textbf{Note:} The following use of the word ``\textbf{inversion}'', refers to reversing/backtracking the typing rule to recover the original premises.
\end{Note}

\vspace{-.5em}
\begin{Proof}[Progress \& Preservation]

    We prove progress and preservation via structural induction on a well-formed expression $e$:
    \begin{itemize}
        \item \textbf{Base cases:} $e = \bullet$, $e = x$ (well-scoped variable), or $e = \lambda x^{\tau}. e'$ are values by Definition(\ref{def:stlc-typing-rules}), so they satisfy progress and need not reduce.

        \item \textbf{Inductive case:} $e = e_1\, e_2$ (application):

        \begin{itemize}
            \item \emph{Progress.} By the inductive hypothesis, either $e_1$ and $e_2$ are values, or they can take a step. 
            If both are values, then $e_1$ must be a lambda abstraction (i.e., $\lambda x^{\tau}. e'$), and $e_1\, e_2$ can step to $[e_2 / x]e'$, given that $e'$ is the body of $e_1$. Thus, progress holds.

            \item \emph{Preservation.} Suppose $\Gamma \vdash e_1\, e_2 : \tau$. Then by inversion, $\Gamma \vdash e_1 : \tau_1 \to \tau$ and $\Gamma \vdash e_2 : \tau_1$ for some $\tau_1$. Assume $e_1 = \lambda x^{\tau_1}. e'$. Then $e_1\, e_2 \rightarrow [e_2/x]e'$. By the typing rule for abstractions, we have $\Gamma, x : \tau_1 \vdash e' : \tau$. Then by the \emph{Substitution Lemma}:

            \begin{quote}
                If $\Gamma \vdash e_2 : \tau_1$ and $\Gamma, x:\tau_1 \vdash e' : \tau$, then $\Gamma \vdash [e_2/x]e' : \tau$
            \end{quote}

            We have, $\Gamma \vdash [e_2/x]e' : \tau$, which proves preservation for this case.
        \end{itemize}
    \end{itemize}

    \noindent
    Hence by induction, if $e$ is well-typed, then either $e$ is a value or there exists $e'$ such that $e \rightarrow e'$, and if $e \rightarrow e'$, then $e'$ is of the same type as $e$.
\end{Proof}

\vspace{-.5em}
\begin{Example}[Determining Type of an Expression]

    We determine the smallest typing context $\Gamma$ for the expression $(\lambda x ^{(\top \to \top) \to \top} .\ x(\lambda z^{\top}.\ x(wz)))\ y$:
    \begin{align*}
         \varnothing \vdash (\lambda x ^{(\top \to \top) \to \top} .\ x(\lambda z^{\top}.\ x(wz)))\ y\ :\ ?&& \text{(Given)}\\
         \{y: (\top \to \top)\to\top\} \vdash  (\lambda x ^{(\top \to \top) \to \top} .\ x(\lambda z^{\top}.\ x(wz)))\ y\ :\ ?&& \text{(Application Arg.)}\\
            \{y: (\top \to \top)\to\top, x: (\top \to \top) \to \top\} \vdash x(\lambda z^{\top}.\ x(wz))\ :\ ?&& \text{(Abstraction Type Sub.)}
    \end{align*}

    \noindent
    We note that $(\lambda z^{\top}.\ x(wz))$ must be of type $(\top \to \top)$ to satisfy $x$:
    \begin{align*}
        \{y: (\top \to \top)\to\top, x: (\top \to \top) \to \top, z: \top\} \vdash x(wz)\ :\ ?&& \text{(Application Arg.)}
    \end{align*}

    \noindent
    We see $(wz)$ must be type $(\top \to \top)$ as well. We know $z$ is of type $\top$, therefore $w$ must accept such type. In addition, 
    $w$ must return type $(\top \to \top)$ for the application of $x$ to be valid. Hence, we conclude:
    
    $$\Gamma := \{y: (\top \to \top)\to\top, x: (\top \to \top) \to \top, z: \top, w: \top \to (\top \to \top)\}$$

    \noindent
    Since $x$ is the outermost abstraction, we can conclude that the output type is $\top$.
\end{Example}

\newpage 

\begin{Example}[Typing an Ocaml Expression]
    
    We find the typing context $\Gamma$ for the following expressions:
    \begin{lstlisting}[language=OCaml, numbers=none, mathescape=true]
    (*1*) fun f -> fun x -> f (x + 1)      : ?
    (*2*) let rec f x = f (f (x + 1)) in f : ?
    \end{lstlisting}
    
    \noindent
    1. We note the application of $f\ (x + 1)$. Therefore, $f$ must be an (\texttt{int}$\to$\texttt{?}), as addition returns an \texttt{int}. Subsequently,
    the $x$ used in such addition and the function argument, must also be an \texttt{int}. Instead of leaving (\texttt{f (x + 1)})'s type as ?, we give it some arbitrary type \texttt{'a}. 
    Hence:
    $$\Gamma := \{f: \texttt{int} \to \texttt{'a}, x: \texttt{int}\}$$
    \noindent
    With a final type of \texttt{(int -> 'a) -> int -> 'a} for the entire expression. As we pass the type of $f$, then $x$, and our body returns the $\texttt{'a}$.\\

    \noindent
    2. Again, we note the application of $f\ (x + 1)$. Therefore, $f$ must be an (\texttt{int}$\to$\texttt{?}). This function is enclosed
    within another $f$ yielding $f(f(x+1))$, therefore, $f$ must return an \texttt{int} to satisfy the outer $f$. Hence, we conclude $\Gamma$ as: 

    $$\Gamma := \{f: \texttt{int} \to \texttt{int}, x: \texttt{int}\}$$

\end{Example}

\subsection{Polymorphism}

\noindent
There are moments when we might redundantly define functions such as:

\begin{lstlisting}[language=OCaml, numbers=none]
    let rec rev_int (l : int list) : int list =
        match l with
        | [] -> []
        | x :: l -> rev l @ [x]
    let rec rev_string (l : string list) : string list =
        match l with
        | [] -> []
        | x :: l -> rev l @ [x]
\end{lstlisting}

\noindent
Here we have two functions that are identical in structure, but differ in type.

\begin{Def}[Polymorphism]

    \textbf{Polymorphism} is the ability of a function to operate on values of different types while using a single uniform interface (signature).
    There are two types:
    \begin{itemize}
        \item \textbf{Ad hoc:} The ability to \textbf{overload} (redefine) a function name to accept different types.
        \item \textbf{Parametric:} The ability to define a function that can accept any type as an argument, and return a value of the same type.
    \end{itemize}
\end{Def}

\newpage
\noindent
We will focus on \textbf{parametric polymorphism}, as simply overloading in OCaml redefines the function name.
An example of a parametric polymorphic function in OCaml is the identity function:
\begin{lstlisting}[language=OCaml, numbers=none]
    let id = fun x -> x (* 'a -> 'a *)
    let a = id 0        (* int *)
    let b = id (0 = 0)  (* bool *)
    let c = id id       (* (int -> int) *)
\end{lstlisting}

\begin{Def}[Polymorphism vs Type Inference]

    Polymorphism and type inference are distinct concepts: \textbf{polymorphism} allows a function to work uniformly over many types, while \textbf{type inference} is the compiler's ability to deduce types automatically. Polymorphism does not require inference, and \textbf{inference does not imply polymorphism}.\\

    \noindent
    Additionally, Parametric Polymorphism \textbf{cannot be used for dispatch} (inspecting types at runtime) as it is statically typed (types don't change at run-time).
    \end{Def}
    
    \noindent
    To implement such, there are two main systems:
    \begin{Def}[Implementing Polymorphism]

        Parametric polymorphism can be implemented in two main ways:
        \begin{itemize}
            \item \textbf{Hindley-Milner (OCaml):} Automatically infer the most general polymorphic type for an expression, without requiring explicit type annotations.
            \item \textbf{System F (Second-Order $\lambda$-Calculus):} Extend the language to take types as explicit arguments in functions.
        \end{itemize}
        Both approaches introduce the concept of a \textbf{type variable}, representing an unknown or arbitrary type.
        For example:
        \begin{lstlisting}[language=OCaml, numbers=none]
    let id : 'a -> 'a = fun x -> x
        \end{lstlisting}

        \noindent
        Here \texttt{'a} is a type variable, and the function \texttt{id} can accept any type as an argument and return a value of the same type.
    \end{Def}

    \noindent
    Though we will focus on OCaml, we discuss System F briefly.

    \newpage 
    \noindent
    
    \begin{Def}[Quantification]

        A polymorphic type like \texttt{'a -> 'a} is read as: 
        \large
        \begin{center}
            ``for any type \texttt{'a}, this function has type \texttt{'a -> 'a}.'' 
        
        \end{center}
        \normalsize

        \noindent
        Also notated as: \texttt{'a . 'a -> 'a},\ or,\ $\forall \alpha.\alpha \to \alpha$
    \end{Def}

    \noindent
    System F expands on this idea, providing extended syntax:
        
    \begin{Def}[System F Syntax]

        The following is System F syntax:
        \begin{lstlisting}[numbers=none, mathescape=true, escapeinside={(*}{*)}]
            e ::=  (*$\bullet$*)  |  x  |  (*$\lambda x^\tau. e$*)  |  e e  |  (*$\Lambda \alpha. e$*)  |  e (*$\tau$*)
            (*$\tau$*) ::=  (*$\top$*)  |  (*$\tau \to \tau$*)  |  (*$\alpha$*)  |  (*$\forall \alpha. \tau$*)
            x ::= variables
            (*$\alpha$*) ::= type variables
        \end{lstlisting}

        \noindent
    Notably: $\Lambda$ (capital lambda) refers to type variables in the same way $\lambda$ refers to expression variables. Moreover,
    $e$ and $\tau$ are expressions and types respectively.
    \end{Def}

    \begin{Def}[Polymorphic Identity Function]

        The identity function $\lambda x.x$ can be expressed in System F as a polymorphic function:
        \large 
        \[
        id \triangleq \Lambda \alpha. \lambda x^{\alpha}. x
        \]
        \normalsize

        \noindent
        This motivates application: ($id\ \tau) \rightarrow^\star (\lambda x^\tau.x):\tau\to\tau$. Note $\Lambda\alpha$ is dropped after substitution.
    \end{Def}

    \begin{Def}[System F Typing Rules]

        The typing rules for System F are as follows:
        \begin{center}
            \begin{prooftree}
                \hypo{}
                \infer1
                {\Gamma \vdash \bullet : \top}
            \end{prooftree}
            \quad
            \begin{prooftree}
                \hypo{(x : \tau) \in \Gamma}
                \infer1
                {\Gamma \vdash x : \tau}
            \end{prooftree}
            \quad
            \begin{prooftree}
                \hypo{\Gamma, x : \tau \vdash e : \tau'}
                \infer1
                {\Gamma \vdash \lambda x^\tau . e : \tau \to \tau'}
            \end{prooftree}
            \quad
            \begin{prooftree}
                \hypo{\Gamma \vdash e_1 : \tau \to \tau'}
                \hypo{\Gamma \vdash e_2 : \tau}
                \infer2
                {\Gamma \vdash e_1 e_2 : \tau'}
            \end{prooftree}
            \end{center}
            
            \vspace{1em}
            
            \begin{center}
            \begin{prooftree}
                \hypo{\Gamma \vdash e : \tau}
                \hypo{\alpha \text{ not free in } \Gamma}
                \infer2[(var abstr.)]
                {\Gamma \vdash \Lambda \alpha . e : \forall \alpha . \tau}
            \end{prooftree}
            \quad
            \begin{prooftree}
                \hypo{\Gamma \vdash e : \forall \alpha . \tau}
                \hypo{\tau' \text{ is a type}}
                \infer2[(type app.)]
                {\Gamma \vdash e \tau' : [\tau' / \alpha] \tau}
            \end{prooftree}
            \end{center}
            \noindent
            Unit, variable, abstraction, application, type abstraction, and type application respectively.
        \end{Def}            

\newpage

\noindent
Now to define how we handle our substitution:

\begin{Def}[System F Substitution]

    The rules for substitution in System F are as follows:
   
    \[  \hspace{-4.8em}
[\tau/\alpha]\, \top = \top
\]

\[
[\tau/\alpha]\, \alpha' =
\begin{cases}
    \tau     & \alpha' = \alpha \\
    \alpha'  & \texttt{else}
\end{cases}
\]

\[
\hspace{-.8em}
[\tau/\alpha](\tau_1 \to \tau_2) = [\tau/\alpha]\tau_1 \to [\tau/\alpha]\tau_2
\]

\[
\hspace{8.8em}
[\tau/\alpha](\forall \alpha' . \tau') =
\begin{cases}
    \forall \alpha' . \tau' & \alpha' = \alpha \\
    \forall \beta . [\tau/\alpha][\beta/\alpha']\tau' & \texttt{else } (\beta \text{ is fresh})
\end{cases}
\]
\end{Def}
    
\begin{Example}[Typing a System F Expression]

    \label{ex:system-f-typing}
    We derive the type of
    $(\Lambda \alpha . \lambda x^\alpha . x)\ (\top \to \top)\ \lambda x^\top . x
    $ in System F (read from bottom to top):
    
    \vspace{2em}
    \resizebox{.9\textwidth}{!}{%
    \begin{prooftree}
        \infer0{\{ x : \alpha \} \vdash x : \alpha}
        % Abs rule
        \infer1{\cdot\vdash \lambda x^\alpha . x : \alpha \to \alpha}
        % TyAbs rule
        \infer1{\cdot\vdash \Lambda \alpha . \lambda x^\alpha . x : \forall \alpha. \alpha \to \alpha}
        \Infer1{\textcolor{blue}{\cdot \vdash (\Lambda \alpha . \lambda x^\alpha . x)\ (\top \to \top) : (\top \to \top) \to (\top \to \top)}}
        \infer0{\{ x : \top \} \vdash x : \top}
        \Infer1{\textcolor{red}{\cdot \vdash \lambda x^\top . x : \top \to \top}}
        \infer2{\cdot \vdash (\Lambda \alpha . \lambda x^\alpha . x)\ (\top \to \top)\ \lambda x^\top . x : \top \to \top}
    \end{prooftree}

}
    
    \end{Example}
    \newpage 

\noindent
% \DTsetlength{offset}{width}{sep}{rule-width}{dot-size}
\DTsetlength{.2em}{1.5em}{.2em}{0.4pt}{0pt}

\noindent
We switch from doing bottom up proof trees, to a top down file tree structure to save on space:
\begin{Def}[File Tree Derivations]

    Given the above Example (\ref{ex:system-f-typing}), we represent it as a file tree:

    \noindent
    \rule{\textwidth}{0.4pt}
    \vspace{1em}
    \LARGE
    \dirtree{%
    .1 \ensuremath{\cdot \vdash (\Lambda \alpha . \lambda x^\alpha . x)\ (\top \to \top)\ \lambda x^\top . x : \top \to \top}.
    .2 \ensuremath{\textcolor{red}{\cdot \vdash \lambda x^\top . x : \top \to \top}}.
    .3 \ensuremath{\{ x : \top \} \vdash x : \top}.
    .2 \ensuremath{\textcolor{blue}{\cdot \vdash (\Lambda \alpha . \lambda x^\alpha . x)\ (\top \to \top) : (\top \to \top) \to (\top \to \top)}}.
    .3 \ensuremath{\cdot \vdash \Lambda \alpha . \lambda x^\alpha . x : \forall \alpha. \alpha \to \alpha}.
    .4 \ensuremath{\cdot \vdash \lambda x^\alpha . x : \alpha \to \alpha}.
    .5 \ensuremath{\{ x : \alpha \} \vdash x : \alpha}.
    }
    \normalsize

    \vspace{1em}
    \noindent
    \rule{\textwidth}{0.4pt}\\


    \noindent
    Where the conclusion is the root node, each directory level defines the premises 
    for the parent node, and the leaf nodes are the base cases.
\end{Def}
    
\noindent

\begin{Def}[Hindley-Milner Type Systems Corollary]

    \textbf{A Hindley-Milner (HM)} enables automatic type inference of polymorphic types of non-explicitly typed expressions.
    It supports a limited form of polymorphism where type variables are always quantified at the outermost level (e.g., $\forall \alpha.\forall \beta.\alpha \to \beta$, not $\forall \alpha.\alpha.\to \forall \beta.\beta\to\alpha$).
    
    These systems power languages like OCaml and Haskell, and make type inference both \textbf{decidable} and fairly \textbf{efficient}.
    \end{Def}

\noindent
HM does this by employing a constraint-based approach to type inference:
\begin{Def}[Type Inference with Constraints]

    In Hindley-Milner type inference, we aim to assign the most general type $\tau$ to an expression $e$, while collecting a set of constraints $\mathcal{C}$ that must hold for $\tau$ to be valid.
    If the type of a subexpression is unknown, we generate a fresh type variable to stand in for it.
    
    \vspace{-1em}
    \LARGE
    \[
    \Gamma \vdash e : \tau \dashv \mathcal{C}
    \]
    \normalsize
    
    \noindent
    Meaning, under context $\Gamma$, expression $e$ has type $\tau$ if constraints $\mathcal{C}$ are satisfied.
\end{Def}

\newpage 


\noindent
What are constraints?

\begin{Def}[Type Constraint]

    A \textbf{type constraint} is a requirement that two types must be equal. We write this as:
    
    \vspace{-1.5em}
    \LARGE
    \[
    \tau_1 \doteq \tau_2
    \]
    \normalsize
    
    \noindent
    This means ``$\tau_1$ should be the same as $\tau_2$.''
    Solving such a constraint—i.e., making $\tau_1$ and $\tau_2$ equal—is called \textbf{unification}.
    In particular, we are unifying $\tau_1$ and $\tau_2$.
    \end{Def}
    
\noindent

\noindent
Next, to introduce Hindley-Milner syntax: 
\begin{Def}[Expressions and Types in Hindley-Milner]

    We define the syntax of expressions and types under the Hindley-Milner type system:
    
    \begin{lstlisting}[numbers=none, mathescape=true, escapeinside={(*}{*)}]
        e ::=  (*$\lambda x. e$*)  |  (*$e\ e$*) 
            |  let (*$x = e$*) in (*$e$*) 
            |  if (*$e$*) then (*$e$*) else (*$e$*) 
            |  (*$e + e$*)  |  (*$e = e$*) 
            |  (*$n$*) 
            |  (*$x$*)

        (*$\sigma$*) ::= int  |  bool  |  (*$\alpha$*)  |  (*$\sigma \to \sigma$*)  (monotypes)
        
        (*$\tau$*) ::= (*$\sigma$*)  |  (*$\forall \alpha. \tau$*)                    (type schemes)
    \end{lstlisting}
    
    \noindent
    \textbf{Monotypes} ($\sigma$), are types without any quantification. A type is called \textbf{monomorphic} if it is a monotype
    with no type variables.\\
    
    \noindent
    \textbf{Type schemes} ($\tau$), allow quantification over type variables via the $\forall$ operator. These represent \textbf{polymorphic types}, and a type is polymorphic if it is a closed type scheme; Meaning,
    it contains no free type variables.
    
    \end{Def}

\begin{Example}[OCaml Quantification]

    OCaml utilizes the Hindley-Milner type system, enabling polymorphic types. For example, the identity function can be expressed as:
    \begin{lstlisting}[language=OCaml, numbers=none]
        let id : 'a . 'a -> 'a = fun x -> x
    \end{lstlisting}
    
    \noindent
    When placed in utop it returns (the quantifier is implicit):
    \begin{lstlisting}[language=OCaml, numbers=none]
        val id : 'a -> 'a = <fun>
    \end{lstlisting}
\end{Example}
    
\newpage
\noindent 
For now we introduce a reduced form of the Hindley-Milner typing rules:
\begin{Def}[Hindly-Milner Light]

    \label{def:hml}
    \textbf{Hindley-Milner Light (HM$^-$)} contains the following typing rules:

        \begin{center}
        \begin{prooftree}
            \hypo{n \text{ is an integer}}
            \infer1[(int)]{\Gamma \vdash n : \texttt{int} \dashv \varnothing}
        \end{prooftree}
        
        \vspace{1.5em}
        \begin{prooftree}
            \hypo{\Gamma \vdash e_1 : \tau_1 \dashv \mathcal{C}_1}
            \hypo{\Gamma \vdash e_2 : \tau_2 \dashv \mathcal{C}_2}
            \infer2[(add)]{\Gamma \vdash e_1 + e_2 : \texttt{int} \dashv \tau_1 \doteq \texttt{int},\ \tau_2 \doteq \texttt{int},\ \mathcal{C}_1,\ \mathcal{C}_2}
        \end{prooftree}
        
        \vspace{1.5em}
        \begin{prooftree}
            \hypo{\Gamma \vdash e_1 : \tau_1 \dashv \mathcal{C}_1}
            \hypo{\Gamma \vdash e_2 : \tau_2 \dashv \mathcal{C}_2}
            \infer2[(eq)]{\Gamma \vdash e_1 = e_2 : \texttt{bool} \dashv \tau_1 \doteq \tau_2,\ \mathcal{C}_1,\ \mathcal{C}_2}
        \end{prooftree}
        
        \vspace{1.5em}
        \begin{prooftree}
            \hypo{\Gamma \vdash e_1 : \tau_1 \dashv \mathcal{C}_1}
            \hypo{\Gamma \vdash e_2 : \tau_2 \dashv \mathcal{C}_2}
            \hypo{\Gamma \vdash e_3 : \tau_3 \dashv \mathcal{C}_3}
            \infer3[(if)]{
                \Gamma \vdash \texttt{if } e_1 \texttt{ then } e_2 \texttt{ else } e_3 : \tau_3 \dashv \tau_1 \doteq \texttt{bool},\ \tau_2 \doteq \tau_3,\ \mathcal{C}_1,\ \mathcal{C}_2,\ \mathcal{C}_3
            }
        \end{prooftree}
        
        \vspace{1.5em}
        \begin{prooftree}
            \hypo{\alpha \text{ is fresh} \quad \Gamma, x : \alpha \vdash e : \tau \dashv \mathcal{C}}
            \infer1[(fun)]{\Gamma \vdash \lambda x . e : \alpha \to \tau \dashv \mathcal{C}}
        \end{prooftree}
        
        \vspace{1.5em}
        \begin{prooftree}
            \hypo{\Gamma \vdash e_1 : \tau_1 \dashv \mathcal{C}_1}
            \hypo{\Gamma \vdash e_2 : \tau_2 \dashv \mathcal{C}_2 \quad \alpha \text{ is fresh}}
            \infer2[(app)]{\Gamma \vdash e_1\ e_2 : \alpha \dashv \tau_1 \doteq \tau_2 \to \alpha,\ \mathcal{C}_1,\ \mathcal{C}_2}
        \end{prooftree}
        
        \vspace{1.5em}
        \begin{prooftree}
            \hypo{(x : \forall \alpha_1 \ldots \forall \alpha_k . \tau) \in \Gamma \quad \beta_1, \ldots, \beta_k \text{ are fresh}}
            \infer1[(var)]{
                \Gamma \vdash x : [\beta_1/\alpha_1] \ldots [\beta_k/\alpha_k]\tau \dashv \varnothing
            }
        \end{prooftree}
        
        \vspace{1.5em}  
        \begin{prooftree}
            \hypo{\Gamma \vdash e_1 : \tau_1 \dashv \mathcal{C}_1}
            \hypo{\Gamma, x : \tau_1 \vdash e_2 : \tau_2 \dashv \mathcal{C}_2}
            \infer2[(let)]{\Gamma \vdash \texttt{let } x = e_1\ \texttt{in } e_2 : \tau_2 \dashv \mathcal{C}_1, \mathcal{C}_2}
          \end{prooftree}
          
        \end{center}
        
        \vspace{1em}
        \noindent 
        This differs from the actual Hindley-Milner typing rules as our (let) is not polymorphic.\\
        
        \noindent
        The (add) rule reads, given a fixed-context $\Gamma$, if we have integer addition of two expressions $e_1$ and $e_2$, we inspect their types left to right;
        $e_1$ is of type $\tau_1$ under constraints $\mathcal{C}_1$. Such constraints may arise if $e_1$ is complex (e.g., it's another addition which returns its own constraints; Otherwise, if it is an monotype (base-case), the constraint is empty).
        Then we do the same for $e_2$. Finally, we make the declaration that $\tau_1$ and $\tau_2$ must be of type \texttt{int} (i.e., $\tau_1 \doteq \texttt{int}$ and $\tau_2 \doteq \texttt{int}$), and then union any accumulated constraints from both expressions.\\

        \noindent
        The (var) case reads that for any variable $x$ encountered, we look to our context $\Gamma$ to find its type. If it is polymorphic (quantified), we replace the quantifiers with fresh type variables.
        If it is plainly monomorphic, we simply return its type.

\end{Def}
        
\newpage
\noindent
Finally to define this system's substitution behavior:
\begin{Def}[Substitution on Monotypes]

    Let \(\alpha\) be a type variable and \(\tau\) a monotype.  We define the result of substituting \(\tau\) for \(\alpha\) in any monotype \(\sigma\), written \([\tau/\alpha]\sigma\), by induction on the structure of \(\sigma\):
    
    \[
    \begin{array}{rcl}
    [\tau/\alpha]\,\texttt{int}
    &=& 
    \texttt{int},
    \\[6pt]
    [\tau/\alpha]\,\texttt{bool}
    &=& 
    \texttt{bool},
    \\[6pt]
    [\tau/\alpha]\,\beta
    &=&
    \begin{cases}
      \tau       & \text{if }\beta = \alpha,\\
      \beta      & \text{otherwise},
    \end{cases}
    \\[12pt]
    [\tau/\alpha](\sigma_1 \to \sigma_2)
    &=&
    \bigl([\tau/\alpha]\sigma_1\bigr)\;\to\;\bigl([\tau/\alpha]\sigma_2\bigr).
    \end{array}
    \]

    \noindent
    Since we are dealing with monotypes, we do not need to worry about quantification (needing a fresh variable).
    Our rule (var) in Definition \ref{def:hml} handles this for us.
    \end{Def}

    \begin{Example}[Deriving Constraints]

        Given the following expression $\texttt{(x + 1) + (2 + y)}$, and context $\{x:\alpha, y: \beta\}$, we find its type and constraints:\\

        \Large
        \dirtree{%
        .1 \ensuremath{\{x:\alpha, y: \beta\} \vdash (x + 1) + (2 + y) : \texttt{int} \dashv\ ?}.
        .2 \ensuremath{\{x:\alpha, y: \beta\} \vdash (x + 1) :\texttt{int} \dashv\ \mathcal{C}_1:=\{\alpha\doteq\texttt{int}, \texttt{int}\doteq\texttt{int}\}}.
        .3 \ensuremath{\{x:\alpha, y: \beta\} \vdash x : \alpha \dashv\ \varnothing}.
        .3 \ensuremath{\{x:\alpha, y: \beta\} \vdash 1 : \texttt{int} \dashv\ \varnothing}.
        }
        \normalsize

        \vspace{1em}
        \noindent
        We pause here in our derivation to show that we are unaware of the constraints (?) left side \texttt{(x + 1)}.
        We continue:\\

        \Large
        \dirtree{%
        .1 \ensuremath{\{x:\alpha, y: \beta\} \vdash (x + 1) + (2 + y) : \texttt{int} \dashv \mathcal{C}_3 :=\{\textcolor{red}{\texttt{int}}\doteq\texttt{int}, \textcolor{blue}{\texttt{int}}\doteq\texttt{int}\},\mathcal{C}_1, \mathcal{C}_2}.
        .2 \ensuremath{\{x:\alpha, y: \beta\} \vdash (x + 1) :\textcolor{red}{\texttt{int}} \dashv\ \mathcal{C}_1:=\{\alpha\doteq\texttt{int}, \texttt{int}\doteq\texttt{int}\}}.
        .3 \ensuremath{\{x:\alpha, y: \beta\} \vdash x : \alpha \dashv\ \varnothing}.
        .3 \ensuremath{\{x:\alpha, y: \beta\} \vdash 1 : \texttt{int} \dashv\ \varnothing}.
        .2 \ensuremath{\{x:\alpha, y: \beta\} \vdash (2 + y) :\textcolor{blue}{\texttt{int}} \dashv\ \mathcal{C}_2:=\{\beta\doteq\texttt{int}, \texttt{int}\doteq\texttt{int}\}}.
        .3 \ensuremath{\{x:\alpha, y: \beta\} \vdash 2 : \texttt{int} \dashv\ \varnothing}.
        .3 \ensuremath{\{x:\alpha, y: \beta\} \vdash y : \beta \dashv\ \varnothing}.
        }
        \normalsize

        \vspace{1em}
        \noindent
        Once we finish both the left (red) and right (blue) branches they return a type of \texttt{int}. This yields a final type of \texttt{int} and constraints:
        $\{\alpha\doteq\texttt{int}, \beta\doteq\texttt{int}, \texttt{int}\doteq\texttt{int}, \texttt{int}\doteq\texttt{int}\}$.
    \end{Example}
\newpage 
\subsection{Unification \& The Principle Type}

\noindent
Constraints provide a set of equations which we want to unify:
\begin{Def}[Unification, Unifiers, \& Most General Unifiers]

    A \textbf{Unification} problem $\mathcal{U}$ is a system of constraints of form:
    \begin{align*}
        S_1 &\doteq t_1 \\
        S_2 &\doteq t_2 \\
        &\hspace{.5em}\vdots \\
        S_k &\doteq t_k
    \end{align*}
    \noindent
    Where $S_1, \ldots, S_k$ and $t_1, \ldots, t_k$ are terms. Solving $U$ generates a \textbf{unifier} (solution) $\mathcal{S}$, which involves
    sequence of \textbf{ordered} substitutions of the form:
    \[ \mathcal{S} := \{x_1 \mapsto t_1, \ldots, x_k \mapsto t_k\} \]
    \noindent
    Where $St$ represents the application of all such mappings (in order) to $t$:
    \[St:= [t_n/x_n] \ldots [t_1/x_1]t\]
    This solution \textbf{must} satisfy every equation in $\mathcal{U}$, such that:
    \begin{align*}
        \mathcal{S}t_1 &= \mathcal{S}s_1 \\
        &\hspace{.5em}\vdots \\
        \mathcal{S}t_k &= \mathcal{S}s_k
    \end{align*}
    \noindent
    In particular, \underline{\textbf{not all}} unification problems have a solution (e.g., \texttt{int }$\doteq$\texttt{ bool}). Moreover,\\
    \textbf{The Most General Unifier (MGU)} is a unifier $\mathcal{S}$, such that all such other solutions $\mathcal{S}'$ are an extension of $\mathcal{S}$, where
    $\mathcal{S}' = \mathcal{S} \cup \mathcal{S}''$.
    In particular, $\mathcal{S}''$ is a substitution that does not change the type of any free variables in $\mathcal{S}$. Notably, the MGU is not unique. For example, $\{\alpha \mapsto \beta, \beta \mapsto \texttt{int}\}$ and $\{\alpha \mapsto \texttt{int}, \beta \mapsto \texttt{int}\}$ 
    are both MGUs for $\{\alpha \doteq \beta, \beta\doteq\texttt{int}\}$.\\
    
    \noindent
    \textbf{To solve} a unification problem $\mathcal{U}$, we may choose 
    an equation in any order and append either side of the equality as a new mapping to our unifier set $\mathcal{S}$, and remove it from $\mathcal{U}$. 
    After each new member, we apply their mapping to the remaining equations. Semantically we have:\\
    \begin{center}
    \begin{prooftree}
    \hypo{\alpha = \beta}
    \infer1[(Id.)]{\langle\, \mathcal{S}, \{\alpha \doteq \,\beta, R\} \rangle \to \langle\, \mathcal{S}, \{R\} \,\rangle}
    \end{prooftree}
    \quad
    \begin{prooftree}
    \hypo{\alpha \neq \beta}
    \hypo{\alpha \text{ is a type variable}}
    \hypo{\alpha\not\in FV(\beta)}
    \infer3[(Sub.)]{\langle\, \mathcal{S}, \{\alpha \doteq \beta, R\} \rangle \to \langle\, \mathcal{S} [\alpha\mapsto\beta],\{[\alpha/\beta]R\} \,\rangle}
    \end{prooftree}\\

    \vspace{1em}
    \begin{prooftree}
    \infer0[(Decomp.)]{\langle\, \mathcal{S}, \{\alpha\to\beta \doteq \gamma\to\delta, R\} \rangle \to \langle\, \mathcal{S}, \{\alpha \doteq \gamma, \beta \doteq \delta, R\} \,\rangle}
    \end{prooftree}
    
\end{center}

\noindent
Where $R$ is the remaining equations in $\mathcal{U}$. Such method results in an MGU.
\end{Def}

\newpage 

\noindent
The next example swaps from lambda abstractions to arrows for clarity:
\begin{Example}[Solving a Unification Problem]

    Say constraint $\mathcal{C}$ provides the system of equations and a unifier $\mathcal{S}$:
    \begin{align*}
        \mathcal{S} &:= \{\varnothing\}\\
        \gamma &\doteq \gamma\\ 
        \alpha &\doteq \delta \to \eta \\
        \beta &\doteq \texttt{int} \to \delta\\
        \texttt{int} \to \texttt{int} \to \texttt{int} &\doteq \sigma \to \beta
    \end{align*}
    \noindent
    We go from top to bottom, and assign from left to right (though order doesn't matter):
    \begin{align*}
    \mathcal{S} &:= \{\alpha \mapsto (\delta \to \eta)\}\\
    \beta &\doteq \texttt{int} \to \delta\\
    \texttt{int} \to \texttt{int} \to \texttt{int} &\doteq \sigma \to \beta
    \end{align*}
    \noindent
    We collapsed the first two equations, the $\gamma$s add nothing to the unifier. We then add the $\alpha$ mapping to the unifier.
    We continue:
    \begin{align*} 
        \mathcal{S} &:= \{\alpha \mapsto \delta \to \eta), \beta \mapsto (\texttt{int} \to \delta)\}\\
        \texttt{int} \to \texttt{int} \to \texttt{int} &\doteq \sigma \to (\texttt{int} \to \delta)
    \end{align*}
    \noindent
    Here, the $\beta$ mapping is added to the unifier and substituted into the rest of the equations. Continuing:
    \begin{align*}
        \mathcal{S} &:= \{\alpha \mapsto \delta \to \eta), \beta \mapsto (\texttt{int} \to \delta)\}\\
        \texttt{int} &\doteq \sigma\\
        \texttt{int} \to \texttt{int} &\doteq (\texttt{int} \to \delta)
    \end{align*}
    \noindent
    Decomposing the abstractions, the rule of thumb is to match the first arguments with each other (e.g., $\alpha\to\gamma \doteq \beta\to\beta\to\beta$) is decomposed to $\alpha \doteq \beta$, $\gamma \doteq \beta\to\beta$). Continuing:
    \begin{align*}
        \mathcal{S} := \{\alpha \mapsto \delta \to \eta), \beta \mapsto &(\texttt{int} \to \delta), \sigma \mapsto \texttt{int}\}\\
        \texttt{int} &\doteq \texttt{int}\\
        \texttt{int} &\doteq \delta
    \end{align*}
    \noindent
    We have added the $\sigma$ mapping to the unifier, nothing to substitute, and decomposed the bottom abstraction. Finally:
    \begin{align*}
        \mathcal{S} := \{\alpha \mapsto \delta \to \eta), \beta \mapsto &(\texttt{int} \to \delta), \sigma \mapsto \texttt{int},\delta \mapsto \texttt{int}\}
    \end{align*}
    \noindent
    We discarded the trivial $\texttt{int}$ equality, and are left with the above unifier.
\end{Example}

\newpage 

\noindent
We may derive an algorithm for our unification problem:
\begin{Func}[Type Unification Algorithm - \textit{Unify()}]
    \noindent
    \textbf{Input:} $\mathcal{U}$, a type unification problem \\
    \textbf{Output:} $\mathcal{S}$, the most general unifier (MGU) for $\mathcal{U}$ \\

    \vspace{-.5em}
    \begin{algorithm}[H]
        \SetAlgoLined
        \SetKwProg{Fn}{Function}{:}{\KwRet}
        \Fn{\textit{Unify}($\,\mathcal{U}$)}{
            $\mathcal{S} \gets$ empty solution

            WHILE $eq \in \mathcal{U}$ \quad // $\mathcal{U}$ is not empty

            \Indp
            MATCH $eq$:

            $t_1 \doteq t_2 \Longrightarrow \mathcal{U} \gets \mathcal{U} \setminus \{eq\}$ \quad // if $t_1$ and $t_2$ are syntactically equal

            $s_1 \to t_1 \doteq s_2 \to t_2 \Longrightarrow \mathcal{U} \gets \mathcal{U} \setminus \{eq\} \cup \{s_1 \doteq s_2,\ t_1 \doteq t_2\}$ \quad // split arrow type

            $\alpha \doteq t$ or $t \doteq \alpha$ where $\alpha \notin FV(t) \Longrightarrow$

            \Indp
            $\mathcal{S} \gets \mathcal{S} \cup \{\alpha \mapsto t\}$ \quad // add substitution

            $\mathcal{U} \gets \mathcal{U} \setminus \{eq\}$

            perform the substitution $\alpha \mapsto t$ to every equation in $\mathcal{U}$

            \Indm
            OTHERWISE $\Longrightarrow$ FAIL

            
            \hspace{-1.5em}RETURN $\mathcal{S}$
        }
    \end{algorithm}
\end{Func}

\noindent
Now to actually use our unifier to derive a type:
\begin{Def} [The Principle Type (Generalization)]

    Under some expression $e$, we first make a constraint-based type inference, determining $\tau$ and $\mathcal{C}$ such that:
    $\Gamma \vdash e : \tau \dashv \mathcal{C}$.

    
    We must first check $\tau$ against its constraints. Hence, we solve the unification problem $\mathcal{C}$ to find a unifier $\mathcal{S}$. If 
    we fail to find a unifier, then there must be a type error in the expression $e$. Otherwise, we may proceed with \textbf{generalization}:
    \[\forall \alpha_1, \ldots, \alpha_k . \mathcal{S}\tau \text{ where } FV(\mathcal{S}\tau) = \{\alpha_1, \ldots, \alpha_k\}\]
    \noindent
    I.e., we take all such free type variables in $\mathcal{S\tau}$ and quantify over them. The final result is the \textbf{principal type} $\mathcal{P}$ of $e$. The 
    quantifiers indicate that $e$ is polymorphic (ready to receive a type as an argument to propagate). It may also be the case that $e$ is monomorphic.\\
    \noindent
    Finally, we then add $\mathcal{P}$ back to our context $\Gamma$ as a new variable $x : \mathcal{P}$. In short: 
    \begin{enumerate}
        \item $\mathcal{C}$: Find the constraints $\mathcal{C}$ and type $\tau$, such that $\Gamma \vdash e : \tau \dashv \mathcal{C}$.
        \item $\mathcal{S}$: Solve the unification problem $\mathcal{C}$ to find a unifier $\mathcal{S}$. Throw a type error on failure.
        \item $FV( \mathcal{S}\tau$): Find $\mathcal{S}\tau$ and its free type variables
        \item $\forall \alpha_1, \ldots, \alpha_k . \mathcal{S}\tau$: Quantify over the free type variables in $\mathcal{S}\tau$ to get the principal type $\mathcal{P}$.
    \end{enumerate}    
\end{Def}

\newpage 
\noindent
Let's try this out on an example:

\begin{Example}[Finding the Principal Type]

    \noindent 
    Say you want to find the type of the OCaml expression, \texttt{fun f -> fun x -> f(x+1)}. We first tunnel our way down in the 
    derivation tree:\\ 
    
    \noindent
    \Large
    \dirtree{%
    .1 \ensuremath{\cdot \vdash \texttt{fun f -> fun x -> f(x+1)}:\ ? \dashv\ ?}.
    .2 \ensuremath{\{f:\alpha\} \vdash \texttt{fun x -> f(x+1)}:\ ? \dashv\ ?}.
    .3 \ensuremath{\{f:\alpha, x: \beta \} \vdash \texttt{f(x+1)}:\ ? \dashv\ ? }.
    .4 \ensuremath{\{f:\alpha, x: \beta \} \vdash f : \alpha \dashv \varnothing}.
    .4 \ensuremath{\{f:\alpha, x: \beta \} \vdash \texttt{(x+1)}: \texttt{int} \dashv\ ?}.
    .5 \ensuremath{\{f:\alpha, x: \beta\} \vdash \texttt{x}: \beta \dashv \varnothing}.
    .5 \ensuremath{\{f:\alpha, x: \beta\} \vdash \texttt{1}: \texttt{int}\dashv \varnothing}.
    }
    \normalsize
    
    \vspace{1em}
    \noindent
    Tunneling down, we use `?' to indicate that the top level expression is waiting for its component's types to be inferred before its inference. We 
    start by unwrapping the abstraction arguments, placing them with fresh type variables in our context. On our way down, we hit a couple of base 
    cases (monotypes) that we can infer based on our context. We continue, popping back up our tree:\\
    
    
    \Large
    \dirtree{%
    .1 \ensuremath{\cdot \vdash \texttt{fun f -> fun x -> f(x+1)}:\alpha\to(\beta\to\gamma) \dashv \mathcal{C}_1,\mathcal{C}_2}.
    .2 \ensuremath{\{f:\alpha,\} \vdash \texttt{fun x -> f(x+1)}:\beta\to\gamma \dashv \mathcal{C}_1,\mathcal{C}_2}.
    .3 \ensuremath{\{f:\alpha, x: \beta \} \vdash \texttt{f(x+1)}: \gamma \dashv \mathcal{C}_2:=\{\alpha\doteq\textcolor{red}{\texttt{int}}\to\gamma\},\mathcal{C}_1}.
    .4 \ensuremath{\{f:\alpha, x: \beta \} \vdash f : \alpha \dashv \varnothing}.
    .4 \ensuremath{\{f:\alpha, x: \beta \} \vdash \texttt{(x+1)}: \textcolor{red}{\texttt{int}} \dashv \mathcal{C}_1:=\{\beta\doteq \texttt{int}, \texttt{int} \doteq \texttt{int} \}}.
    .5 \ensuremath{\{f:\alpha, x: \beta\} \vdash \texttt{x}: \beta \dashv \varnothing}.
    .5 \ensuremath{\{f:\alpha, x: \beta\} \vdash \texttt{1}: \texttt{int}\dashv \varnothing}.
    }
    \normalsize

    \vspace{1em}
    \noindent
    Utilizing HM$^-$ from Definition (\ref{def:hml}) we make our way up. We can now confidently gather 
    the types from \texttt{(x+1)}'s children branches and construct a constraint. We then continue to pass the constraints towards the top following HM$^-$. Now we must unify the constraints provided by 
    $\mathcal{C}_1$ and $\mathcal{C}_2$:
    \begin{align*}
        \alpha &\doteq \texttt{int} \to \gamma \\
        \beta &\doteq \texttt{int}\\
        \texttt{int} &\doteq \texttt{int}
    \end{align*}
    This yields the unifier $\mathcal{S}:=\{\alpha\mapsto(\texttt{int}\to\gamma),\beta\mapsto\texttt{int}\}$. We now find $S(\alpha\to(\beta\to\gamma))$ defined as $[\texttt{int}/\beta][\texttt{int}\to\gamma/\alpha](\alpha\to(\beta\to\gamma))=((\texttt{int}\to\gamma)\to(\texttt{int}\to\gamma))$.
    We now collect all free variables $FV(S(\alpha\to(\beta\to\gamma)))=\{\gamma\}$, and generalize by quantifying over the original type,\\
    yielding, 
    $\forall\gamma.(\texttt{int}\to\gamma)\to(\texttt{int}\to\gamma)$,
    \noindent
    the principle type.

\end{Example}

\newpage 

\noindent
Now lets pass type arguments to our newly created generalized form specifying them to one form.
\begin{Def}[Specializing \& Instantiating the Principal Type]

    Given some type scheme, we can \textbf{specialize} it by substituting a monotype for quantified type variables. This is called \textbf{instantiation}, as we are instantiating the type variables to a specific type.
    For example, given the principal type 
    $$\forall\gamma.(\texttt{int}\to\gamma)\to(\texttt{int}\to\gamma)$$
    \noindent
    We can instantiate $\gamma$ to $\texttt{int}$, yielding the type $(\texttt{int}\to\texttt{int})\to(\texttt{int}\to\texttt{int})$.\\

    \noindent
    We may also do partial specialization, where we only substitute some of the type variables. For example, given the principal type:

    $$\forall\alpha.\forall\gamma.(\alpha\to\gamma)\to(\alpha\to\gamma)$$

    \noindent
    We can instantiate $\alpha$ to $\beta$, and $\gamma$ to $\texttt{int}$, yielding the type $\forall\beta.(\beta\to\texttt{int})\to(\beta\to\texttt{int})$.\\

    \noindent
    Formally, we can write this as:

    \begin{center}
        \begin{prooftree}
            \hypo{\tau_1, \ldots, \tau_m \text{ are monotypes}}
            \hypo{\tau' = [\tau_m/\alpha_m] \ldots [\tau_1/\alpha_1] \tau}
            \hypo{\beta_1, \ldots, \beta_n \not\in FV(\tau) \setminus \{\alpha_1, \ldots, \alpha_m\}}
            \infer3{\forall \alpha_1 \ldots \forall \alpha_m .\ \tau \sqsubseteq \forall \beta_1 \ldots \forall \beta_n .\ \tau'}
        \end{prooftree}
    \end{center}

    \noindent
    Rewriting the above using the new notation, we have:
    \begin{align*}
        \forall\gamma.(\texttt{int}\to\gamma)\to(\texttt{int}\to\gamma) &\sqsubseteq (\texttt{int}\to\texttt{int})\to(\texttt{int}\to\texttt{int}) \\
        \forall\alpha.\forall\gamma.(\alpha\to\gamma)\to(\alpha\to\gamma) &\sqsubseteq \forall\beta.(\beta\to\texttt{int})\to(\beta\to\texttt{int}) \\
        &\sqsubseteq (\texttt{bool}\to\texttt{int})\to(\texttt{bool}\to\texttt{int}) \\
    \end{align*}

    \noindent
    The above example finishes our partial specialization by instantiating $\beta$ to $\texttt{bool}$.
\end{Def}

\noindent
For emphasis, order matters in the applying a unifier:
\begin{Example}[Order Matters]

    \noindent
    Given the unification application $\{\beta\mapsto\alpha,\alpha\mapsto\texttt{bool}\}\beta$. The correct order:

    $$
        [\texttt{bool}/\alpha][\alpha/\beta]\beta\quad\to\quad[\texttt{bool}/\alpha]\alpha\quad\to\quad\texttt{bool}
    $$

    \noindent
    If we mix the order, we get a completely different result (wrong):
    $$
        [\alpha/\beta][\texttt{bool}/\alpha]\beta\quad\to\quad [\alpha/\beta]\beta\quad\to\quad\alpha
    $$
\end{Example}

\newpage 

\noindent
Before we mentioned that the let binding in HM$^-$ is not polymorphic:
\begin{Example}[Showing Let is not Polymorphic]

\label{ex:let-not-poly}
    
Given the expression, \texttt{let f = $\lambda$x.x in f (f\ 2 = 2)}. We can see that the let binding does not have a principal type:

\vspace{1em}
\Large
\dirtree{%
.1 \ensuremath{\cdot \vdash \texttt{let f = $\lambda$x.x in f (f\ 2 = 2)}:\gamma \dashv \mathcal{C}_1,\mathcal{C}_2,\mathcal{C}_3}.
.2 \ensuremath{\cdot \vdash \texttt{$\lambda$x.x} : \alpha\to\alpha \dashv \varnothing}.
.2 \ensuremath{\{f:\alpha\to\alpha\}\vdash f\,(f\ 2 = 2): \gamma \dashv \mathcal{C}_3:=\{\;\alpha\to\alpha \doteq \texttt{bool}\to\gamma\},\mathcal{C}_1,\mathcal{C}_2}.
.3 \ensuremath{\{f:\alpha\to\alpha\}\vdash f : \alpha\to\alpha \dashv \varnothing}.
.3 \ensuremath{\{f:\alpha\to\alpha\}\vdash f\ 2 = 2 : \texttt{bool} \dashv \mathcal{C}_2:=\{\beta\doteq\texttt{int}\;\},\mathcal{C}_1}.
.4 \ensuremath{\{f:\alpha\to\alpha\}\vdash f\ 2 : \beta \dashv \mathcal{C}_1:=\{\;\alpha\to\alpha \doteq \texttt{int}\to\beta\;\}}.
.5 \ensuremath{\{f:\alpha\to\alpha\}\vdash f : \alpha\to\alpha \dashv \varnothing}.
.5 \ensuremath{\{f:\alpha\to\alpha\}\vdash 2 : \texttt{int} \dashv \varnothing}.
.4 \ensuremath{\{f:\alpha\to\alpha\}\vdash 2 : \texttt{int} \dashv \varnothing}.
}
\normalsize

\vspace{1em}
\noindent
Now onto the unification problem:
\begin{align*}
    S&:=\{\}\\
    \alpha\to\alpha &\doteq \texttt{int}\to\beta\\
    \beta &\doteq \texttt{int}\\
    \alpha\to\alpha &\doteq \texttt{bool}\to\gamma
\end{align*}

\noindent
We see that this requires $\alpha$ to be both $\texttt{int}$ and $\texttt{bool}$, which is impossible. Therefore, we have a type error.
The problem is that the let binding ($\lambda x.x$) is not polymorphic. The moment we use it for one type, we cannot re-assign it to another type.
As first we use it for $\texttt{f 2: int}$, and then we try to use it for $\texttt{f (f\ 2 = 2): bool}$. This is not allowed in HM$^-$.
\end{Example}

\begin{Def}[Polymorphic Let Binding]

    For let bindings to be polymorphic, we generalize their binding first, adding it to the context:
    \begin{center}
    
    
    \begin{prooftree}
        \hypo{%
          \shortstack[c]{%
            $\Gamma \vdash e_{1} : \tau_{1} \dashv \mathcal{C}_{1}$\\
            $S = \mathrm{Unify}(\mathcal{C}_{1})$\\
            $\displaystyle
               \sigma = \forall\overline\alpha.\,S(\tau_{1})
               \quad\bigl(\overline\alpha = FV(S(\tau_{1}))\setminus FV(\Gamma)\bigr)$\\
            $\Gamma,\,x:\sigma \;\vdash\; e_{2} : \tau_{2} \dashv \mathcal{C}_{2}$
          }
        }
        \infer1[(let-poly)]
          {%
            \Gamma \vdash \texttt{let }x=e_{1}\;\texttt{in}\;e_{2}
            : \tau_{2} \dashv \mathcal{C}_{2}
          }
      \end{prooftree}
    \end{center}

    \noindent
    Upon calling $x$ from the context, we instantiate its typing scheme $\sigma$ with fresh variable types:
    \begin{center}
    \begin{prooftree}
        \hypo{(x : \forall \alpha_1\ldots\forall \alpha_k.\,\tau) \in \Gamma}
        \hypo{\beta_1,\dots,\beta_k\ \text{fresh}}
        \infer2[(var-inst)]
          {\Gamma \vdash x : [\beta_1/\alpha_1]\dots[\beta_k/\alpha_k]\tau \dashv \varnothing}
      \end{prooftree}   
    \end{center}   
    \end{Def}
    
\newpage 

\noindent
Now lets try Example (\ref{ex:let-not-poly}) again with the new polymorphic let binding:
\begin{Example}[Polymorphic Let Binding]

    \noindent
    We find the principal type of the expression \texttt{let f = $\lambda$x.x in f (f\ 2 = 2)}:

    \vspace{1em}
\Large
\dirtree{%
.1 \ensuremath{\cdot \vdash \texttt{let f = $\lambda$x.x in f (f\ 2 = 2)}:\gamma \dashv \mathcal{C}_1,\mathcal{C}_2,\mathcal{C}_3}.
.2 \ensuremath{\cdot \vdash \texttt{$\lambda$x.x} : \alpha\to\alpha \dashv \varnothing}.
.2 \ensuremath{\{f:\forall\alpha.\alpha\to\alpha\}\vdash f\,(f\ 2 = 2): \gamma \dashv \mathcal{C}_3:=\{\;\textcolor{blue}{\tau_1\to\tau_1} \doteq \texttt{bool}\to\gamma\},\mathcal{C}_1,\mathcal{C}_2}.
.3 \ensuremath{\{f:\forall\alpha.\alpha\to\alpha\}\vdash f : \textcolor{blue}{\tau_1\to\tau_1} \dashv \varnothing}.
.3 \ensuremath{\{f:\forall\alpha.\alpha\to\alpha\}\vdash f\ 2 = 2 : \texttt{bool} \dashv \mathcal{C}_2:=\{\beta\doteq\texttt{int}\;\},\mathcal{C}_1}.
.4 \ensuremath{\{f:\forall\alpha.\alpha\to\alpha\}\vdash f\ 2 : \beta \dashv \mathcal{C}_1:=\{\;\textcolor{red}{\tau_2\to\tau_2} \doteq \texttt{int}\to\beta\;\}}.
.5 \ensuremath{\{f:\forall\alpha.\alpha\to\alpha\}\vdash f : \textcolor{red}{\tau_2\to\tau_2} \dashv \varnothing}.
.5 \ensuremath{\{f:\forall\alpha.\alpha\to\alpha\}\vdash 2 : \texttt{int} \dashv \varnothing}.
.4 \ensuremath{\{f:\forall\alpha.\alpha\to\alpha\}\vdash 2 : \texttt{int} \dashv \varnothing}.
}
\normalsize

\vspace{1em}
  
\noindent
Now onto the unification problem:
\begin{align*}
    \tau_2\to\tau_2 &\doteq \texttt{int}\to\beta\\
    \beta &\doteq \texttt{int}\\
    \tau_1\to\tau_1 &\doteq \texttt{bool}\to\gamma
\end{align*}

\noindent
After decomposition, we have:
\begin{align*}
    \tau_2&\doteq\texttt{int}\\
    \tau_2 &\doteq \beta\\
    \beta &\doteq \texttt{int}\\
    \tau_1&\doteq\texttt{bool}\\
    \tau_1 &\doteq \gamma
\end{align*}

\noindent
Which gives us a unifier $\mathcal{S}:=\{\tau_2\mapsto\texttt{int},\beta\mapsto\texttt{int},\tau_1\mapsto\texttt{bool},\gamma\mapsto\texttt{bool}\}$. We can now find the principal type:

\[
[\texttt{bool}/\gamma][\texttt{bool}/\tau_1][\texttt{int}/\beta][\texttt{int}/\tau_2]\gamma
\]

\noindent
Hence, the principal type is ($\texttt{bool}$) after all other substitutions are ignored before the final matching substitution
$[\texttt{bool}/\gamma]\gamma$.
\end{Example}

\newpage 
\noindent
One last example showing what happens when there are existing type variables in the context:
\begin{Example}[Monomorphic \texttt{let}-binding due to context variable]
    
    Let $\Gamma = \{g: \gamma \to \texttt{int}\}$, and an expression $e:=$\texttt{let h = fun x -> g x in h (42)}. Then,\\

    \Large
    \dirtree{%
    .1 \ensuremath{\{g: \gamma \to \texttt{int}\} \vdash \texttt{let h = fun x -> g x in h (42)}:\ ? \dashv\ ?}.
    .2 \ensuremath{\{g: \gamma \to \texttt{int}\} \vdash \texttt{fun x -> g x}: \alpha\to\beta \dashv \mathcal{C}_1}.
    .3 \ensuremath{\{g: \gamma \to \texttt{int}, x: \alpha\} \vdash \texttt{g x}: \beta \dashv \mathcal{C}_1:=\{\gamma \to \texttt{int}\doteq \alpha\to\beta\}}. 
    .4 \ensuremath{\{g: \gamma \to \texttt{int}, x: \alpha\} \vdash \texttt{g}: \gamma \to \texttt{int} \dashv \varnothing}. 
    .4 \ensuremath{\{g: \gamma \to \texttt{int}, x: \alpha\} \vdash \texttt{x}: \alpha \dashv \varnothing}. 
    }

    \normalsize
    \vspace{1em}
    \noindent
    We solve the unification problem $\{\gamma \to \texttt{int} \doteq \alpha\to\beta\}$ which yields $\mathcal{S}:=\{\alpha\mapsto\gamma, \beta\mapsto\texttt{int}\}$. Then 
    $$S(\alpha\to\beta):=[\texttt{int}/\beta][\gamma/\alpha](\alpha\to\beta)=(\gamma\to\texttt{int})$$

    \noindent
    We can now find the free type variables $FV((\gamma\to\texttt{int}))\setminus FV(\Gamma)=\{\}$.
    This generalizes the binding to $(\gamma\to\texttt{int})$. We add it to the context and continue:\\
        
    \Large
    \dirtree{%
    .1 \ensuremath{\{g: \gamma \to \texttt{int}\} \vdash \texttt{let h = fun x -> g x in h (42)}: \sigma \dashv \mathcal{C}_2}.
    .2 \textcolor{gray}{\ensuremath{\{g: \gamma \to \texttt{int}\} \vdash \texttt{fun x -> g x}: \alpha\to\beta \dashv \mathcal{C}_1}}.
    .3 \textcolor{gray}{\ensuremath{\{g: \gamma \to \texttt{int}, x: \alpha\} \vdash \texttt{g x}: \beta \dashv \mathcal{C}_1:=\{\gamma \to \texttt{int}\doteq \alpha\}}}. 
    .4 \textcolor{gray}{\ensuremath{\{g: \gamma \to \texttt{int}, x: \alpha\} \vdash \texttt{g}: \gamma \to \texttt{int} \dashv \varnothing}}. 
    .4 \textcolor{gray}{\ensuremath{\{g: \gamma \to \texttt{int}, x: \alpha\} \vdash \texttt{x}: \alpha \dashv \varnothing}}. 
    .2 \ensuremath{\{g: (\dots), h :(\gamma\to\texttt{int})\} \vdash \texttt{h (42)}: \sigma \dashv \mathcal{C}_2:=\{(\gamma\to\texttt{int})\doteq \texttt{int}\to\sigma\}}.
    .3 \ensuremath{\{g: (\dots), h :(\gamma\to\texttt{int})\} \vdash \texttt{h}: (\gamma\to\texttt{int}) \dashv \varnothing}.
    .3 \ensuremath{\{g: (\dots), h :(\gamma\to\texttt{int})\} \vdash \texttt{(42)}: \texttt{int} \dashv \varnothing}.
    }
    \normalsize

    \vspace{1em}
    \noindent
    We solve the unification problem $\{(\gamma\to\texttt{int})\doteq (\texttt{int}\to\sigma)\}$ which yields\\
    $$\mathcal{S}:=\{\gamma\mapsto\texttt{int},\sigma\mapsto\texttt{int}\}$$
    
    \noindent
    Then, 
    $$S(\sigma):=[\texttt{int}/\sigma][\texttt{int}/\gamma](\sigma)=\texttt{int}$$
    
    \noindent
    Hence, (\texttt{int}) is the principal type of $e$. 
  \end{Example}
  