
        
\newpage 

\subsection{Handling Lambda Recursion}
\noindent
We have $\Omega$ which allows us to do recursion, but we need self-referencing.

\begin{Def}[Fixed-point Combinator]
    
    A \textbf{fixed point} is a value unchanged by a transformation (e.g.,
    the fixed point of $f$ is some value $x$ such that, $f(x) = x$). A fixed-point combinator is a higher-order function that
    satisfies:
    \LARGE
    \[
     \texttt{FIX}\ f = f (\texttt{FIX} f)
    \]

    \normalsize
    \noindent
    i.e., functions \texttt{FIX} and $f$ when applied returns $f$ 
    whose argument is the original application. This enables recursion, as there is a 
    self reference in scope. This unfolds to an infinite series of applications:
    ($\texttt{FIX}\ f = f (\texttt{FIX} f)= f(f(\texttt{FIX} f)) = f(f(f(\texttt{FIX} f))) = \ldots$).
    Whether or not it converges depends on the behavior of $f$ (i.e., a base-case).
\end{Def}

\begin{Example}[Writing Recursive Functions]

    \label{ex:recursion}
    Say we defined the following recursive factorial function, extending our lambda syntax:
    \Large
    \[
        \texttt{FACT} \triangleq \lambda n. \texttt{if } n = 0 \texttt{ then } 1 \texttt{ else } n * \texttt{FACT}(n - 1)
    \]
    \normalsize
    \noindent
    To supply \texttt{FACT} with its own definition, we may preform an intermediary step:
    \Large
    \[
        \texttt{FACT'} \triangleq \lambda f. \lambda n. \texttt{if } n = 0 \texttt{ then } 1 \texttt{ else } n * (f f(n - 1))
    \]
    \normalsize
    We define \texttt{FACT'}, which takes an additional function $f$ to supply its recursive case. Now, we can apply \texttt{FACT'} to itself to render 
    our desired \texttt{FACT} function: 
    
    \vspace{-1em}
    \Large
    \[
        \texttt{FACT} \triangleq \texttt{ FACT' FACT'}
    \]
    \normalsize
    \noindent
    For example, let's supply 3 to \texttt{FACT}:
    \begin{align*}
        \texttt{FACT}\ 3 
        &= (\texttt{FACT}'\ \texttt{FACT}')\ 3 && \text{Definition of FACT} \\
        &= ((\lambda f.\, \lambda n.\, \texttt{if } n = 0 \texttt{ then } 1 \texttt{ else } n \times (f\ f\ (n - 1)))\ \texttt{FACT}')\ 3 && \text{Definition of FACT'} \\
        &\to (\lambda n.\, \texttt{if } n = 0 \texttt{ then } 1 \texttt{ else } n \times (\texttt{FACT}'\ \texttt{FACT}'\ (n - 1)))\ 3 && \text{Application to FACT'} \\
        &\to \texttt{if } 3 = 0 \texttt{ then } 1 \texttt{ else } 3 \times (\texttt{FACT}'\ \texttt{FACT}'\ (3 - 1)) && \text{Application to } n \\
        &\to 3 \times (\texttt{FACT}'\ \texttt{FACT}'\ (3 - 1)) && \text{Evaluating \texttt{if}} \\
        &\to \ldots \\
        &\to 3 \times 2 \times 1 \times 1 \\
        &\to^* 6 && \cite{CS4110Lecture17}
        \end{align*}
        

\end{Example}

\newpage 

\noindent
We make the following distinction to emphasize the meaning of a fixed-point:
\begin{theo}[The identity function \& fixed-points]

    Any function $f$ is a fixed-point of the identity function $I$ $(\lambda x.x)$, i.e., $I\ f = f$.
\end{theo}

\noindent
Our previous implementation of \texttt{FACT} in Example (\ref{ex:recursion}) was manual. This would be 
quite tedious for every recursive function. We can automate this with the following fixed-point combinator:
\begin{Def}[Y-Combinator]

    \label{def:y-combinator}
    In lambda calculus, the \textbf{Y combinator} is a fixed-point combinator of form:
    \LARGE
    \[
        \texttt{Y} \triangleq \lambda f.\, (\lambda x.\, f (x\ x))\, (\lambda x.\, f (x\ x))
    \]
    \normalsize
    \noindent
    \textbf{E.g.,} $Y\ f = (\lambda x.\, f (x\ x))\, (\lambda x.\, f (x\ x))$ = $f((\lambda x.f(xx)(\lambda x.f(xx))))=f(f(\dots))=\dots$.

\end{Def}

\begin{Example}[Factorial with Y-Combinator]

    \label{ex:factorial-y}
    We can now define \texttt{FACT} using the Y combinator:
    \begin{align*}
    \texttt{FACT} &\triangleq \lambda f.\lambda n. \texttt{if } n = 0 \texttt{ then } 1 \texttt{ else } n * (f(n - 1)) && \text{( Definition of FACT' )} \\
    \texttt{Y FACT } 3 & = ((\lambda x.\, \texttt{FACT} (x\ x))\, (\lambda x.\, \texttt{FACT} (x\ x)))\ 3 && (\ [\texttt{FACT}/f]\texttt{Y}\ )\\
    & = \texttt{FACT}\ ((\lambda x.\, \texttt{FACT} (x\ x))\ (\lambda x.\, \texttt{FACT} (x\ x)))\ 3 && (\ [\lambda x.\, \texttt{FACT} (x\ x)/x] \texttt{FACT} (x\ x)\ ) \\
    \end{align*}

    \vspace{-1em}
   \noindent
   Remember that \texttt{FACT} still requires two arguments $f$ and $n$, for which we now supply:
    \begin{align*}
        &= \texttt{if } 3 = 0 \texttt{ then } 1 \texttt{ else } 3 * ((\lambda x.\, \texttt{FACT} (x\ x))\ (\lambda x.\, \texttt{FACT} (x\ x))(3 - 1))\\
        &= \texttt{if } 3 = 0 \texttt{ then } 1 \texttt{ else } 3 * \texttt{FACT} ((\lambda x.\, \texttt{FACT} (x\ x))\ (\lambda x.\, \texttt{FACT} (x\ x)))\ (3 - 1)\\
        &= \texttt{if } 3 = 0 \texttt{ then } 1 \texttt{ else } 3 * (\texttt{Y FACT}(3 - 1)) \\
        &\vdots \\
        &= \texttt{if } 2 = 0 \texttt{ then } 1 \texttt{ else } 2 * (\texttt{Y FACT}(2 - 1)) \\
        &= \texttt{if } 1 = 0 \texttt{ then } 1 \texttt{ else } 1 * (\texttt{Y FACT}(1 - 1)) \\
        &= \texttt{if } 0 = 0 \texttt{ then } 1 \texttt{ else } 0 * (\texttt{Y FACT}(0 - 1)) \\
    \end{align*}
    \noindent
    We hit the base-case and then evaluate $3 * (2 * (1 * 1))$ to get $6$.
\end{Example}
\newpage

\noindent
If we aren't careful step three of our derivation in Example (\ref{ex:factorial-y}) could lead to an infinite loop:

\begin{Def}[Strict vs. Lazy Evaluation]

    \label{def:strict-vs-lazy}
    \noindent
    \textbf{Strict evaluation} means that all arguments to a function are evaluated before the function is applied, i.e., CBV (call-by-value).

    \noindent
    \textbf{Lazy evaluation} means that an argument to a function is not evaluated until it is actually used in the body of the function, i.e., CBN (call-by-name).
\end{Def}

\begin{theo}[Y-Combinator \& Lazy-Evaluation]

    \label{theo:y-combinator-lazy}
    \noindent
    The Y-combinator only works in lazy-evaluation settings. In strict evaluation, the Y-combinator will infinitely reduce.
\end{theo}

\noindent
To stop this we introduce another type of combinator for eager evaluation:
\begin{Def}[Z-Combinator]
    
        \label{def:z-combinator}
        \noindent
        The Z-combinator is a fixed-point combinator that works in strict evaluation settings:
        \LARGE
        \[
        Z \;\triangleq\; 
        \lambda f.\,\bigl(\lambda x.\,f\bigl(\lambda v.\,x\,x\,v\bigr)\bigr)
                    \bigl(\lambda x.\,f\bigl(\lambda v.\,x\,x\,v\bigr)\bigr).
        \]
        \normalsize
\end{Def}

\begin{Example}[Factorial with Z-Combinator (Part-1)]
    
    We now define the \texttt{FACT} using the Z combinator:
    \begin{align*}
        \texttt{Z FACT 3} & = ((\lambda x.\, \texttt{FACT} (\lambda v.x\ x\ v))\, (\lambda x.\, \texttt{FACT} (\lambda v.x\ x\ v)))\ 3 && (\ [\texttt{FACT}/f]\texttt{Z}\ )\\
        & = \texttt{FACT} (\lambda v.(\lambda x.\, \texttt{FACT} (\lambda v.x\ x\ v))\ (\lambda x.\, \texttt{FACT} (\lambda v.x\ x\ v)\ v))\ 3 && ( \text{ Application }) \\
    \end{align*}

    \vspace{-1em}
    \noindent
    The extra argument waiting for a value delays \texttt{Z} long enough to evaluate \texttt{FACT}:
    \begin{align*}
        &= \texttt{if } 3 = 0 \texttt{ then } 1 \texttt{ else } 3 * (\lambda v.(\lambda x.\, \texttt{FACT} (\lambda v.x\ x\ v))\ (\lambda x.\, \texttt{FACT} (\lambda v.x\ x\ v))\ v)\ (3 - 1))) \\
        &= \texttt{if } 3 = 0 \texttt{ then } 1 \texttt{ else } 3 * (\lambda v.(\lambda x.\, \texttt{FACT} (\lambda v.x\ x\ v))\ (\lambda x.\, \texttt{FACT} (\lambda v.x\ x\ v))\ v)\ (2))) \\
        &= \texttt{if } 3 = 0 \texttt{ then } 1 \texttt{ else } 3 * (\lambda x.\, \texttt{FACT} (\lambda v.x\ x\ v))\ (\lambda x.\, \texttt{FACT} (\lambda v.x\ x\ v))\ 2 \\
        &= \texttt{if } 3 = 0 \texttt{ then } 1 \texttt{ else } 3 * \texttt{FACT} (\lambda v.(\lambda x.\, \texttt{FACT} (\lambda v.x\ x\ v))\ (\lambda x.\, \texttt{FACT} (\lambda v.x\ x\ v))\ v)\ \ 2 \\
        &= \texttt{if } 3 = 0 \texttt{ then } 1 \texttt{ else } 3 * (\texttt{Z FACT} \ 2)
    \end{align*}
    \noindent
    This assumes that truthy \texttt{if} expressions don't evaluate the \texttt{else} branch.
\end{Example}
\noindent

\newpage 

\noindent
Last example touches on the idea of short-circuiting:
\begin{Def}[Short-Circuiting]

    \label{def:short-circuiting}
    \noindent
    \textbf{Short-circuiting} is a semantic trick which skips additional computation of a boolean expressions 
    if some former part of the expression is sufficient to determine the value of the entire expression.
    For example ($\mathbb{B}$ is the set of booleans):
    
    \vspace{-1em}
  
        \[
        \begin{matrix}
        \begin{prooftree}
            \hypo{e_1 \Downarrow \bot}
            \infer1[(andEvalFalse)]{e_1 \;\&\&\; e_2 \Downarrow \bot}
        \end{prooftree}
        &
        \begin{prooftree}
            \hypo{e_1 \Downarrow \top}
            \hypo{e_2 \Downarrow v,\; v \in \mathbb{B}}
            \infer2[(andEvalTrue)]{e_1 \;\&\&\; e_2 \Downarrow v}
        \end{prooftree}
        \vspace{1em}\\ 

        \begin{prooftree}
            \hypo{e_1 \Downarrow \top}
            \infer1[(orEvalTrue)]{e_1 \;||\; e_2 \Downarrow \top}
        \end{prooftree}
        &
        \begin{prooftree}
            \hypo{e_1 \Downarrow \bot}
            \hypo{e_2 \Downarrow v,\; v \in \mathbb{B}}
            \infer2[(orEvalFalse)]{e_1 \;||\; e_2 \Downarrow v}
        \end{prooftree}
        \vspace{1em} \\ 

        \begin{prooftree}
            \hypo{e_1 \Downarrow \top}
            \hypo{e_2 \Downarrow v}
            \infer2[(ifTrueEval)]{\text{if } e_1 \text{ then } e_2 \text{ else } e_3 \Downarrow v}
        \end{prooftree}
        &
        \begin{prooftree}
            \hypo{e_1 \Downarrow \bot}
            \hypo{e_3 \Downarrow v}
            \infer2[(ifFalseEval)]{\text{if } e_1 \text{ then } e_2 \text{ else } e_3 \Downarrow v}
        \end{prooftree}
        \end{matrix}
        \]

    \noindent
    Notice that in (andEvalFalse), (orEvalTrue), and (ifTrueEval) the second expression is never evaluated. This is a form of \textbf{short-circuiting}.
    \end{Def}
