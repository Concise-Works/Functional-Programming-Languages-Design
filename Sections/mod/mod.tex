\section{Modules In Ocaml}

This section details how OCaml deals with modular programming, including abstractions and interfaces.

\begin{Def}[Modular Programming]

Modular programming is a software design approach that emphasizes separating a program's functionality into independent, interchangeable modules,
which are composed of three key elements:

\begin{enumerate}
    \item \textbf{Namespaces:} A way of separating code into logical units of functions, types, and values together while avoiding name conflicts.
    
    \item \textbf{Abstraction/Encapsulation:} A way of abstracting away implementation details and organizing core functionality. This creates a clear boundary between the module's intent and its implementation for clarity.
    
    \item \textbf{Code Reuse:} Well-designed modules can serve as reusable components across multiple projects, reducing duplication.
\end{enumerate}

\end{Def}

\begin{Def}[The (\texttt{module}) \& (\texttt{struct}) Keyword in OCaml]

    The \textbf{module} keyword in OCaml is used to define a collection of related code elements (types, values, functions) that are grouped together into a single namespace.

    \begin{lstlisting}[language=OCaml, numbers=none, caption= \textbf{Basic Module Syntax:}]
    module ModuleName = struct
    (* Types, values, and functions *)
    type t = int * int
    let create x y = (x, y)
    let add (x1, y1) (x2, y2) = (x1 + x2, y1 + y2)
    end
    \end{lstlisting}

    \noindent
    Where the \snippet{struct} keyword defines the collection of
    definitions under the module.
    Once defined, module elements are accessed using the dot notation:
    \begin{lstlisting}[language=OCaml, numbers=none]
    let point = ModuleName.create 10 20
    let sum = ModuleName.add point point
    \end{lstlisting}

    \noindent
    Multiple modules may be defined in a single file, and be used in and between other files.

\end{Def}

\newpage 

\begin{Def}[Module Access: (\texttt{open}) and Local Opens]
    OCaml provides multiple ways to access module contents:\\

    \noindent
    \textbf{Qualified access:} uses dot notation: \snippet{ModuleName.function\_name}\\
        
    \noindent
    \textbf{Global open:} brings all module contents into the current scope:
        \begin{lstlisting}[language=OCaml, numbers=none]
    (* All List functions now available without qualification *)
    open List  
    let x = map (fun x -> x * 2) [1; 2; 3]  (* No need for List.map *)
        \end{lstlisting}
        
    \noindent
    \textbf{Local open:} provides temporary access within a limited scope:
        \begin{lstlisting}[language=OCaml, numbers=none]
    (* Using Module.(expr) syntax *)
    let result = List.(
    map (fun x -> x * 2) [1; 2; 3]  
    (* List is open only in this scope *)
    )

    (* Outside the parentheses, module is not opened *)
    let standard = List.length [1; 2; 3]  (* Need qualification again *)
        \end{lstlisting}
 
\end{Def}

\begin{Def}[Module Signatures: (\texttt{sig}) \& (\texttt{module type})]
    
    Signatures are interfaces to modules. They 
    are created in \snippet{.mli} files:

    \begin{lstlisting}[language=OCaml, numbers=none]
    module type POINT = sig
      (* Abstract type - implementation hidden *)  
      type t              
      val private create : int -> int -> t
      val add : t -> t -> t
    end
    \end{lstlisting}

    \noindent
    The \snippet{val} keyword explains rather than defines like \snippet{let}. Signatures are then applied to modules to ensure they conform to the interface:
    
    \begin{lstlisting}[language=OCaml, numbers=none]
    module Point : POINT = struct
      type t = int * int
      let create x y = (x, y)
      let add (x1, y1) (x2, y2) = (x1 + x2, y1 + y2)
      
      (* This function is private as its not in the signature *)
      let sub x y = x - y 
    end
    \end{lstlisting}
\end{Def}

\begin{Def}[Module Helper Pattern for Data Structures]
    
    A common pattern in OCaml is to create helper functions inside modules to simplify working with complex data structures. 
    This approach makes code more concise and readable by providing shorthand notations for constructors.\\

    \noindent
    For example, with a binary tree:
    
    \begin{lstlisting}[language=OCaml, numbers=none]
    type 'a tree =
      | Leaf
      | Node of 'a * 'a tree * 'a tree

    module TreeExample = struct
      let l = Leaf
      let n l r = Node ((), l, r)
    end
    
    (* Using local open for concise tree construction *)
    let example = TreeExample.(n (n (n l l) l) (n l l))
    \end{lstlisting}
    
    \noindent
    Instead of writing the full constructor names repeatedly, the module provides shorthand aliases (\snippet{l} for \snippet{Leaf} and \snippet{n} for creating \snippet{Node}s). Combined with local opens, this makes tree construction much more readable than the equivalent:
    \begin{lstlisting}[language=OCaml, numbers=none]
    let example = Node ((), Node ((), Node ((), Leaf, Leaf), Leaf), Node ((), Leaf, Leaf))
    \end{lstlisting}
\end{Def}