\subsection{Environments: Variable Binding Data-Structure}
\noindent
Though Y and Z combinators allow us to write recursive functions, this method quickly grows unwieldy as the complexity of our programs increases.
Things like variable bindings and jumping between scopes become difficult to manage. That's where environments come in:
\begin{Def}[Environment]

    \label{def:environment}

    \noindent
    An \textbf{environment} is a data-structure that keeps track of \textbf{variable bindings}, i.e., associations between variables and their corresponding values. Environments are written as finite mappings:
    \LARGE
    \[
        \{ x \mapsto v,\ y \mapsto w,\ z \mapsto f \}
    \]
    \normalsize
    Where each variable is mapped to a value. We may use such data-structure for state configurations. For example $\langle\{x\mapsto \lambda y.y\}, x  \rangle \Downarrow \lambda y.y$. We shall denote environments as $\mathcal{E}$.
    Though we track state in configurations, the program itself is \textbf{still functionally pure}.
\end{Def}
\noindent

\begin{theo}[Substitution vs. Environment Model]

    When we care about the speed of our program, the substitution model quickly becomes inefficient.
    This is because we have to read our entire program to find free variables and handle additional logic.
\end{theo}

\newpage
\noindent
Here are the following operations we can preform on environments:
\begin{Def}[Operations on Environments]

   \label{def:env-operations}

   \noindent
   Environments support basic operations for managing variable bindings, similar to a map:

   \begin{itemize}
       \item \(\varnothing\) — represents the empty environment (OCaml: \texttt{empty}).
       \item $\langle\mathcal{E} \rangle$ — represents the current environment (OCaml: \texttt{env}).
       \item $\langle\mathcal{E}[x \mapsto v] \rangle$ — adds a new binding of variable \(x\) to value \(v\) (OCaml: \texttt{add x v env}).
       
       \item $\langle\mathcal{E}(x) \rangle$ — looks up the value of variable \(x\) (OCaml: \texttt{find\_opt x env}).
       
       \item $\langle\mathcal{E}(x) = \bot \rangle$ — indicates that \(x\) is unbound in the environment\\ (OCaml: \texttt{find\_opt x env = None}).
   \end{itemize}

   \noindent
   Additionally, if a new binding is added for a variable that already exists, the new binding \textbf{shadows} the old one:
   \[
   \mathcal{E}[x \mapsto v][x \mapsto w] = \mathcal{E}[x \mapsto w]
   \]
\end{Def}

\begin{Def}[Semantic Closures]

  A snapshot of the environment and its bindings is called a \textbf{closure}. There are two types:
  \[
      \underbracket{\llp\mathcal{E}, \cdot \mapsto \lambda x.\,e\rrp}_{\text{unnamed}} \quad \text{and} \quad \underbracket{\llp \mathcal{E}, f\mapsto\lambda x.\,e\rrp}_{\text{named}}
  \]
  \textbf{Unnamed and named closures}: The former captures the environment and program. The 
  latter includes the function name for safe self-referencing. \underline{Closures are \textbf{values}.}
\end{Def}
\begin{Example}[Extended Lambda Calculus (Part-1)]

We momentarily create a grammar, \textbf{Lambda Calculus$^+$ (LC$^+$)}, for demonstration:

\begin{lstlisting}[numbers=none, mathescape=true,escapeinside={(*}{*)}]
    <expr> ::= <expr><expr>
            | let <var> = <expr> in <expr>
            | let rec <var> <var> = <expr> in <expr>
            | <val>
    <var> ::= [a-zA-Z]
    <val> ::= (*$\lambda$*)<var>.<expr> | <num>
    \end{lstlisting}

\end{Example}        

\newpage 

\noindent
Let's give \textbf{LC$^+$} some semantics:
\begin{Example}[LC$^+$ Semantics (Part-2)]

    \textbf{Values and variables}

\[
\begin{matrix}
% (E, λx.e) ⇓ (E, λx.e)
\begin{prooftree}
  \infer0{
    \langle\mathcal{E}, \lambda x.\,e\rangle \Downarrow \llp \mathcal{E}, \lambda x.\,e\rrp
  }
\end{prooftree}
&
% (E, x) ⇓ E(x)
\begin{prooftree}
    \hypo{\langle\mathcal{E}, x\rangle \neq \bot}
  \infer1{
    \langle\mathcal{E}, x\rangle \Downarrow \mathcal{E}(x)
  }
\end{prooftree}
&
% (E, n) ⇓ n
\begin{prooftree}
  \infer0{
    \langle\mathcal{E}, n\rangle \Downarrow n
  }
\end{prooftree}
&
\end{matrix}
\]


\noindent
\textbf{Let expressions}

\[
\begin{matrix}
% let f = e1 in e2
\begin{prooftree}
  \hypo{\langle\mathcal{E}, e_{1}\rangle \Downarrow v_{1}}
  \hypo{\langle\mathcal{E}[\,x \mapsto v_{1}],\; e_{2}\rangle \Downarrow v_2}
  \infer2{
    \langle\mathcal{E},\; \text{let}\; x = e_{1}\; \text{in}\; e_{2}\rangle \Downarrow v_2
  }
\end{prooftree}
&
% let rec f x = e1 in e2
\begin{prooftree}
  \hypo{\langle\mathcal{E}[\,f \mapsto \llp \mathcal{E}',\;f\mapsto\lambda x.\,e_1\rrp\,],\; e_{2}\rangle \Downarrow v_2}
  \infer1{
    \langle\mathcal{E},\; \text{let rec}\; f\,x = e_{1}\; \text{in}\; e_{2}\rangle \Downarrow v_2
  }
\end{prooftree}
\end{matrix}
\]
\noindent
\textbf{Application (unnamed closure)}

\[
\begin{prooftree}
  \hypo{\langle\mathcal{E}, e_{1}\rangle \Downarrow \llp \mathcal{E}', \cdot \mapsto\lambda x.\,e\rrp}
  \hypo{\langle\mathcal{E}, e_{2}\rangle \Downarrow v_{2}}
  \hypo{\langle\mathcal{E}'[x \mapsto v_{2}],\, e\rangle \Downarrow v}
  \infer3{
    \langle\mathcal{E},\, e_{1}\,e_{2}\rangle \Downarrow v
  }
\end{prooftree}
\]
\noindent
\textbf{Application (named closure)}

\[
\begin{prooftree}
  \hypo{\langle\mathcal{E}, e_{1}\rangle \Downarrow \llp \;\mathcal{E}',\;f\mapsto \lambda x.\,e \rrp}
  \hypo{\langle\mathcal{E}, e_{2}\rangle \Downarrow v_{2}}
  \hypo{\langle\mathcal{E}'[\,f \mapsto \llp \mathcal{E}',\;f\mapsto\lambda x.\,e\rrp\,][x \mapsto v_{2}]\,e\rangle \Downarrow v}
  \infer3{
    \langle\mathcal{E},\, e_{1}\,e_{2}\rangle \Downarrow v
  }
\end{prooftree}
\]
\end{Example}

\begin{Example}[Unnamed Closure Derivation]

   Observe the program and its following single-column derivation. This derivation is 
   partly informal for teaching purposes (where $\twoheadrightarrow$ means some number of steps):
   \begin{lstlisting}[numbers=none, mathescape=true,escapeinside={(*}{*)}]
    let x = 1 in
    let f = (*$\lambda$*)y.x in
    let x = 0 in
    f
    \end{lstlisting}

    \vspace{-1em}
    \begin{align*}
      \langle\varnothing,\; \text{let } x = 1 \text{ in }  \dots \rangle&\twoheadrightarrow  \langle \{x\mapsto 1\},\; \texttt{let } f = \lambda y.x \text{ in } \dots \rangle \\
            &\twoheadrightarrow \langle \{x\mapsto 1, f\mapsto \llp\{x\mapsto 1\},\cdot\mapsto\lambda y.x\rrp\},\; \text{let } x = 0 \text{ in } f \rangle \\
            &\twoheadrightarrow \langle \{x\mapsto 0, f\mapsto \llp\{x\mapsto 1\},\cdot\mapsto\lambda y.x\rrp\},\; f \rangle \\
            &\twoheadrightarrow \llp \{x\mapsto 1\},f\mapsto\lambda y.x\rrp\\
    \end{align*}
    \noindent
    Here the final program $f$ returns the mapped value in the environment, which is a closure.
    % \begin{prooftree}
    %   \hypo{\langle\varnothing,\; 1 \rangle\Downarrow 1}
    %   \hypo{\langle\{x\mapsto 1\},\; \lambda y.x \rangle\Downarrow \llp\{x\mapsto 1\},f\mapsto\lambda y.x\rrp}
    %   \Infer0[\text{(Let)}]{\langle\varnothing,\; \text{let } x = 1 \text{ in } \dots \rangle\Downarrow \llp\{x\mapsto 1\},f\mapsto\lambda y.x\rrp}
    %   \Infer2[\text{(Let)}]{\langle\{x\mapsto 1\},\; \text{let } f = \lambda y.x \text{ in } \dots \rangle\Downarrow \llp\{x\mapsto 1\},f\mapsto\lambda y.x\rrp}
    %   \Infer2[\text{(Let)}]{\langle\varnothing,\; \text{let } x = 1 \text{ in }  \dots \rangle\Downarrow \llp\{x\mapsto 1\},\;f\mapsto\lambda y.x\rrp}
    % \end{prooftree}
  \end{Example}

\newpage
\begin{Example}[Named Closure Derivation]

    Now an example with a tree-derivation. let $\alpha:= f \mapsto \llp\{x \mapsto 0\},f\mapsto \lambda y.x\rrp$, and $\beta$ define the following premises:
    \begin{itemize}
        \item $\left\langle \{x \mapsto 1, \alpha\}, f \right\rangle \Downarrow \llp \{x \mapsto 0\},f\mapsto\lambda y.x \rrp\to\llp \{x \mapsto 0\},f\mapsto\lambda y.0 \rrp$
        \item $\left\langle \{x \mapsto 1, \alpha\}, 2 \right\rangle \Downarrow 2$
        \item $\left\langle \{x \mapsto 1, \alpha\},[y\mapsto 2]0 \right\rangle \Downarrow 0$
    \end{itemize}
    \noindent
    We derive the following:
\[
\resizebox{\textwidth}{!}{%
    \begin{prooftree}
        \hypo{\left\langle \{x \mapsto 0 \},\lambda y.x\right\rangle\Downarrow \llp\{x \mapsto 0 \},\lambda y.x\rrp}
        \hypo{\left\langle \{x \mapsto 0, \alpha\}, 1 \right\rangle \Downarrow 1}
        
        \hypo{\beta }
        \infer1[(NC)]{\left\langle \{x \mapsto 1, \alpha\}, f\; 2 \right\rangle \Downarrow 0}
        \infer2[\text{(L)}]{\left\langle \{x \mapsto 0, \alpha\},\text{let } x = 1\text{ in } f\; 2 \right\rangle \Downarrow 0}
        \infer2[\text{(L)}]{\left\langle \{ x \mapsto 0 \},\; \text{let } f = \lambda y.\,x\; \text{in let } x = 1\; \text{in } f\; 2 \right\rangle \Downarrow 0}
    \end{prooftree}
    }\]
    \\
    \noindent
    Shorthanded rule names, NC:= Named Closure, L:= Let.
\end{Example}
